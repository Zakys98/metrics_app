\chapter{Úvod}

\chapter{Rozbor řešené problematiky}

\section{Metody}

Vyhodnocování výkonu počítačů je důležitou technologii pro výzkum v oblasti počítačů. Neustálý vývoj počítačů dělá tento úkol stále stále složitější.
Obecný problém rozvoje efektivní vyhodnocovací techniky lze vyjádřit jako hledání nejlepšího kompromisu mezi přesností a rychlostí. Tento kompromis závisí na
použití vyhodnocovací metody.

\subsection{Existující řešení}

\section{Přehled nástrojů}

\subsection{Berkeley Packet Filter}

Berkeley Packet Filter nebo také BPF je technologie, která umožňuje spouštět programy v jádře operačního systému na linuxových systémech.
Efektivně a bezpečně rozšíří schopnosti jádra bez nutnosti měnit zdrojový kód jádra. Tato schopnost je často použita ke sledování využití systému či sledování síťového provozu.
BPF je k dispozici na většině unixových operačních systémů a extended BPF je také k dispozici pro Microsoft Windows.

\subsection*{Sledování síťového provozu}
BPF poskytuje rozhranní k datové lince, což umožňuje odesílání a přijímání paketů spojové vrstvy. BPF podporuje filtrování paketů a umožňuje user space(vysvětlit proč jsem to napsal anglicky) procesům,
aby si samy vyfiltrovaly pakety, které chtějí obdržet a které naopak zahodit. Toto může vést ke zvýšení výkonu operačního systému, protože se tím zamezí kopírování nepotřebných paketů.

\subsection*{Verze BPF}
BPF má více verzí. Výše zmíňenou extended BPF (eBPF) nebo také classic BPF (cBPF).

\subsection*{extended BPF - eBPF}
Extended BPF povoluje spouštět programy v rámci operačního systému a přidávat další funkce za běhu.
Operační systém pak zaručuje bezpečnost a efektivitu provádění. Díky to tomu je na eBPF je založeno hned několik projektů.
eBPF se používá pro  poskytování vysoce výkonných sítí a vyvažování zátěže v moderních datových centrech a cloudových nativních prostředích,
získávání podrobných dat o pozorovatelnosti zabezpečení při nízké režii, pomáhá vývojářům aplikací sledovat aplikace,
poskytování přehledů pro řešení problémů s výkonem a mnoho dalšího.

\subsection*{classic BPF - cBPF}
Classic BPF je jenom přejmenovaná originální verze BPF.

https://man7.org/linux/man-pages/man2/bpf.2.html
https://ebpf.io/

\subsection{Rozdíl mezi user spacem a kernal spacem}
\subsection*{kernel space}
Kernel space je prostor kde je uložen kód kernelu a kde je vykonáván.

\subsection*{user space}
User space je sada míst, kde běží uživatelské procesy(všechno ostatní kromě jádra). Jádro řídí aplikace v tomto prostoru tak, aby se nepletly mezi sebou.
Procesy, které běží v user spacu mají omezenou část paměti a také nemají přístup ke kernel spacu. Procesy běžící v user spacu mohou přistupovat do kernel spacu
jedině přes rozhranní vystavenný kernel spacem - systémová volání. Pokud proces provede systémové volání, tak se do jádra odešle systémové přerušení a jádro poté obslouží
příslušný proces. Jádro pokračuje ve své práci až po dokončení přerušení.

\subsection{Netem}
Netem je softwarová utilita pomocí které lze vykonávat síťovou emulaci. Síťovou emulaci lze využít k omezení internetových zdrojů pro měřenou aplikaci pro lepší testování.
Netem umí simulovat zpomalení, ztrátu, poškození nebo duplikaci paketů a případně změnu pořadí paketů. Netem je řízen pomocí nástroje 'tc', který se ovládá pomocí příkazové řádky.
Jeho aktuální distribuce je multiplatformě podporována.

\subsection*{Zpoždění paketů}
Tento příkaz přidá zpoždění všech paketů o 100 milisekund. Toto nastavení je ještě omezeno rozlišením jádra.
\begin{lstlisting}[language=bash]
    $ tc qdisc add dev <interface> root netem delay 100ms
\end{lstlisting}

Lze i nastavit zpoždění pomocí rozdělení. Příklad s normalním rozdělením:
\begin{lstlisting}[language=bash]
    $ tc qdisc change dev <interface> root netem delay 100ms 20ms distribution normal
\end{lstlisting}

\subsection*{Ztráta paketů}
Tento příkaz zajistí ztrátu paketů. 1 paket z 1000 je zahozen. nejnižší možná nastavitelná hodnota je 0.0000000232%.
\begin{lstlisting}[language=bash]
    $ tc qdisc change dev <interface> root netem loss 0.1%
\end{lstlisting}

\subsection*{Duplikace paketů}
Příkaz pro duplikaci paketů je podobný jako příkaz pro ztrátu paketů.
\begin{lstlisting}[language=bash]
    $ tc qdisc change dev <interface> root netem duplicate 1%
\end{lstlisting}

\subsection*{Poškození paketů}
Tímto příkazem lze nastavit poškození paketů. Ten zavede bitovou chybu v náhodném offsetu v paketu.
\begin{lstlisting}[language=bash]
    $ tc qdisc change dev <interface> root netem corrupt 0.1%
\end{lstlisting}

\subsection*{Přeuspořádání paketů}
Existují dva způsoby jak nastavit přeuřpořádání paketů.
První způsob je pomocí metody gap, která používá přednastavenou sequekci a změní pořadí každého Ntýho paketu.
\begin{lstlisting}[language=bash]
    $ tc qdisc change dev <interface> root netem gap 5 delay 10ms
\end{lstlisting}
Druhá metoda je reorder způsobí, že určité procento paketů se špatně seřadí.
\begin{lstlisting}[language=bash]
    $ tc qdisc change dev <interface> root netem delay 10ms reorder 25% 50%
\end{lstlisting}

\subsection*{Obnovení rozhranní}
Tento příkaz smaže všechno nastavení na rozhranní. Rozhranní bude poté fungovat jako by na něm žádné nastavení nebylo.
\begin{lstlisting}[language=bash]
    sudo tc qdisc del dev <interface> root
\end{lstlisting}

https://wiki.linuxfoundation.org/networking/netem

\subsection{Virtuální počítač s omezenými zdroji}

Pro zajištění více testovacích platforem je jedna z
Bohužel výpočetní hardware bude stejný pro všechny virtuální platformy i když bude omezený.

\subsection*{Raspberry pi}