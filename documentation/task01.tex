\chapter{Rozbor řešené problematiky}

\section{Zpracování přehled nástrojů Linuxu pro zjištování využití / omezování zdrojů, simulace HW prostředí, odhad rychlosti běhu na dané platformě, doporučování optimalizace a propojit je s požadavky firmy}

\subsection{Berkeley Packet Filter}

Berkeley Packet Filter nebo také BPF je technologie, která umožňuje spouštět programy v jádře operačního systému na linuxových systémech. 
Efektivně a bezpečně rozšíří schopnosti jádra bez nutnosti měnit zdrojový kód jádra. Tato schopnost je často použita ke sledování využití systému či sledování síťového provozu.
BPF je k dispozici na většině unixových operačních systémů a extended BPF je také k dispozici pro Microsoft Windows.

\subsection*{Sledování síťového provozu}
BPF poskytuje rozhranní k datové lince, což umožňuje odesílání a přijímání paketů spojové vrstvy. BPF podporuje filtrování paketů a umožňuje userspace(vysvětlit proč jsem to napsal anglicky) procesům,
aby si samy vyfiltrovaly pakety, které chtějí obdržet a které naopak zahodit. Toto může vést ke zvýšení výkonu operačního systému, protože se tím zamezí kopírování nepotřebných paketů.

\subsection*{Verze BPF}
BPF má více verzí. Výše zmíňenou extended BPF (eBPF) nebo také classic BPF (cBPF).

https://ebpf.io/

Vysvětlit co je userspace

\subsection{Netem}
Netem je softwarová utilita pomocí které lze vykonávat síťovou emulaci. Síťovou emulaci lze využít k omezení internetových zdrojů pro měřenou aplikaci pro lepší testování.
Netem umí simulovat zpomalení, ztrátu, poškození nebo duplikaci paketů a případně změnu pořadí paketů. Netem je řízen pomocí nástroje 'tc', který se ovládá pomocí příkazové řádky.
Jeho aktuální distribuce je multiplatformě podporována.

\subsection*{Zpoždění paketů}
Tento příkaz přidá zpoždění všech paketů o 100 milisekund. Toto nastavení je ještě omezeno rozlišením jádra.
\begin{lstlisting}[language=bash]
    $ tc qdisc add dev eth0 root netem delay 100ms
\end{lstlisting}

Lze i nastavit zpoždění pomocí rozdělení. Příklad s normalním rozdělením:
\begin{lstlisting}[language=bash]
    $ tc qdisc change dev eth0 root netem delay 100ms 20ms distribution normal
\end{lstlisting}

\subsection*{Ztráta paketů}
Tento příkaz zajistí ztrátu paketů. 1 paket z 1000 je zahozen. nejnižší možná nastavitelná hodnota je 0.0000000232%.  
\begin{lstlisting}[language=bash]
    $ tc qdisc change dev eth0 root netem loss 0.1%
\end{lstlisting}

\subsection*{Duplikace paketů}
Příkaz pro duplikaci paketů je podobný jako příkaz pro ztrátu paketů.
\begin{lstlisting}[language=bash]
    $ tc qdisc change dev eth0 root netem duplicate 1%
\end{lstlisting}

\subsection*{Poškození paketů}
Tímto příkazem lze nastavit poškození paketů. Ten zavede bitovou chybu v náhodném offsetu v paketu. 
\begin{lstlisting}[language=bash]
    $ tc qdisc change dev eth0 root netem corrupt 0.1%
\end{lstlisting}

\subsection*{Přeuspořádání paketů}
Jsou dva způsoby jak nastavit přeuřpořádání paketů.
První způsob je pomocí metody gap, která používá přednastavenou sequekci a změní pořadí každého Ntýho paketu.
\begin{lstlisting}[language=bash]
    $ tc qdisc change dev eth0 root netem gap 5 delay 10ms
\end{lstlisting}
Druhá metoda je reorder způsobí, že určité procento paketů se špatně seřadí.
\begin{lstlisting}[language=bash]
    $ tc qdisc change dev eth0 root netem delay 10ms reorder 25% 50%
\end{lstlisting}

\subsection*{Obnovení rozhranní}

https://wiki.linuxfoundation.org/networking/netem

\subsection{Virtuální počítač s omezenými zdroji}

Pro zajištění více testovacích platforem je jedna z
Bohužel výpočetní hardware bude stejný pro všechny virtuální platformy i když bude omezený.

\subsection{Raspberry pi}