\chapter{Rozbor řešené problematiky}

\section{Zpracování přehled nástrojů Linuxu pro zjištování využití / omezování zdrojů, simulace HW prostředí, odhad rychlosti běhu na dané platformě, doporučování optimalizace a propojit je s požadavky firmy}

\subsection{Berkeley Packet Filter}

Berkeley Packet Filter nebo také BPF je technologie v unixových systémech pro zajištění sledování využití systému čí sledování síťového provozu.
BPF poskytuje rozhranní k datové lince, což umožňuje odesílání a přijímání paketů spojové vrstvy. BPF je k dispozici na většině unixových operačních systémů
a extended BPF je také k dispozici pro Microsoft Windows. BPF podporuje filtrování paketů a umožňuje userspace(vysvětlit proč jsem to napsal anglicky) procesům,
aby si samy vyfiltrovaly pakety, které chtějí obdržet a které naopak zahodit. Toto může vést ke zvýšení výkonu operačního systému, protože se tím zamezí kopírování nepotřebných paketů.

\subsection*{Verze BPF}
BPF má více verzí. Výše zmíňenou extended BPF (eBPF) nebo také classic BPF (cBPF).

Vysvětlit co je userspace

\subsection{Netem}
Netem je softwarová utilita pomocí které lze vykonávat síťovou emulaci. Síťovou emulaci lze využít k omezení internetových zdrojů pro měřenou aplikaci pro lepší testování.
Netem umí simulovat zpomalení, ztrátu, poškození nebo duplikaci paketů a případně změnu pořadí paketů. Netem je řízen pomocí nástroje 'tc', který se ovládá pomocí příkazové řádky.
Jeho aktuální distribuce je multiplatformě podporována.

\subsection*{Zpoždění paketů}


\subsection*{Ztráta paketů}

\subsection*{Duplikace paketů}

\subsection*{Poškození paketů}

\subsection*{Změna pořadí paketů}

https://wiki.linuxfoundation.org/networking/netem

\subsection{Virtuální počítač s omezenými zdroji}

Pro zajištění testování na více platformách, kter

\subsection{Raspberry pi}