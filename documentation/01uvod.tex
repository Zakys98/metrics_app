\chapter{Úvod}

V dnešní době žijeme ve světě, který každý den využívá informační technologie. Toto je důvodem, že nároky na tyto technologie se
každý rok zvyšují. Platí to jak pro hardware tak i pro software. U hardwaru obecně platí, že na konci roku lze pořídít 2x výkonější,
úspornější a lepší hardware než na začátku roku. Proto je potřeba všechno monitorovat a sledovat, jestli je stávající hardware stále
dostatečně výkonný pro běh všech potřebných aplikací.

Cílem této bakalářské práce je navrhnout a implementovat systém, který sleduje a zjištuje, zda je linuxový operační systém schopen spustit
všechny potřebné aplikace a zda je jeho hardware dostačující pro jejich běh. Pomocí této funkce lze pak snadno vyměňovat výpočetní jednotky
nebo aktualizovat operační systém bez obav o výkon a stabilitu prostředí. Toto je spraveno pomocí systémových volání, které jsou popsány dále
v textu. Hlavní motivací pro vypracování této bakalářské práce je její reálné využití v praxi.

\iffalse
Na trhu neni nic podobneho nejak to napsat
\fi

Tato bakalářská práce se liší hlavně v tom, že jejích cílem není měřit výkon a zátěž jako takového, ale zjistit, jestli jsou daný operační systém spolu s hardwarem dostačující pro běh všech potřebných aplikací. Na toto je potřeba myslet celou dobu při čtení této práce.

V kapitole \hyperref[sec:RozborReseneProblematiky]{2} je diskutováno měření výkonu počítačů, způsoby měření, následné testování systémů, nástroje využité k vypracování této bakalářské práce a způsoby jakým se používají. Kapitola \hyperref[sec:FungovaniOperacnihoSystemuLinux]{3} se týká fungování operačního systému linux. Jsou zde popsány všechny důležité informace, které byly využity při návrhu a samotné implementaci této práce. Návrhu a postupu při vypracování tohoto systému je věnována kapitola \hyperref[sec:NavrhSystemu]{4}. Tato kapitola také obsahuje základní požadavky. Informace ohledně implementace s ukázkou zdrojového kódu jsou popsány v kapitole \hyperref[sec:ImplementaceSystemu]{5}. Poslední část je, o testování jednotlivých části i celého systému, obsažena v kapitole \hyperref[sec:Testovani]{6}.
