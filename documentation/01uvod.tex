\chapter{Úvod}

V dnešní době žijeme ve světě, který každý den využívá informační technologie. Toto je důvodem, že nároky na tyto technologie se
každý rok zvyšují. Platí to jak pro hardware tak i pro software. U hardwaru obecně platí, že na konci roku lze pořídit 2x výkonnější,
úspornější a lepší hardware než na začátku roku. Proto je potřeba monitorovat a sledovat, jestli je stávající hardware stále
dostatečně výkonný pro běh všech potřebných aplikací.

Cílem této bakalářské práce je navrhnout a implementovat systém, který sleduje a zjišťuje, zda je výpočetní platforma schopna spustit a nechat stabilně běžet
všechny potřebné aplikace. Pomocí této funkce lze pak snadno vyměňovat výpočetní jednotky
nebo aktualizovat operační systém bez obav o výkon a stabilitu prostředí. 
Toto je docíleno pomocí technologie Berkeley Packet Filter, která je dále popsána v textu práce. Hlavní motivací pro vypracování této bakalářské práce je její reálné využití v praxi. 

\iffalse
Na trhu neni nic podobneho nejak to napsat
\fi

Tato bakalářská práce se liší hlavně v tom, že jejích cílem není měřit výkon a zátěž jako takového, ale zjistit, jestli je daná výpočetní platforma dostačující pro běh všech potřebných aplikací. Na toto je potřeba myslet celou dobu při čtení této práce.

V kapitole \hyperref[sec:RozborReseneProblematiky]{2} je diskutováno měření výkonu počítačů, způsoby měření, následné testování systémů, nástroje využité k vypracování této bakalářské práce a způsoby jakým se používají. Kapitola \hyperref[sec:FungovaniOperacnihoSystemuLinux]{3} se týká fungování operačního systému linux. Jsou zde popsány všechny důležité informace, které byly využity při návrhu a samotné implementaci této práce. Návrhu a postupu při vypracování tohoto systému je věnována kapitola \hyperref[sec:NavrhSystemu]{4}. Tato kapitola také obsahuje základní požadavky. Informace ohledně implementace s ukázkou zdrojového kódu jsou popsány v kapitole \hyperref[sec:ImplementaceSystemu]{5}. V předposlední kapitole \hyperref[sec:Testovani]{6} této bakalářské práce je popsáno testování jednotlivých částí i celého systému.


\iffalse
uvod:
proc tam ktera cast je, motivace, proc to delam, limit soucasneho sveta a o co se snazim ja
uvod napsat tam nejaky priklad ... autonomni rizeni
potreba optimalizovat vykon primo na miste
zamyslet se nad strukturou uvodu
\fi
