\chapter{Implementace systému
\label{sec:ImplementaceSystemu}}

\iffalse

dodelat do programu:
zachytavat comma a tim zjistit ktere procesy to vyvolavaji

sizer = k cemu je, struktura formatu jeho vystupu
generator
track
    - rozdeleni na kernel a user cast
    data structure alligment
    debugovani pomoci sys/kernel/debug/tracing/trace_pipe
evaulator
cmake
    = vypsat vsechny dulezite veci co se tam pouzivaji jako vmlinux a prepinace
velky vystupni soubor
pripadne problemy

\fi

Tato kapitola popisuje implementaci jednotlivých programů, které jsou součástí tohoto systému. K popisu implementační kódové části a ukázky zajímavých částí kódu, se také kapitola věnuje přípravě jednotlivých částí systému, jejich instalaci a použité technologii. Návod na vytvoření celého systému je přiložený v \hyperref[sec:Prikazy]{příloze}.

\section{Nástroj na zjištění velikostí datových typů}

Nástroj na zjištění velikostí datových typů je velice jednoduchý a krátký program. Jak již bylo zmíněno v \hyperref[sec:NavrhSystemu]{návrhu}, tak je aplikace použita pro dynamické zjištění datových typů na daném linuxovém operačním systému. Nástroj je naprogramován v programovacím jazyce C. Obsahuje pouze pár řádků, které do souboru vypíší velikosti některých datových typů. Formát každého řádku je \emph{datový typ:velikost}. Spouští se pomocí příkazu \emph{./sizer}. Výstup se zapisuje do souboru s názvem \emph{sizes}. 

\section{Generátor}

\section{Sledovací aplikace}

\subsection{Tvorba sledovací aplikace}

\iffalse
pragma pack == data structure alligment
\fi

\section{Evaulační aplikace}

\subsection{Tvorba evaulační aplikace}
