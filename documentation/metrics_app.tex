% kodovani: UTF-8 (zmena prikazem iconv, recode nebo cstocs)
%------------------------------------------------------------------------------
% zpracování: make, make pdf, make clean
%==============================================================================
% Soubory, které je nutné upravit nebo smazat:
%   projekt-20-literatura-bibliography.bib - literatura
%   projekt-01-kapitoly-chapters.tex - obsah práce
%   projekt-30-prilohy-appendices.tex - přílohy
%==============================================================================
\documentclass[]{fitthesis} % bez zadání - pro začátek práce, aby nebyl problém s překladem
%\documentclass[english]{fitthesis} % without assignment - for the work start to avoid compilation problem
%\documentclass[zadani]{fitthesis} % odevzdani do wisu a/nebo tisk s barevnými odkazy - odkazy jsou barevné
%\documentclass[english,zadani]{fitthesis} % for submission to the IS FIT and/or print with color links - links are color
%\documentclass[zadani,print]{fitthesis} % pro černobílý tisk - odkazy jsou černé
%\documentclass[english,zadani,print]{fitthesis} % for the black and white print - links are black
%\documentclass[zadani,cprint]{fitthesis} % pro barevný tisk - odkazy jsou černé, znak VUT barevný
%\documentclass[english,zadani,cprint]{fitthesis} % for the print - links are black, logo is color
% * Je-li práce psaná ve slovenském jazyce, je zapotřebí u třídy použít
%   parametr slovak následovně:
%      \documentclass[slovak]{fitthesis}
% * Je-li práce psaná v anglickém jazyce se slovenským abstraktem apod.,
%   je zapotřebí u třídy použít parametry english a enslovak následovně:
%      \documentclass[english,enslovak]{fitthesis}

% zde můžeme vložit vlastní balíčky


% Nastavení cesty k obrázkům
%\graphicspath{{obrazky-figures/}{./obrazky-figures/}}
%\graphicspath{{obrazky-figures/}{../obrazky-figures/}}

%---rm---------------
\renewcommand{\rmdefault}{lmr}%zavede Latin Modern Roman jako rm
%---sf---------------
\renewcommand{\sfdefault}{qhv}%zavede TeX Gyre Heros jako sf
%---tt------------
\renewcommand{\ttdefault}{lmtt}% zavede Latin Modern tt jako tt

% vypne funkci šablony, která automaticky nahrazuje uvozovky,
% aby nebyly prováděny nevhodné náhrady v popisech API apod.
\csdoublequotesoff


\usepackage{url}


% =======================================================================
% balíček "hyperref" vytváří klikací odkazy v pdf, pokud tedy použijeme pdflatex
% problém je, že balíček hyperref musí být uveden jako poslední, takže nemůže
% být v šabloně
\ifWis
\ifx\pdfoutput\undefined % nejedeme pod pdflatexem
\else
  \usepackage{color}
  \usepackage[unicode,colorlinks,hyperindex,plainpages=false,pdftex]{hyperref}
  \definecolor{hrcolor-ref}{RGB}{223,52,30}
  \definecolor{hrcolor-cite}{HTML}{2F8F00}
  \definecolor{hrcolor-urls}{HTML}{092EAB}
  \hypersetup{
	linkcolor=hrcolor-ref,
	citecolor=hrcolor-cite,
	filecolor=magenta,
	urlcolor=hrcolor-urls
  }
  \def\pdfBorderAttrs{/Border [0 0 0] }  % bez okrajů kolem odkazů
  \pdfcompresslevel=9
\fi
\else % pro tisk budou odkazy, na které se dá klikat, černé
\ifx\pdfoutput\undefined % nejedeme pod pdflatexem
\else
  \usepackage{color}
  \usepackage[unicode,colorlinks,hyperindex,plainpages=false,pdftex,urlcolor=black,linkcolor=black,citecolor=black]{hyperref}
  \definecolor{links}{rgb}{0,0,0}
  \definecolor{anchors}{rgb}{0,0,0}
  \def\AnchorColor{anchors}
  \def\LinkColor{links}
  \def\pdfBorderAttrs{/Border [0 0 0] } % bez okrajů kolem odkazů
  \pdfcompresslevel=9
\fi
\fi
% Řešení problému, kdy klikací odkazy na obrázky vedou za obrázek
\usepackage[all]{hypcap}

% Informace o práci/projektu
%---------------------------------------------------------------------------
\projectinfo{
  %Prace
  project={BP},            %typ práce BP/SP/DP/DR
  year={2022},             % rok odevzdání
  date=\today,             % datum odevzdání
  %Nazev prace
  title.cs={Systém pro ověření minimálních potřebných zdrojů pro běh aplikace},  % název práce v češtině či slovenštině (dle zadání)
  title.en={System for Verifying the Minimum Resources Required to Run an Application}, % název práce v angličtině
  %title.length={14.5cm}, % nastavení délky bloku s titulkem pro úpravu zalomení řádku (lze definovat zde nebo níže)
  %sectitle.length={14.5cm}, % nastavení délky bloku s druhým titulkem pro úpravu zalomení řádku (lze definovat zde nebo níže)
  %dectitle.length={14.5cm}, % nastavení délky bloku s titulkem nad prohlášením pro úpravu zalomení řádku (lze definovat zde nebo níže)
  %Autor / Author
  author.name={Jiří},   % jméno autora
  author.surname={Žák},   % příjmení autora
  %Ustav
  department={UPGM},
  % Školitel
  supervisor.name={Pavel},   % jméno školitele
  supervisor.surname={Smrž},   % příjmení školitele
  supervisor.title.p={doc. RNDr.},   %titul před jménem (nepovinné)
  supervisor.title.a={Ph.D.},    %titul za jménem (nepovinné)
  % Klíčová slova
  keywords.cs={Sem budou zapsána jednotlivá klíčová slova v českém (slovenském) jazyce, oddělená čárkami.},
  keywords.en={Sem budou zapsána jednotlivá klíčová slova v anglickém jazyce, oddělená čárkami.},
  % Abstrakt
  abstract.cs={Do tohoto odstavce bude zapsán výtah (abstrakt) práce v českém (slovenském) jazyce.},
  abstract.en={Do tohoto odstavce bude zapsán výtah (abstrakt) práce v anglickém jazyce.},
  % Prohlášení (u anglicky psané práce anglicky, u slovensky psané práce slovensky)
  declaration={Prohlašuji, že jsem tuto bakalářskou práci vypracoval samostatně pod vedením pana X...
Další informace mi poskytli...
Uvedl jsem všechny literární prameny, publikace a další zdroje, ze kterých jsem čerpal.},
  % Poděkování (nepovinné, nejlépe v jazyce práce)
  acknowledgment={V této sekci je možno uvést poděkování vedoucímu práce a těm, kteří poskytli odbornou pomoc
(externí zadavatel, konzultant apod.).},
  % Rozšířený abstrakt (cca 3 normostrany) - lze definovat zde nebo níže
  %extendedabstract={Do tohoto odstavce bude zapsán rozšířený výtah (abstrakt) práce v českém (slovenském) jazyce.},
  %extabstract.odd={true}, % Začít rozšířený abstrakt na liché stránce?
  %faculty={FIT}, % FIT/FEKT/FSI/FA/FCH/FP/FAST/FAVU/USI/DEF
  faculty.cs={Fakulta informačních technologií}, % Fakulta v češtině - pro využití této položky výše zvolte fakultu DEF
  faculty.en={Faculty of Information Technology}, % Fakulta v angličtině - pro využití této položky výše zvolte fakultu DEF
  department.cs={Ústav matematiky}, % Ústav v češtině - pro využití této položky výše zvolte ústav DEF nebo jej zakomentujte
  department.en={Institute of Mathematics} % Ústav v angličtině - pro využití této položky výše zvolte ústav DEF nebo jej zakomentujte
}

% Rozšířený abstrakt (cca 3 normostrany) - lze definovat zde nebo výše
%\extendedabstract{Do tohoto odstavce bude zapsán výtah (abstrakt) práce v českém (slovenském) jazyce.}
% Začít rozšířený abstrakt na liché stránce?
%\extabstractodd{true}

% nastavení délky bloku s titulkem pro úpravu zalomení řádku - lze definovat zde nebo výše
%\titlelength{14.5cm}
% nastavení délky bloku s druhým titulkem pro úpravu zalomení řádku - lze definovat zde nebo výše
%\sectitlelength{14.5cm}
% nastavení délky bloku s titulkem nad prohlášením pro úpravu zalomení řádku - lze definovat zde nebo výše
%\dectitlelength{14.5cm}

% řeší první/poslední řádek odstavce na předchozí/následující stránce
\clubpenalty=10000
\widowpenalty=10000

% checklist
\newlist{checklist}{itemize}{1}
\setlist[checklist]{label=$\square$}

% Nechcete-li, aby se u oboustranného tisku roztahovaly mezery pro zaplnění stránky, odkomentujte následující řádek
% \raggedbottom

\begin{document}
  % Vysazeni titulnich stran
  % ----------------------------------------------
  \maketitle
  % Obsah
  % ----------------------------------------------
  \setlength{\parskip}{0pt}

  {\hypersetup{hidelinks}\tableofcontents}

  % Seznam obrazku a tabulek (pokud prace obsahuje velke mnozstvi obrazku, tak se to hodi)
  \renewcommand\listfigurename{Seznam obrázků}
  % {\hypersetup{hidelinks}\listoffigures}

  \renewcommand\listtablename{Seznam tabulek}
  % {\hypersetup{hidelinks}\listoftables}

  \ifODSAZ
    \setlength{\parskip}{0.5\bigskipamount}
  \else
    \setlength{\parskip}{0pt}
  \fi

  % vynechani stranky v oboustrannem rezimu
  \iftwoside
    \cleardoublepage
  \fi

  % Text prace / Thesis text
  % ----------------------------------------------
  \chapter{Úvod}

V dnešní době žijeme ve světě, který každý den využívá informační technologie. Toto je důvodem, že nároky na tyto technologie se
každý rok zvyšují. Platí to jak pro hardware tak i pro software. U hardwaru obecně platí, že na konci roku lze pořídit 2x výkonnější,
úspornější a lepší hardware než na začátku roku. Proto je potřeba monitorovat a sledovat, jestli je stávající hardware stále
dostatečně výkonný pro běh všech potřebných aplikací.

Cílem této bakalářské práce je navrhnout a implementovat systém, který sleduje a zjišťuje, zda je výpočetní platforma schopna spustit a nechat stabilně běžet
všechny potřebné aplikace. Pomocí této funkce lze pak snadno vyměňovat výpočetní jednotky
nebo aktualizovat operační systém bez obav o výkon a stabilitu prostředí. 
Toto je docíleno pomocí technologie Berkeley Packet Filter, která je dále popsána v textu práce. Hlavní motivací pro vypracování této bakalářské práce je její reálné využití v praxi. 

\iffalse
Na trhu neni nic podobneho nejak to napsat
\fi

Tato bakalářská práce se liší hlavně v tom, že jejích cílem není měřit výkon a zátěž jako takového, ale zjistit, jestli je daná výpočetní platforma dostačující pro běh všech potřebných aplikací. Na toto je potřeba myslet celou dobu při čtení této práce.

V kapitole \hyperref[sec:RozborReseneProblematiky]{2} je diskutováno měření výkonu počítačů, způsoby měření, následné testování systémů, nástroje využité k vypracování této bakalářské práce a způsoby jakým se používají. Kapitola \hyperref[sec:FungovaniOperacnihoSystemuLinux]{3} se týká fungování operačního systému linux. Jsou zde popsány všechny důležité informace, které byly využity při návrhu a samotné implementaci této práce. Návrhu a postupu při vypracování tohoto systému je věnována kapitola \hyperref[sec:NavrhSystemu]{4}. Tato kapitola také obsahuje základní požadavky. Informace ohledně implementace s ukázkou zdrojového kódu jsou popsány v kapitole \hyperref[sec:ImplementaceSystemu]{5}. V předposlední kapitole \hyperref[sec:Testovani]{6} této bakalářské práce je popsáno testování jednotlivých částí i celého systému.


\iffalse
uvod:
proc tam ktera cast je, motivace, proc to delam, limit soucasneho sveta a o co se snazim ja
uvod napsat tam nejaky priklad ... autonomni rizeni
potreba optimalizovat vykon primo na miste
zamyslet se nad strukturou uvodu
\fi

  \section{Zparacování přehled nástrojů Linuxu pro zjištování využití / omezování zdrojů, simulace HW prostředí, odhad rychlosti běhu na dané platformě, doporučování optimalizace a propojit je s požadavky firmy}

\subsection{asdasd}

sdfsfsd

  %\input{projekt-01-kapitoly-chapters}

  % Kompilace po částech (viz výše, nutno odkomentovat)
  %\subfile{projekt-01-uvod-introduction}

  % Pouzita literatura / Bibliography
  % ----------------------------------------------
\ifslovak
  \makeatletter
  \def\@openbib@code{\addcontentsline{toc}{chapter}{Literatúra}}
  \makeatother
  \bibliographystyle{bib-styles/Pysny/skplain}
\else
  \ifczech
    \makeatletter
    \def\@openbib@code{\addcontentsline{toc}{chapter}{Literatura}}
    \makeatother
    \bibliographystyle{bib-styles/Pysny/czplain}
  \else
    \makeatletter
    \def\@openbib@code{\addcontentsline{toc}{chapter}{Bibliography}}
    \makeatother
    \bibliographystyle{bib-styles/Pysny/enplain}
  %  \bibliographystyle{alpha}
  \fi
\fi
  \begin{flushleft}
  \bibliography{bibliography}
  \end{flushleft}

  % vynechani stranky v oboustrannem rezimu
  \iftwoside
    \cleardoublepage
  \fi

  % Prilohy / Appendices
  % ---------------------------------------------
  \appendix
%\ifczech
%  \renewcommand{\appendixpagename}{Přílohy}
%  \renewcommand{\appendixtocname}{Přílohy}
%  \renewcommand{\appendixname}{Příloha}
%\fi
%\ifslovak
%  \renewcommand{\appendixpagename}{Prílohy}
%  \renewcommand{\appendixtocname}{Prílohy}
%  \renewcommand{\appendixname}{Príloha}
%\fi
%  \appendixpage

% vynechani stranky v oboustrannem rezimu
%\iftwoside
%  \cleardoublepage
%\fi

\ifslovak
%  \section*{Zoznam príloh}
%  \addcontentsline{toc}{section}{Zoznam príloh}
\else
  \ifczech
%    \section*{Seznam příloh}
%    \addcontentsline{toc}{section}{Seznam příloh}
  \else
%    \section*{List of Appendices}
%    \addcontentsline{toc}{section}{List of Appendices}
  \fi
\fi
  \startcontents[chapters]
  \setlength{\parskip}{0pt}
  % seznam příloh / list of appendices
  % \printcontents[chapters]{l}{0}{\setcounter{tocdepth}{2}}

  \ifODSAZ
    \setlength{\parskip}{0.5\bigskipamount}
  \else
    \setlength{\parskip}{0pt}
  \fi

  % vynechani stranky v oboustrannem rezimu
  \iftwoside
    \cleardoublepage
  \fi

\end{document}
