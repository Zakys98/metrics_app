%==============================================================================
% tento soubor pouzijte jako zaklad
% this file should be used as a base for the thesis
% Autoři / Authors: 2008 Michal Bidlo, 2019 Jaroslav Dytrych
% Kontakt pro dotazy a připomínky: sablona@fit.vutbr.cz
%==============================================================================
% kodovani: UTF-8 (zmena prikazem iconv, recode nebo cstocs)
%------------------------------------------------------------------------------
% zpracování / processing: make, make pdf, make clean
%==============================================================================
% Soubory, které je nutné upravit nebo smazat: / Files which have to be edited or deleted:
%   projekt-20-literatura-bibliography.bib - literatura / bibliography
%   projekt-01-kapitoly-chapters.tex - obsah práce / the thesis content
%   projekt-01-kapitoly-chapters-en.tex - obsah práce v angličtině / the thesis content in English
%   projekt-30-prilohy-appendices.tex - přílohy / appendices
%   projekt-30-prilohy-appendices-en.tex - přílohy v angličtině / appendices in English
%==============================================================================
\documentclass[]{fitthesis} % bez zadání - pro začátek práce, aby nebyl problém s překladem
%\documentclass[english]{fitthesis} % without assignment - for the work start to avoid compilation problem
%\documentclass[zadani]{fitthesis} % odevzdani do wisu a/nebo tisk s barevnými odkazy - odkazy jsou barevné
%\documentclass[english,zadani]{fitthesis} % for submission to the IS FIT and/or print with color links - links are color
%\documentclass[zadani,print]{fitthesis} % pro černobílý tisk - odkazy jsou černé
%\documentclass[english,zadani,print]{fitthesis} % for the black and white print - links are black
%\documentclass[zadani,cprint]{fitthesis} % pro barevný tisk - odkazy jsou černé, znak VUT barevný
%\documentclass[english,zadani,cprint]{fitthesis} % for the print - links are black, logo is color
% * Je-li práce psaná v anglickém jazyce, je zapotřebí u třídy použít 
%   parametr english následovně:
%      \documentclass[english]{fitthesis}
% * Je-li práce psaná ve slovenském jazyce, je zapotřebí u třídy použít 
%   parametr slovak následovně:
%      \documentclass[slovak]{fitthesis}
% * Je-li práce psaná v anglickém jazyce se slovenským abstraktem apod., 
%   je zapotřebí u třídy použít parametry english a enslovak následovně:
%      \documentclass[english,enslovak]{fitthesis}

% Základní balíčky jsou dole v souboru šablony fitthesis.cls
% Basic packages are at the bottom of template file fitthesis.cls
% zde můžeme vložit vlastní balíčky / you can place own packages here

% Kompilace po částech (rychlejší, ale v náhledu nemusí být vše aktuální)
% \usepackage{subfiles}

% Nastavení cesty k obrázkům
% Setting of a path to the pictures
%\graphicspath{{obrazky-figures/}{./obrazky-figures/}}
%\graphicspath{{obrazky-figures/}{../obrazky-figures/}}

%---rm---------------
\renewcommand{\rmdefault}{lmr}%zavede Latin Modern Roman jako rm / set Latin Modern Roman as rm
%---sf---------------
\renewcommand{\sfdefault}{qhv}%zavede TeX Gyre Heros jako sf
%---tt------------
\renewcommand{\ttdefault}{lmtt}% zavede Latin Modern tt jako tt

% vypne funkci šablony, která automaticky nahrazuje uvozovky,
% aby nebyly prováděny nevhodné náhrady v popisech API apod.
% disables function of the template which replaces quotation marks
% to avoid unnecessary replacements in the API descriptions etc.
\csdoublequotesoff


\usepackage{url}


% =======================================================================
% balíček "hyperref" vytváří klikací odkazy v pdf, pokud tedy použijeme pdflatex
% problém je, že balíček hyperref musí být uveden jako poslední, takže nemůže
% být v šabloně
% "hyperref" package create clickable links in pdf if you are using pdflatex.
% Problem is that this package have to be introduced as the last one so it 
% can not be placed in the template file.
\ifWis
\ifx\pdfoutput\undefined % nejedeme pod pdflatexem / we are not using pdflatex
\else
  \usepackage{color}
  \usepackage[unicode,colorlinks,hyperindex,plainpages=false,pdftex]{hyperref}
  \definecolor{hrcolor-ref}{RGB}{223,52,30}
  \definecolor{hrcolor-cite}{HTML}{2F8F00}
  \definecolor{hrcolor-urls}{HTML}{092EAB}
  \hypersetup{
	linkcolor=hrcolor-ref,
	citecolor=hrcolor-cite,
	filecolor=magenta,
	urlcolor=hrcolor-urls
  }
  \def\pdfBorderAttrs{/Border [0 0 0] }  % bez okrajů kolem odkazů / without margins around links
  \pdfcompresslevel=9
\fi
\else % pro tisk budou odkazy, na které se dá klikat, černé / for the print clickable links will be black
\ifx\pdfoutput\undefined % nejedeme pod pdflatexem / we are not using pdflatex
\else
  \usepackage{color}
  \usepackage[unicode,colorlinks,hyperindex,plainpages=false,pdftex,urlcolor=black,linkcolor=black,citecolor=black]{hyperref}
  \definecolor{links}{rgb}{0,0,0}
  \definecolor{anchors}{rgb}{0,0,0}
  \def\AnchorColor{anchors}
  \def\LinkColor{links}
  \def\pdfBorderAttrs{/Border [0 0 0] } % bez okrajů kolem odkazů / without margins around links
  \pdfcompresslevel=9
\fi
\fi
% Řešení problému, kdy klikací odkazy na obrázky vedou za obrázek
% This solves the problems with links which leads after the picture
\usepackage[all]{hypcap}

% Informace o práci/projektu / Information about the thesis
%---------------------------------------------------------------------------
\projectinfo{
  %Prace / Thesis
  project={BP},            %typ práce BP/SP/DP/DR  / thesis type (SP = term project)
  year={2021},             % rok odevzdání / year of submission
  date=\today,             % datum odevzdání / submission date
  %Nazev prace / thesis title
  title.cs={Název práce},  % název práce v češtině či slovenštině (dle zadání) / thesis title in czech language (according to assignment)
  title.en={Thesis title}, % název práce v angličtině / thesis title in english
  %title.length={14.5cm}, % nastavení délky bloku s titulkem pro úpravu zalomení řádku (lze definovat zde nebo níže) / setting the length of a block with a thesis title for adjusting a line break (can be defined here or below)
  %sectitle.length={14.5cm}, % nastavení délky bloku s druhým titulkem pro úpravu zalomení řádku (lze definovat zde nebo níže) / setting the length of a block with a second thesis title for adjusting a line break (can be defined here or below)
  %dectitle.length={14.5cm}, % nastavení délky bloku s titulkem nad prohlášením pro úpravu zalomení řádku (lze definovat zde nebo níže) / setting the length of a block with a thesis title above declaration for adjusting a line break (can be defined here or below)
  %Autor / Author
  author.name={Jiří},   % jméno autora / author name
  author.surname={Žák},   % příjmení autora / author surname 
  %author.title.p={Bc.}, % titul před jménem (nepovinné) / title before the name (optional)
  %author.title.a={Ph.D.}, % titul za jménem (nepovinné) / title after the name (optional)
  %Ustav / Department
  department={UPGM}, % doplňte příslušnou zkratku dle ústavu na zadání: UPSY/UIFS/UITS/UPGM / fill in appropriate abbreviation of the department according to assignment: UPSY/UIFS/UITS/UPGM
  % Školitel / supervisor
  supervisor.name={Jméno},   % jméno školitele / supervisor name 
  supervisor.surname={Příjmení},   % příjmení školitele / supervisor surname
  supervisor.title.p={prof. RNDr.},   %titul před jménem (nepovinné) / title before the name (optional)
  supervisor.title.a={Ph.D.},    %titul za jménem (nepovinné) / title after the name (optional)
  % Klíčová slova / keywords
  keywords.cs={Sem budou zapsána jednotlivá klíčová slova v českém (slovenském) jazyce, oddělená čárkami.}, % klíčová slova v českém či slovenském jazyce / keywords in czech or slovak language
  keywords.en={Sem budou zapsána jednotlivá klíčová slova v anglickém jazyce, oddělená čárkami.}, % klíčová slova v anglickém jazyce / keywords in english
  %keywords.en={Here, individual keywords separated by commas will be written in English.},
  % Abstrakt / Abstract
  abstract.cs={Do tohoto odstavce bude zapsán výtah (abstrakt) práce v českém (slovenském) jazyce.}, % abstrakt v českém či slovenském jazyce / abstract in czech or slovak language
  abstract.en={Do tohoto odstavce bude zapsán výtah (abstrakt) práce v anglickém jazyce.}, % abstrakt v anglickém jazyce / abstract in english
  %abstract.en={An abstract of the work in English will be written in this paragraph.},
  % Prohlášení (u anglicky psané práce anglicky, u slovensky psané práce slovensky) / Declaration (for thesis in english should be in english)
  declaration={Prohlašuji, že jsem tuto bakalářskou práci vypracoval samostatně pod vedením pana X...
Další informace mi poskytli...
Uvedl jsem všechny literární prameny, publikace a další zdroje, ze kterých jsem čerpal.},
  %declaration={I hereby declare that this Bachelor's thesis was prepared as an original work by the author under the supervision of Mr. X
% The supplementary information was provided by Mr. Y
% I have listed all the literary sources, publications and other sources, which were used during the preparation of this thesis.},
  % Poděkování (nepovinné, nejlépe v jazyce práce) / Acknowledgement (optional, ideally in the language of the thesis)
  acknowledgment={V této sekci je možno uvést poděkování vedoucímu práce a těm, kteří poskytli odbornou pomoc
(externí zadavatel, konzultant apod.).},
  %acknowledgment={Here it is possible to express thanks to the supervisor and to the people which provided professional help
%(external submitter, consultant, etc.).},
  % Rozšířený abstrakt (cca 3 normostrany) - lze definovat zde nebo níže / Extended abstract (approximately 3 standard pages) - can be defined here or below
  %extendedabstract={Do tohoto odstavce bude zapsán rozšířený výtah (abstrakt) práce v českém (slovenském) jazyce.},
  %extabstract.odd={true}, % Začít rozšířený abstrakt na liché stránce? / Should extended abstract start on the odd page?
  %faculty={FIT}, % FIT/FEKT/FSI/FA/FCH/FP/FAST/FAVU/USI/DEF
  faculty.cs={Fakulta informačních technologií}, % Fakulta v češtině - pro využití této položky výše zvolte fakultu DEF / Faculty in Czech - for use of this entry select DEF above
  faculty.en={Faculty of Information Technology}, % Fakulta v angličtině - pro využití této položky výše zvolte fakultu DEF / Faculty in English - for use of this entry select DEF above
  department.cs={Ústav matematiky}, % Ústav v češtině - pro využití této položky výše zvolte ústav DEF nebo jej zakomentujte / Department in Czech - for use of this entry select DEF above or comment it out
  department.en={Institute of Mathematics} % Ústav v angličtině - pro využití této položky výše zvolte ústav DEF nebo jej zakomentujte / Department in English - for use of this entry select DEF above or comment it out
}

% Rozšířený abstrakt (cca 3 normostrany) - lze definovat zde nebo výše / Extended abstract (approximately 3 standard pages) - can be defined here or above
%\extendedabstract{Do tohoto odstavce bude zapsán výtah (abstrakt) práce v českém (slovenském) jazyce.}
% Začít rozšířený abstrakt na liché stránce? / Should extended abstract start on the odd page?
%\extabstractodd{true}

% nastavení délky bloku s titulkem pro úpravu zalomení řádku - lze definovat zde nebo výše / setting the length of a block with a thesis title for adjusting a line break - can be defined here or above
%\titlelength{14.5cm}
% nastavení délky bloku s druhým titulkem pro úpravu zalomení řádku - lze definovat zde nebo výše / setting the length of a block with a second thesis title for adjusting a line break - can be defined here or above
%\sectitlelength{14.5cm}
% nastavení délky bloku s titulkem nad prohlášením pro úpravu zalomení řádku - lze definovat zde nebo výše / setting the length of a block with a thesis title above declaration for adjusting a line break - can be defined here or above
%\dectitlelength{14.5cm}

% řeší první/poslední řádek odstavce na předchozí/následující stránce
% solves first/last row of the paragraph on the previous/next page
\clubpenalty=10000
\widowpenalty=10000

% checklist
\newlist{checklist}{itemize}{1}
\setlist[checklist]{label=$\square$}

% Nechcete-li, aby se u oboustranného tisku roztahovaly mezery pro zaplnění stránky, odkomentujte následující řádek / If you do not want enlarged spacing for filling of the pages in case of duplex printing, uncomment the following line
% \raggedbottom

\begin{document}
  % Vysazeni titulnich stran / Typesetting of the title pages
  % ----------------------------------------------
  \maketitle
  % Obsah
  % ----------------------------------------------
  \setlength{\parskip}{0pt}

  {\hypersetup{hidelinks}\tableofcontents}
  
  % Seznam obrazku a tabulek (pokud prace obsahuje velke mnozstvi obrazku, tak se to hodi)
  \ifczech
    \renewcommand\listfigurename{Seznam obrázků}
  \fi
  \ifslovak
    \renewcommand\listfigurename{Zoznam obrázkov}
  \fi
  % {\hypersetup{hidelinks}\listoffigures}
  
  \ifczech
    \renewcommand\listtablename{Seznam tabulek}
  \fi
  \ifslovak
    \renewcommand\listtablename{Zoznam tabuliek}
  \fi
  % {\hypersetup{hidelinks}\listoftables}

  \ifODSAZ
    \setlength{\parskip}{0.5\bigskipamount}
  \else
    \setlength{\parskip}{0pt}
  \fi

  % vynechani stranky v oboustrannem rezimu
  % Skip the page in the two-sided mode
  \iftwoside
    \cleardoublepage
  \fi

  % Text prace / Thesis text
  % ----------------------------------------------
  \ifenglish
    % This file should be replaced with your file with an thesis content.
%=========================================================================
% Authors: Michal Bidlo, Bohuslav Křena, Jaroslav Dytrych, Petr Veigend and Adam Herout 2019

\chapter{Introduction}

This text serves as example content of this template and as a recap of the most important information from regulations, it also provides additional useful information, that you will need when you write a technical report for your academic work. Check out appendix \ref{jak} before you use this template as it contains vital information on how to use it.

Even though some students only need to know and comply with the official formal requirements stated in regulations as well as typographical principles to write a good diploma thesis (bachelor's thesis is a diploma thesis too -- you get a diploma for it), it is never a~bad idea to familiarize yourself with some of the well-established procedures for writing a~technical text and make things easier for yourself. Some supervisors had prepared breakdowns of proven procedures that have lead to tens of successfully presented academic works. A~selection of the most interesting procedures available to the authors of this work at the time of writing can be found in chaptes below. If your supervisor has their own web page with recommended procedures, you can skip these chapters and follow their instructions instead. If that is not the case, you should read the respective chapters proir to consulting your supervisor about the structure and contents of your academic work.

Diploma thesis is an extensive work and the technical report should reflect it. It is not easy for everyone to sit down and simply write it. You need to know where to begin and how to progress. One of many viable approaches is to start with keywords and abstract, this helps you establish what the most important part of your work is. More on that in~chapter~\ref{abstrakt}.

Once the abstract is finished, you can start with the text of the technical report. The first thing you should do is create a structure for your work, that you'll later fill with text. Chapter \ref{struktura} provides basic information and hints on writing a technical text, that can help you avoid mistakes beginners make, create chapter titles and figure out what the approximate length of individual chapters should be. The chapter concludes with an approach that should make writing a thesis much easier.

Diploma theses in the field of information technology have a specific  structure. The introduction is followed by a chapter or chapters dealing with the summary of the current state. The next chapter should evaluate the current state and provide a solution, that will be implemented and tested. The conclusion should contain evaluated results and ideas for future development. Even though the chapter titles and their length may differ from other theses, you can always find chapters that correspond with this structure. Chapter \ref{kapitoly} deals with the contents of chapters that commonly occur in dimploma theses in the field of information technology. Most students will only use a subset of all the described chapters as not everything will be relevant for their thesis. The descriptions and hints provided help students with the inner structure and the contents of chapters as well as decide whether or they should even include given chapter. 

The final chapter of a thesis is always followed by a list of references. Citations that this list is comprised of and their respective links is the subject of chapter \ref{citace}. An inexperienced student may not perceive it that way, but the list of references is a vital part of a thesis. One of the important aspects of your reviewer's evaluation is how you work with literature. A single missing entry can lead to an F for your grade, disciplinary proceedings for plagiarism and ultimately to being expelled. There are other consequences to this as two czech ministers resigned over allegations of plagiarism in 2018. Be as thorough as possible in creating your list of references.

When you're done with the text, it is necessary to figure out what the requirements for a thesis at BUT FIT are and work the kinks out. Formal requirements that are stated in~regulations and at faculty web pages can be found in chapter \ref{formality}. This chapter also contains information about the required length of different types of academic works and other helpful information. The chapter concludes with an overview of the most common mistakes that the reviewers have to deal with and that you should avoid. The review of the formal aspect of thesis is just another important part of the reviewer's assessment.

Once you deal with the formal deficiencies, you can sumbit your thesis. Before you do so, go through the checklist in appendix \ref{checklist}. The submission of paper and electronic versions of a thesis is described in chapter \ref{odevzdani}.

Chapter \ref{zaver} contains a summary of what you can learn by reading this text, and most importantly things to keep in mind before you submit your thesis.

\chapter{Abstract}
\label{abstrakt}

Ther should be a summary of work at most 10 lines long under the Abstract heading. Despite how short it is, a good abstract provides enough information to know what the problem is, what was the chosen solution as well as the results achieved. The purpose of abstract is to let the reader know whether or not they can find the answer to their question here. The rest of this chapter was taken from professor Herout's blog \cite{Herout}.
\bigskip

\noindent First and foremost - abstract matters. Second - It's not that hard to write one. Without further ado, let's dive into it.

\subsection*{What is the purpose of an abstract}
An abstract is used for \bf searching \rm purposes, together with the title of thesis and a list of keywords. These parts (perhaps except for the title) are not directly part of the text and it's not expected that anyone who will read your thesis actually reads them. The fact that they're reading your thesis means they're past the abstract stage. Abstract serves them well to decide \bf whether or not \rm they want to read your thesis.

When someone looks for an answer to their problem, they give the librarian or a search engine (these days) keywords that directly relate to their problem. They then receive list of theses, that could possibly offer a solution based on the match between the keywords used and keywords in the theses. A good thesis title can help the person guess which texts could have a direct relation with their problem and can get them to read your thesis.

This is where abstract is crucial. The reader reads abstracts of the theses and decides, whether or not they want to read them. It also informs the them that their filter based on a title alone is wrong.

At this point, they don't have a PDF with full text or a printed version of the thesis available. Abstracts are \bf not \rm supposed to be in the text itself, but to be available on servers aggregating scientific texts. Therefore the first rule is: an abstract needs to work on it's own -- if it contains references to literature or text (``The efficiency of a method is summarized on page 51.''), it only makes a reader less interested in the author, won't read their work or cite them.


\subsection*{When and how to write an abstract}
It makes the sense to write an abstract when the writing is done -- as a summary and real annotation of the thesis.
I however like the opposite approach -- write an abstract in the beginning. Whenever I write a scientific article, I start with a long list of keywords that are related to eachother. It's a lot more than end up as the final keywords used for indexing. It help me understand where the article is headed at all times -- what should I talk about, what needs to be in the text, what does it deal with. As soon as I'm done with keywords, I form a title and an abstract.

I consider the following four parts of an abstract especially useful -- Which problem does it solve? What solution does it offer? What are the results? What is the meaning of these results? Once all of this is clear, the text essentially writes itself. If this is unclear, how on earth can you form a coherent, meaningful sentence in the same text?


\subsection*{Recommended structure of the abstract}
An abstract of a scientific thesis can consist of four parts and be useful. Each individual part consists of two to three sentences, in some cases even a single sentence is enough.

The term ``elevator pitch'' is often used in bussiness. It is not a coincidence that its structure is similar to the recommended structure of an abstract. Realistically, an abstract should contain anything the author would say about their scientific thesis if they had at most 2 minute and could not use slides, images or text. What should they talk about then?

\paragraph{Part one -- What is the problem? What is the topic? What's the goal of the text?}
\begin{itemize}
  \item{This thesis deals with.}
  \item{The goal of this thesis is.}
  \item{My aim was.}
\end{itemize}
There is no place for fairy tales specific to wrong scientific literature: ``Our five-year-plan of~work open new and bold goals for us'', ``With the evolution of computing technology and especially the display devices, it is more important than ever \ldots'' do not belong anywhere near a good text, especially an abstract. If you can express the purpose of your text in one sentence, do it and forget about everything else. Less is always more when it comes to the abstract.

\paragraph{Part two -- How is the problem solved? Is the goal fulfilled?}
\begin{itemize}
  \item{I solved the problem using this and that.}
  \item{I used this method, this procedure and analysed this.}
  \item{The work represents an algorithm that.}
  \item{I used these tools to process data and evaluated results like this.}
  \item{The principle of our algorithm is.}
\end{itemize}

There is a new methodology in the nature of scientific text (= ``how to do something''), it needs to include a description. If the solution consists of three parts, it probably means that this part of an abstract will have three sentences, where each sentence is about a~different part of the solution. A good abstract is be honest and accurate in this section -- no ``revealing secrets'' in the text itself. Vague formulations of a solution principle in~an~abstract usually means that the authors can't write or don't have anything to write about -- neither one is good enough to waste your time.

\paragraph{Part three -- What are the results? How good is the solution?}
\begin{itemize}
  \item{It was 87,3\,\% successful.}
  \item{We created a system that.}
  \item{The solution offers these options.}
  \item{As a result, we found out that.}
\end{itemize}

Stating a specific number is not a bad habit in this part -- ``we made existing XY method five times faster''. If the contributon of your work cannot be summarized in two or three sentences, something is wrong and the entire text is probably not worth writing.

\paragraph{Part four -- Well then? What does it bring to science and the reader?}
\begin{itemize}
  \item{The contribution of this thesis is.}
  \item{The primary discovery is.}
  \item{The primary result is.}
  \item{Based on the data it is possible to.}
  \item{The results allow us to.}
\end{itemize}

When writing scientific articles, I myself struggle with the similarity of third and fourth part. Both of them speak about the results and contribution of the text. The goal of the third part is to be specific and name achieved results whereas the goal of the fourth part is to interpret their meaning and significance. I guess it's fine if these two statements merge to an extent and both parts not only don't have their own paragraph, but they sometimes even intertwine with their common sentences.

\paragraph{Part zero -- What is it about? Where are we?}
\begin{itemize}
  \item{The context is this and this.}
  \item{It deals with studies of this and that.}
  \item{We build on these recent advances in our field.}
\end{itemize}

Sometimes it is necessary to insert a short specification of context at the beginning of your abstract. It can be a great asset when it comes to obscure and esoteric field, that is off to the side of the main flow. Usually this part is not needed and sentences contained here are prime examples for pseudoscientific nonsense. It's not necessarily bad to forget that this part can even exist in an abstract. If an expert in the field shakes their head after reading an abstract: ``I have no idea what this is about.'' only then it makes sense to include this part to specify context.


\subsection*{Innovation is not ignorance}

In this text I describe a general model of a general thesis. I would like to state that this is my opinion and taste and I'm interested in alternate opinions and tastes (I really am!). Every graduate (Mgr. and Bc.) feels that their thesis is special and extraordinary. Therefore they won't follow some scheme meant for common and average theses -- i.e. the others. I see good reasons to divert from the outlined scheme and recommend some students to divert from the scheme myself every year. Indeed, every thesis is unique and extraordinary. If they weren't, there would be no reason to write them, just copy them instead. Before you divert from the standard and canonical way of organizing scientific text, put some effort into learning, understanding and tackling it. The way of scientific work, structuring scientific text or citing sources can look rigid and clumsy, but for now it is the best way mankind could come up with. If you learn it, understand it's advantages and disadvantages and innovate it, it's great and you're welcome to do so. If you choose to ignore it, you'll most likely end up with a very poor innovation.

\chapter{Drafting the basic thesis structure}
\label{struktura}

This chapter contains a selection of useful hints and procedures useful for writing a dissertation (bachelor's thesis is technically a dissertation -- you receive a diploma for it) from experienced supervisors. First a number of general princples are listed, followed by a more detailed description of the advised procedure of drafting a thesis structure.

Before you into it head first, ask your supervisor for any advices or if they have their own web page with hints and guidelines. Their area of expertise will probably coincide with that of your thesis and help you with the most appropriate structure that you should comply with. If the authors learn about a another collection of useful hints, you'll find it here in the future.

This text deals with the general recommendations and thesis structure, that always has to be modified and thought of based on the specification \cite{Cernocky}.


\section{Useful hints for writing a technical text}

The following instructions can also be found on faculty web pages\cite{fitWeb}. The overview of basic typography and the creation of documents using \LaTeX{} system can be found in Jiří Rybička's book\cite{Rybicka}.

One of the evaluated parts of a potential engineer is quality of language and literacy. Your goal is to create a clear and comprehensive text. You should express yourself accurately, use the appropriate level of Czech or Slovak grammar (or English if need be) and comply with the generally accepted customs. Slang words and phrases are not allowed. If you are uncertain of the translation or transcription of foreign terms, use literature available in FIT library.

The text should be a short path to understanding a problem, predict it's difficulties and preventing them. Good writing means perfect grammar, correct punctuation and use of appropriate words. You should strive for a good text, one that is not monotonous due to use of small selection of words, or overuse of some of your favorite words. If you use foreign words, it is expected that you know their exact meaning. Obviously, you need to use english words correctly as well. For example, there are certain rules when using the word {\it obvious}. Is {\it obvious} really obvious? And did you make sure, that the {\it obvious} is really valid? You should be careful about using the subject {\it it} with a passife voice too often. For example, you should never use the Czech {\it it has proved itself that...}, ever.

It is advised to think the use of symbols for {\it labelling} through thoroughly. By this we mean a well thought out choice of abbreviations and symbols used for distinguishing types of parts, labelling program's main functions, naming control keys on keyboard, naming variables in math formulas etc. Apt and consistent labelling can help reader a lot. It is advised to provide a list of labelling at the beginning of text. Not just in labelling, consistency is important in references and in typesetting in general.

There are numerous typographical principles that can be used to distinguish things. You can choose different styles for different purposes. As an example, keys can be placed in a rectangle, identifiers from source text can be written in {\tt typewriter font} etc.

Whenever you state facts, you should also state their origin and your attitude to them. If you claim something, you always have to explicitly state, which part of it was proven, what will be proven in your text and what you took from literature (including it's source). You should never let the reader doubt whose idea they're presented with, your own or someone else's.


If you want to write a clear and comprehensive academic text, you need follow these rules:
\begin{itemize}
\item You must have something to say,
\item you must know, who you want to say it to,
\item you must thoroughly think through the contents,
\item you must write in a structured way 
\end{itemize}

\subsection*{You must have something to say}
The most important prerequisite of a good academic text is to have an idea. If the idea is significant enough, it will last, even if it is not formulated in the best way. However, if you want to articulate your idea as precisely as possible and save the reader's time, you must comply with certain principles discussed further below.

\subsection*{You must know, who you want to say it to}
Another imporant prerequisite of a good academic text is the audience. If you write down notes for yourself, you usually write in a differently than when you write a scientific report, article, thesis, book or a letter. Depending on who the target audience is, you can decide the writing style, the amount of information and how detailed it is. 

\subsection*{You must thoroughly think through the contents}
You must thoroughly think through the contents of your thesis as well as the order in which you want to present the reader with your ideas. As soon as you know what you want to say and to whom, you need to create a structure. The ideal structure should be one that is logically accurate and psychologically digestible, where everything has it's place and it's individual parts fit into each other. All the conncetions between them are clear and it's apparently what belongs where.

To achieve this goal, you must precisely structure the contents of your text. Decide what the main chapters and subchapters are going to be, as well as the connection between them. A diagram of such structure would be a tree-like graph, not a chain. When structuring the contents, it is important to ask yourself what you want to include as well as what you want to exclude. Too much detail could discourage the reader, and so could lack of detail.

The result of this stage is an outline of the text consisting of the main ideas and all the details between connecting them to eachother.

\subsection*{You must write in a structured way}

You must write in a structured way and work the most comprehensible format simultaneously, good writing and perfect labelling included. If you have an idea, you know who your target audience is, you know your goal and outline of the text, then you're ready to start writing. When writing your initial draft, you should try to include all your ideas and opinions regarding the individual chapters and subchapters. You need to explain every single idea, describe and prove. The main idea should always be formulated in a main clause, rather than the dependant clause.

You should approach the writing itself in a structured way too. As you're working on the structure of your thesis, you're creating a framework that you are gradually completing. You should use a DPT\footnote{Desktop publishing (DTP) - creating printed document on a computer.} program that offers a structured layout of text (pre-defined types of headings and text blocks).

\subsection*{It will never be perfect}
Once you have written everything you thought about, you should take a day or two off, then read what you wrote again. Make last changes and move on. You should know that something will always remain unfinished, and that there is always a better way of explaining something, but every stage of the editing must come to an end.

\section{Who is the target audience of a thesis}

This subsection was taken from professor Herout's blog \cite{Herout}.

\bigskip
\noindent \bf Write your thesis for a student, that will build on your work. \rm
\bigskip

Imagine that another student will continue the work in your thesis, they're about as~smart as you are. You have four hours to show them your thesis, explain everything necessary for them to be able to continue your work. The student studies at the same school and knows about as much as an average student would, they're not an expert in the area of your thesis, but they're not stupid either. The student just found out that they'll continue your work, so they had no time to learn about the topic, just like you.

It's best to start by telling the student what the goal of this thesis is and what the results should look like.

Nobody in their right mind will spend an hour explaining things like this to the student: ``{\mbox{Interet} was invented by the american army in 1962, then www was invented in CERN in 1991, nowadays it is used for various purposes.''}(all of this on six pages with countless references and figures).
The student usually doesn't need countless pages of details regarding colorful modules to represent figures, history and details of Hough transformation, complete description of the ISO/OSI reference model layers or a set of pie graphs representing individual mobile platforms on the market over the last decade.
The student needs to be shown the valuable sources that helped you and wants a brief description of tools and algorithms that were a vital part of the solution: ``{You need XY tool that does this and that, especially it's PQ module, that is used then and then. The most useful information can be found in~this documentation.}''

Tell the student everything about what worked for you and what helped, but also make sure to tell them what seemed like a good idea, but ended up being useless.

Try to provide just the right amount of detail. Explain an optimization method one line of code at a time, a module in one paragraph enriched with description of input and output data, and a reference to the respective library.

Imagine this four hours long session with the student.
\begin{itemize}
  \item{What do you think you would talk about in the beginning, how long did it take for the student to understand?}
  \item{What are some things that should be mentioned?}
  \item{What kind of figures would you draw during the session?}
  \item{What would the student ask about, what is important but not clear?}
  \item{What restrictions and unfinished things would you need to inform the student about to prevent them from falling into a trap?}
  \item{How can the student continue? What is left unfinished and is worth trying? What could be improved?}
  \item{What would you say in the very first and very last minute of the session?}
\end{itemize}


\section{Thesis structure -- Five chapters}

This subsection was taken from the blog of professor Herout\cite{Herout} (partially inspired by a book from Jean-Luc Lebrun \cite{Lebrun2011}) and from a document on professor Zemčík's web page \cite{Zemcik}.
\bigskip

Thesis is something that students work on for 2 semesters of their studies and then write a small book about it. The misconception is that this little book is the master's thesis. The book is in fact a technical report about the year long work and master's thesis represents the result of student's work.

The year long work includes studies first and foremost: ``What already exists in the area of my thesis? How did the others do it?'' It is implied that a student tries to innovate and design some things: ``The problem has several solutions, I chose this approach, because it is the most efficient option for the given platform.''
The researcher should implement and evaluate their designs to validate them: ``I used these tools for implementation, the entire system is split into these modules. The result is this fast, it is this effective and user reviews are such and such.''

The basic structure of master's thesis according to professor Herout is as follows:
\begin{enumerate}
  \item{Introduction (1 page)}
  \item{What had to be studied (including assessment of the current state; 40\,\%)}
  \item{New ideas that this thesis explores (30\,\%)}
  \item{Implementation and evaluation (30\,\%)}
  \item{Conclusion (1 page)}
\end{enumerate}

If the text has these 5 chapters, it is not a mistake, neither is if one of them is split in two parts -- more on that later. A big mistake is when any of the parts are missing, or when it's content differs from the rest.

Names of chapters can vary from this structure, even though your thesis will have an introduction and a conclusion. The content of your thesis itself is what really matters and if it means you have to break the structure, then do so.

Many supervisors agree with this basic structure, even though some of the recommend different chapter titles and for example assessment of the current state can be used in second chapter, and even in third chapter according to professor Zemčík:
\begin{enumerate}
\item Introduction (1--2 pages)
\item Summary of the current state (40--50\,\%)
\item Assesment of the current state and solution draft (3-5 pages)
\item Your own work (roughly 40\,\%)
\item Conclusion (at most 1 page)
\end{enumerate}

Opinions of supervisors also differ in length depending on the specification of thesis, this can be seen for example in recommendations from assistant professor Beran \cite{Beran}:
\begin{enumerate}
\item Introduction (1 page)
\item Theory (1/3 of pages)
\item Solution draft (1/3 of pages)
\item Implementation, experiments and assessment (1/3 of pages)
\item Conclusion (1 page)
\end{enumerate}

When it comes to practice-oriented theses, where the most important things are data and user interface, associate professor Černocký \cite{Cernocky} recommends the following:
\begin{enumerate}
\item Introduction (single digit of pages)
\item Theoretical part (roughly 10 pages)
\item Data (single digit of pages)
\item Breakdown of Your algorithm and testing (roughly 10 pages)
\item Draft and implementation (several pages)
\item User interface (several pages)
\item Testing (roughly 10 pages)
\item Conclusion (single digit of pages)
\end{enumerate}

\section{Thesis -- comics edition}

This subsection was taken from professor Herout's blog. \cite{Herout}.

Thesis (including bachelor's) is a fairly complex work comprised of a large amount of letters. These letters follow a certain hierarchy, organised into chapters. The whole thing should follow a logical order -- reader should first learn one thing so that they can then learn and understand other things. It should contain figures, tables, formulas; these non-textual objects should fit in with the surrounding text and enrich it. Every single aspect of the specification of your thesis has to be explored, it has to be finished by a certain date, printed and bundled into a book. If you want your thesis to be good, you need to make it good. Whenever you bake a cake from ten ingredients and one of them is rotten, it doesn't matter how hard you try, the cake will not taste good.

\subsection*{How should you approach everything?}

Whenever we write an article (all the time lately), we create something called a ``Comics Edition'' in the early stage. We do it because I insist that we do it and because it helps us a lot. Perhaps it can help you with your thesis.

First, make sure you know answers to the following questions:
\begin{itemize}
  \item{How would you explain the principle of your solution in three to five short sentences?}
  \item{What are the strenghts of your solution?}
  \item{What arguments would you use to support the correctness of your solution?}
  \item{If someone wanted to criticize your thesis -- what would they criticize it for?}
  \item{What would your answer be?}
  \item{What keywords should one use in a search engine to find your thesis in the search results?}
\end{itemize}

If you are finished, let's get into it...

\subsection*{Create THE document}

I often see people write an ``initial'' version of a thesis somewhere in their notes. First of all, that's extra work and second it is entirely pointless. It's best to create a document where the fight begins and also ends.

\subsection*{Chapter titles}

Chapter titles are an important part of a comics edition, put them in your document.

Put them in for the sake of formatting -- no provisional bullet point lists: ``I'll just do it later''. You need to see how the automatically generated content looks now, before it's too late to change everything.

Put them in for the sake of wording. Chapter title says, what the chapter is about. Chapter titles represent a skeleton covered in flesh and skin of text and figures.

Of all the words in your thesis, words in the titles are the \textbf most important ones\rm. Put some time and effort in your titles.

\subsection*{Figures}

Picture is worth a thousand words. I went through the last 8 articles that I helped write. They're 80 pages total and contain 87 figures and 17 tables, that's 1,3 of visual information per page (including pages containing references, title pages with abstracts etc). Many figures (about a half) are comprised of multiple subfigures, especially referenced figures. I~counted 221 of subfigures in the mentioned articles, that's an average of 3 visuals per page. This is what I imagine the role of figures in a scientific text should be. I think a~thesis with ``too many figures'' does not exist.

You should think about what images you will use and where they'll be even in the early stage of comics edition.

Figures don't have to be finished. We don't know what exactly the figure will represent just yet. We don't know what the caption will be. We only know that there will be a figure and that it will be comprised of multiple subfigures. It takes roughly a minute (we already have a vector format of a ``TODO Image'') and it shows us how the text will look.

In some cases, we know what the images will look like on a conceptual level. Draw it on a paper or a blackboard, take a picture of it and insert it where the final figure will be (vector image made in Inkscape\footnote{\url{https://inkscape.org}} or generated using Gnuplot\footnote{\url{http://www.gnuplot.info/}}). 

As a side note: Picture is worth a thousand words. Stupid picture is woth a thousand stupid words.

Just one more thing on the side when it comes to figures: If instead of vector images (schemes, graphs, drafts, diagrams, esentially everything except photos and screenshots) you use bitmap images and you use screenshots with a lossy compression (usually JPEG), don't expect a positive feedback or assessment.

\subsection*{Quantity of text}

Just like with figures we don't have yet, we insert even text that we don't have yet.

LaTeX has a beautiful command \textbackslash Blindtext\footnote{Short tutorial: \url{https://texblog.org/2011/02/26/generating-dummy-textblindtext-with-latex-for-testing/}} just for that. If you don't want to use it or don't know how, use \url{http://lipsum.com}. It helps you estimate the length of your work, figure density in text and other characterstics. It takes roughly 5 minutes to create this sort of estimate. To make sure you don't get lost in your work-in-progress text, change it's color to grey (smart person creates a command to change the color of the text unanimously for the whole document). From experience, it's easy to get lost without colors -- what is finished, what isn't and what needs to be worked on. It is advised to spend a couple minutes getting a text coloring package to work. You can use command \verb|\todo| in this template, for example \todo{This needs to be finished}.

The genesis of each chapter begins with 3-5 TODO pieces and some Lorem ipsum. Each TODO is then transformed into a larger number of TODOs of a smaller scale or into text, figure, other subchapters, and pretty much anything. TODOs come and go, but they always move you closer to the final product.

\subsection*{How do you work with it}

When you sit down and LaTeX is having a good day, it takes you an hour to write a thesis (bachelor's thesis included), with the right number of pages and a pretty good idea about where everything will be. It is slowly forming the result that is yet to come, after some more work.

The document does not really grow in size anymore, but it transforms. There's a~big difference between sitting down in front of a blank page and ``write a thesis'' and transforming one TODO into a paragraph. The first one is difficult and sometimes you just can't make it work. The latter works: it has it's head and tail. At least you know what to do.

Still, the thesis won't write itself, but it gets a lot easier and the final result is that much better.


\section{Chapter titles}

To publish well does not mean to fill up as many pages with letters as possible. Scientist's achieve their renown when their work is useful enough to another scientist, that they end up citing them. Therefore it is important that one's article is well written: nobody will cite a thesis that is poorly written because it degrades them.

Though poorly written article won't be cited because \bf nobody can find it\rm. Long before the internet SEO even existed, scientiests used all kinds of methods to make sure that other scientists working on their recherche will include others' materials that they read, make notes and finally -- cite in their work.

There are a lot of articles in the world. Whenever a scientist searches for materials relevant to their work, they enter keywords (formerly on paper in library, nowadys electronically into a search engine) and they get a lot of results -- e.g. titles. First step to get someone to cite an article is to have a good title. Title so good that a scientist is interested in your article enough to actually read it and find out more. Title is the \bf first filter\rm.


The scientist then opens the articles that satisfy the requirements of first filter. That means they get to see the abstract of the article. Abstract is the \bf second filter \rm and quite important one. It's similar to getting to a second round of interview for your dream job. People tend to care at that point.

If the scientist is interested in the title and the abstract, they download the entire PDF and scroll through quickly to get an idea: print it, or close the window and look for other articles? That is the \bf third filter \rm and it's similar to being among the last few dream job applicants. What are the scientist's interests in the third round? Visuals, i.e. figures, tables, formulas, and chapter titles. Does your article make it through the third filter? Will the scientist use your article? We'll talk about figures another time, this chapter is about titles after all.

It can be slightly different when it comes to theses. Not every author wants people to read their thesis. We belive that there are people who wish for the opposite. Let's just work with the hypothesis that the author wants to write a good text, that mankind could find useful and worth reading. One that gets a decent grade.


\subsection*{Keywords -- half the success}

One of the best advices to write an article (scientific text in general) I heard is not entirely intuitive and obvious.
\bf Write a list of key words, that one should enter in search engine to find your work as a relevant search result. \rm

Let your fantasy roam free. Think about how your work can be utilized and concepts connected to it. Write down all the keywords, this will be a couple lines. Keyword can even be a phrase -- typically two or three words.

Choose the important ones. This step needs some intuition and experience. I don't really know where you get those, but you can ask someone for help. By the time you write your thirtieth article, it'll be easy.

\bf All the important keywords need to occur in the article title or in the title of chapters. \rm 

\subsection*{Title too general, title too specific}

So what's up with the keywords? Titles are essentially pointers, ponting you in a certain direction: ``The thing you're looking for is here!'' For someone to appreciate your text, they need to not get lost in it. They need to know that the text offers answers to some of their quesitons. Titles can tell them this is the article they're looking for or to not waste time.

If you're moving and you write ``stuff'' on all of your boxes, you have a truthful and formally correct labels, but you didn't really need to write anything. If they're too general, they're useless.

Chapter titles that contain one or two words are usually the primary suspect of a title too general -- except for the conclusion, where chapter titles are canonical and I would avoid experimenting with them. If there are more single word chapter titles than the two mentioned, they're probably wrong.

Chapter title that could be used in a different thesis of the same branch e.g. ``System implementation'', ``Image processing basics'', ``User interface principles'' is suspicious of being too general. Something like ``Fly movement monitoring system implementation'', ``Algorithms to detect objects and track thier trajectories'' or ``Principles of simple web systems user interfaces'' is much better.

Chapter title that could be used on a completely different faculty is essentially always wrong -- way too general. Title such as ``Theory'' could be used at a university of agriculture, IT, law, university of milk and cheese. It is wrong. Title such as ``A study of existing solutions'' is wrong. ``Exploration of available technologies'' is wrong.

I have never seen a title that is way too specific, and I don't believe anything like that even exists. It can be wrong -- not describe, what the chapter actually contains. And if it does, it's never too specific.

I don't want to see five lines long titles. Vast majority of good specific titles will be on a single line and they'll contain roughly five words. Every now and then a title won't fit in a single line and there will be a good reason for it. Good titles -- specific enough and not too long -- are not easy to put together, and it requires thinking. Much like anything that should be worth something.


\subsection*{Abbreviations in titles}

There is no place for abbreviations in titles, unless they're known worlwide (ČR, AIDS, IT).

You can explain a term in chapter one and say that it will further be referred to with an abbreviation. This abbreviation can then be used in the text of chapter without any further explanation. However, you can't use the abbreviation in the title of chapter two, because a scientist reads chapter titles in third filter when they decide whether or not to even read chapter one. If the scientist gets the feeling that the text is weird, not clear and that they don't really know what it is about (in case of abbreviation in title), they close the article and don't open it again.

Literature references and references to objects in articles (figures, titles, \ldots) do not belong in the title and I've never seen a case where they would be needed (and I've seen quite a few cases, where they occured).


\section{General advice from experienced supervisors}

This subchapter contains a selection of hints from other experienced supervisors, whose students have successfully presented hundreds of academic works. They even took the time to write comprehensive articles containing their recommendations and posted them online. To read the full texts, you can visit their web pages, the links can be found in references \cite{Beran}, \cite{BeranPDF}, \cite{Cernocky}, \cite{Zemcik}.

\subsection*{General advice from assistant professor Beran}

This subsection was taken from assistant professor Beran's web pages \cite{Beran}, \cite{BeranPDF}.

How to write/How not to write
\begin{itemize}
  \item{use chapter numbers up to second level, leave titles of a lower level without a number and don't include them in the table of contents, the final thesis and it's content is much less cluttered}
  \item{Logical structure -- each object -- whole thesis, each chapter, each subchapter has: introduction, treatise and conclusion:
  	\begin{itemize}
  		\item{introduction -- tells the readed what it is about, what are the contents, what can we find out, what the problem is and introduces the context}
  		\item{treatise -- deals with context, problem, details of a problem, solution, steps and the results}
  		\item{conclusion -- summary of the task, achieved results and their meaning, concludes the work}
  	\end{itemize}
  }
  \item{it's a scientific text, leave your emotions and opinions out of it}
  \item{don't use ``WE'' did, wanted, etc.
    \begin{itemize}
      \item{use passive voice, ``tests were carried out'' rather than ``we carried out tests'' -- especially in the theoretical part, where thoughs taken from elsewere occur,}
      \item{if you want to emphasize Your contribution, Your work, Your idea, etc. use ``I'' -- designed a solution, experients, realization}
      \item{because WE (me-you, you-reader, you-world) didn't do anything, YOU did}
      \item{(ignore the fact that you're using my idea, that is expected, it is your thesis on my topic)}
    \end{itemize}}
  \item{when you use figures/ideas/tables from elsewere, \bf cite the source \rm}
  \item{each \bf title \rm (of a chapter or subchapter) should be followed by a paragraph of text that informs the reader about what they can find in the following section}
  \item{don't underestimate introduction and conclusion}
\end{itemize}


\subsection*{General advice from associate professor Černocký}

This subchapter was taken from associate professor Černocký's web pages \cite{Cernocky}.

\begin{enumerate}
  \item{Read a handful of good theses and try to understand what makes a good thesis. Your supervisor will gladly give you some examples.}
  \item{\textbf{Czech/Slovak or English?}
  	\begin{itemize}
  	\item If your english level is decent and there's a chance that someone else outside of BUT FIT will read your thesis (part of international project, work for an~international company, SW description that you want to submit to GooglePlay etc.), I highly recommend you write everything in english. You can tell yourself that you'll translate it later, but there isn't really time for that. On the bright side, you don't have to worry about diacritics.
    \item If you work on a thesis of a local significance and know, that your english isn't that great, I suggest you save yourself, your supervisor and the reviewer the trouble and write in czech or slovak. More information as well as common mistakes students make can be found in appendix \ref{anglicky}
  	\end{itemize}
  }
  \item{Don't worry about the number of pages! Don't write about nonsense, don't copy needless things from wikipedia -- this only makes your supervisor and the reviewer angry. If you follow the advised structure and write about what you have actually accomplished, the final product will be worth it.}
  \item{Take your time -- you don't have to write the text of your thesis in it's final state, but keep some sort of README file to write down what you're working on, current results, what you read, what was it about, etc. I highly recommend you avoid the ``work first, write later'' approach -- in six months time, you won't know what you did and you'll have a hard time remembering or worse, you'll have to relive your past. Writing your thesis as you work helps you keep everything organized and structured.}
  \item{Use a spell checker. Save your supervisor and the reviewer the trouble of correcting stupid errors (typos, etc). MS Word has a pretty good spell checker, linux has a~decent ispall/aspell/hunspell utility (called from popular text editors such as Emacs). Some spell checkers are useless, e.g. the one in PSPad ignores a lot of errors.}
  \item{Give your supervisor a draft of your thesis -- in advance! -- one chapter at a time is probably the best approach or at least two weeks before you submit the final version (tables with results/conclusions don't have to be finished). Your supervisor will cross most of it out, you'll get mad at them, how dare they ruin your (nothing short of beautiful!) work. Once you cool off, you realize they're right, fix everything and the next version will be much better. If you instead opt to submitting a version that only you've read, it will severely affect the thesis review.}
\end{enumerate}

\subsection*{General advice from professor Zemčík}
This section of text is taken from a document on professor Zemčík's web page \cite{Zemcik}. Generally known and respected principles are written in normal font, whereas text in italics contains \uv{extra} personal recommendations from the dean.

Table of contents should not be longer than a single page and an entry should not be longer than one \uv{line}. The thesis should be split in subchapters equally (except for introduction and conclusion). This principle even has a higher priority than the \uv{Universal Decimal Classification} of a thesis\footnote{Decimal classification according to ČSN ISO 7144 and ČSN 01 6910 - for extracts, see \url{http://web.ftvs.cuni.cz/hendl/metodologie/doporuceniupravydizprace.pdf}}. When it comes to font, do not \uv{waste} font styles and typefaces. Less is most definitely more in this case. Other than basic font, headings, captions of images, tables and equations, you should use as little fonts as possible, e.g. use italics or bold font style only to highlight text (preferably not both at once) and an~appropriate \uv{monospace} font for snippets of source texts. It is important to comply with the pre-written thesis formal template, that can be found on the faculty web pages.

\it As for introduction and conclusion, I highly recommend you don't split the in subchapters and keep them as concise blocks of text. The text above also suggests that it is appropriate to only use first level subchapters. If you ever feel the need to use more headings than an~average of one per page (take a moment to think about if it is really necessary), you can use a \uv{secondary} heading without a number that won't be included in the table of contents. When you're coming up with headings, consider the possibility that they might not be a good enough guide for the reader and if that's the case, you should probably change the headings. It's also a good idea to include a short text after each heading (i.e. chapter heading should be followed by \uv{2.~Work status} and rather than following up with \uv{4.1 Initial status}), include a short text and then continue. Another thing you should try to avoid is ending text with an image or equation. It is advised to conclude a chapter with a brief summary.

Equations, images, charts, headings and others are significant typographical elements of a thesis. Their format, however, is for the most part \uv{strictly} set by the template, which means that you don't have to spend too much time with them. Nevertheless, a few things should be said about their integration in the thesis. First of all, make sure that these \uv{graphical} elements are well separeted from the text and that the result \uv{looks good}. Excuses like \uv{TeX did that} or \uv{Word did that} will be short-lived. As for images, make sure you use uniform style of inclusion in text and that the images are either aligned to center or (if they are not as wide as the page) aligned to the inner side of a page (with respect to the future binding, or strictly right side in case of a single-sided print).

Note: If you aren't happy with what this chapter says about table of contents and you want to \uv{guide} the reader better, make an index -- it is a different typographical element and unlike the table of contents, you can use as many keywords as you want. It is, however, very time consuming and for that reason it's not something I'd recommend.
\rm

The primary language of a thesis is either czech, or english. Thesis cannot be in slovak \uv{officially}, but a thesis written in slovak with czech title and type is tolerated. Please, don't take this as some czech chauvinism, it's a matter of laws and academic field accreditation.

When writing a thesis, use exclusively standard language and avoid colloquialism, or slang (technical slang included).

\it I highly recommend you pay extra attention to the following things, as they, unfortunately, are a common source of errors:
\begin{itemize}
  \item{Try to avoid using first person singular (\uv{I/me}) as much as possible. The use of first person plural, despite the fact that it's commonly used in belles-lettres, needs to be emilimated completely. There are some exceptions though:
    \begin{enumerate}
      \item{You can use first person singular in introductio and conclusion of the thesis as means of stating your \uv{personal opinion} (e.g. \uv{Usually this method is used... , but I chose a different approach...}), it can also be used in the assessment of the current state, but never in summary of current state.}
      \item{First person plural can be used in case you're highlighting a part of the thesis, that you did not do on your own, but in a group. Considering the fact that a~thesis should be primarily your work, you have to use it sparingly and it needs to be obvious, that at least 90\,\% of the thesis is your own work (e.g. \uv{I wrote the program myself, but I asked my peers to help me with testing and together we conducted experiment...}). Avoid questions like \uv{So you didn't work alone?}}
      \item{An exception to both rules stated above are mathematical texts, where first person plural is used often (e.g. \uv{We have a cube of side...}), here's where you can use first person without any limitations.}
      \item{Another possible exception are rhetorical questions, if you ever use them in your thesis. I  recommend using first person singular or first person plural at most ten times in the thesis (except for case 3, don't limit youself there), there is no floor, but using it once or twice is fine.}
    \end{enumerate}}
  \item{English expressions shouldn't really be used in a thesis. Considering the fact that our academic field is full of them, my recommendation would be to state both versions of a technical term (czech and english) if it is a first appearance and put the version that you will no longer use in brackets with a comment if you want (e.g. \uv{... octree (oktálový strom, only the english version will be used going forward, because even experts in field are accustomed to it)...})}
  \item{Abbreviations should be explain when they first appear in the text. Alternatively (better, but more time consuming method), you can create a list of abbreviations, where all the abbreviations are explained in detail.}
  \item{Past/present/future tense should should be utilized as follows, generally speaking a~thesis describes facts (use present tense) combined with a description of your work (that already happened, therefore use past tense). When it comes to future plans for the thesis, you obviously use future tense. However, more than anything else, use your \uv{sense} and adjust the language, so that the thesis is easy to read.}
\end{itemize}
\rm


\chapter{Individual thesis chapters}
\label{kapitoly}

In this chapter, We're going to talk about the meaning and recommended content of individual chapters a thesis. We have also included the recommendations from experienced supervisors. 

Structure of a thesis changes depending on it's aim, progress made over time and achieved results. It's likely that you won't put all the things mentioned in this chapter in your technical report or that you will add an extra chapter that is not mentioned here. It's never a bad idea to plan the structure of your text in advance and consult your supervisor about it.

Individual chapters should be in a logical order. \bf Use references. \rm \it In function XYZ, we implement mathematical formula, that we have derived in section 3.2, equation~7. \rm If you reference things further in the work, describe what you're referencing in roughly two sentences. The reader should not get lost in your work, don't make them flip pages. \bf Each chapter should (within reason) make sense on it's own. \rm If some important terms, abbreviations or thoughts are integral to the whole thesis and appear time and time again, always explain them (as soon as you use them in a chapter). If the reader opens your work in the middle (for example chapter 4), they should not immediately drown in abbreviations and technical terms.\cite{rady}

Unless specified otherwise, the rest of this chapter is taken from professor Zemčík's personal web pages \cite{Zemcik}, blogs of professor Herout \cite{Herout} and assistant professor Szöke \cite{rady}, and web pages of assistant professor Černocký \cite{Cernocky} and assistant professor Beran \cite{Beran}.

\section{Table of contents}
\label{obsah}

Long text -- such as a thesis -- comes with automatically created table of contents. It MUST fit a single page in a thesis. I can see how it could be difficult to fit table of contents in a~single page when it comes to a book seven hundred pages long, but not in a thesis.

In a thesis, table of contant should only have first and second level headings, third level headings are not present as they are a bit too specific. Fourth level -- although it wouldn't be included in table of contents and only used in some chapters -- is just wrong in itself.

If the chapter headings are good, just by looking at the table of contents (no further information, without reading the abstract) any reader that somewhat understands the field must be able to recognize what the thesis is about. They can gauge the goal of the thesis. They know what modules the solution consists of and what the purpose of each module is.
They can tell which and how many experiments were carried out during the research. They can also tell who the target reader of the thesis is -- who and what is it useful for. If they can't tell just by looking at the table of contents, the chapter headings are probably wrong and it's up to the author to fix that, or have a poorly written thesis. There is no third option.


\section{Introduction}
\label{uvod}

The first chapter is titled Introduction. It's purpose is to provide broader context of a~problem and define structure of the thesis in the form of an eptiome.

The length of an introduction should be roughly 1--2 pages of text. It is expected that the introduction is readable even by someone just ``passing by'', who is literate, lives in our time, on our earth and that's about it. They don't have to be an expert in the field. They should still be able to understand the introduction. It should be written in a way that makes it work as a standalone figure of literature and if anyone reads the it, they should understand what the thesis is about. It's also common that people only read the introduction.

\subsection*{Advice on thesis introduction from professor Zemčík}
It is not appropriate to split introduction in subchaptes or include references in it -- it should be a figure of literature that is readable and ``feasible'', and it should be easy to read. When it comes to the inner structure of an introduction, it should be as follows:
\begin{itemize}
  \item{Roughly 5 lines long general introduction for a topic, based on well known words without any scientific terminology if possible (example given ``computer, video'' yes, ``tree structure'' no),}
  \item{short text explaining why the thesis is important in the given field, the importance for world and for us, what the relationship with studied field is and other important facts,}
  \item{short text about the past advances in the field, about it's current state, and even about things to look forward to in the future,}
  \item{you should write about ``why I am interested in this'' too, what lead you to choosing this topic and do you find interesting about it -- truthfully and honestly if possible,}
  \item{it's also a pretty good idea to write about the goals of the thesis (in your own words, not word for word from the specification),}
  \item{and finally, it is important to describe the structure of your thesis to make sure the reader knows where to look for things (example given ``the following chapter contains..., described in chapter ``xxx'', in chapter 3 ...'' etc.) --- don't forget that the introduction should be a manual to your thesis.}
\end{itemize}

\begin{samepage}
\subsection*{Advice on thesis introduction from professor Herout}
It is not supposed to be ``an introduction to the problem'', but ``an introduction to a small book'' (technical report). After reading it, the reader should
\begin{enumerate}
  \item{have a pretty good idea what the book is about,}
  \item{look forward to reading it.}
\end{enumerate}
\end{samepage}

\subsection*{Advice on thesis introduction from associate professor Černocký}

Introduction is basically an ``extension of specification''. It contains:

\begin{itemize}
  \item{Why was it researched -- what is Your motivation?}
  \item{Project background -- is it a part of a research project? describe. Any industrial cooperation? describe.}
  \item{If a thesis is a collective effort, be clear about who is responsible for what}
  \item{What are the specific things you would like to share (``claims'').}
  \item{What can the reader read about and where. \texttt{\textbackslash subsection\{Table of contents\}} etc.}
\end{itemize}

\subsection*{Advice on thesis introduction from assistant professor Beran}

Introduction should be brief (1 page). It should contain:
\begin{itemize}
  \item{Introduction to the problem (we're in the IT field, for example: image processing and not chip manufacturing),}
  \item{goal of the thesis -- one clear goal of a thesis and steps leading to it (there is only one goal, but there are many steps to reaching it (e.g. function library, creating dataset))}
  \item{a brief summary of the entire thesis (an overview of existing solutions, introduce a~draft of my solution based on the existing ones, testing, evaluation, ...).}
\end{itemize}


\section{Summary of the current state}
\label{stav}

This should be about 40--50\,\% of the length of your thesis. The purpose of the part is to familiarize the reader with the current state of technological field that is the subject of your thesis and introduce the apparatus used in thesis (mathematical, electronic, IT etc.) to them. It's not expected that the summary of current state will contain everything directly related to the thesis, but it should contain all the necessary information to make sure that a reader who is at least somewhat familiar with the field of your thesis can understands it. This part should be split in chapters, especially if it is a ``multifield topic''. This part should also ``heavily'' reference literature. The length depends on the type of your thesis, if it's theoretical this part will be much longer than in a practice-oriented thesis.

\subsection*{Advice on summary of the current state from professor Zemčík}

It is appropriate to state at the beginning of this part what it contains and why, and that it is not ``encyclopedic dictionary'' of said field -- to make sure that reader doesn't get the impression that ``something is missing''. You should write the text in your own words, but you can essentially copy the cited literature. If you have to take a whole section of text longer than one sentence, it's necessary to distinguish it properly from the rest and cite the source. A thesis should only have as many of these as necessary and the maximum length of such text is (roughly) half a page.

It is best that you don't express your opinions about the technical content in this part -- the current state needs to be described, but not evaluated. Essentially, if someone wrote something in literature, you can use it in this section. The reason for this is that if an expert in field reads the thesis, they should have the option to skip this part (they're an expert, I'm sure they're familiar with it) and not lose out on anything important in the thesis itself. It is advised to write this part as you make progress, while the literature is still in your brain -- so you don't have to read it again. 


\subsection*{Advice on summary of the current state from professor Herout}

Chapter describing what needed to be studied should take about 45\,\% of the thesis.

I intentionally avoid the word ``theory'' that would fit this part of a thesis well. It's because the word theory has this magical property to trigger the urge to write purposeless text on various topics more or less relevant to the topic of a thesis, but also topics that are very distant.

You should ask yourself ``Is this information necessary to understand what I designed and implemented?'' whenever you're about to write a paragraph. If the answer is no, don't bother.

This part of the thesis text can often be comprised of two chapters. If you're developing a web-based accounting system, one chapter should be about accounting and the other one about safe web systems. Here's a typical example: lots and lots of IT solutions need to be studied as a field of work (where can the system help) and as a tool (process of developing such systems).


\subsection*{Advice on summary of the current state from assistant professor Szöke}

This chapter (there can be more of them) is where you show that you understand the problem. You've studied the usual solutions to your problem, state why your solution is different from the others and why is it better. Theory is a great source of entries for the bibliography. Your grandmother does not need to fully understand this chapter, but it's a good idea to leave the mathematics for a second and summarize what your calculations and derivatives actually mean. It is especially useful should the reader lose themselves in the text, this will help them find their way back.

\subsection*{Advice on summary of the current state from associate professor Černocký}

Theoretical part should be about 10 pages long, perhaps a bit longer when it comes to more theoretical theses. You should only write about the theory necessary for your thesis.
\begin{itemize}
  \item{Only describe what you actually need, we're not interested in contents of a script, book, wikipedia ...}
  \item{If you use formulas, you need to provide an explanation for every single symbol and it needs to be clear what the purpose of each formula in your thesis is. Don't use formulas ``just to illustrate'' or ``to make it look more scientific'' or ``just to try it in LaTeX...''. }
  \item{If a theoretical part is too complicated (for example HMM \& co.), don't write about all of it, write about the basics instead and cite a good source to ``pinpoint'' what is absolutely necessary for your thesis.}
\end{itemize}


\subsection*{Advice on summary of the current state from assistant professor Beran}

Theoretical part should contain:
\begin{itemize}
  \item{existing solutions from the perspective of your specification,}
  \item{what already exists in the field of my thesis, what other solutions of my specification exist,}
  \item{what existing tools and procedures could be used as a part of the solution,}
  \item{any and all theory should be \bf emphasized \rm (state why it is important that the reader is familizarized with it and how it relates to the solution your thesis offers).}
\end{itemize}
    
\section{Data}

Data is the key feature of any project that deals with recognition and this chapter should not be missing. Recommended length is single digit of pages. Describe:
\begin{itemize}
  \item{where did you get the data (producer, catalogue number, etc.),}
  \item{technical description -- e.g. Fs, bit width, number of speakers, audio length, etc.,}
  \item{dividing data in subsets -- training, development, testing, evaluation -- who did what (it's best to use existing subsets).}
\end{itemize}

This chapter will probably not be a part of your thesis, unless you work on for example sound processing for real-time play.

This chapter will be a part of theses, that work with data sets, that are processed daily on different workplaces and allow for comparison of results, or theses that were given data sets for testing purposes.


\section{Assessment of the current state and work plan (draft)}
\label{navrh}

The main goal of this part is to write an evaluation of the current state and draft of your innovative solution.

\subsection*{Advice on assessment of the current state and work plan from professor Zemčík}

It is advised to include this part of thesis. It's length should be ``as long as you need it to be''. The purpose of this part is to set a goal of your thesis based on the assessment of the current state and create your own ``detailed specification'', alternatively set the expected parameters of a solution, but not the the actual solution. This part should contain:
\begin{itemize}
  \item{Critical assessment of the current state (what is correct, what is wrong, what is not researched at all and possibly cost paramateres, availability of solution, necessary computing performance etc.),}
  \item{draft, what should be researched and solved based on the knowledge of the current state and personal preferences, specification, requirements etc.,}
  \item{given options even a specification of thesis, as in ``what should it do'', ``what should the parameters be'', ``what tools will be used'', ``how is it going to be evaluated'', ``how do you know that you were successful''.}
\end{itemize}
It is advised to write a truthful deliberation in this part, especially when it comes to ``draft'' so that the rest of the thesis is credible and to convince the reader that all necessary steps were taken, after reading this chapter and the rest of the text.


\subsection*{Advice on draft of a solution from professor Herout}

This part should contain ideas that the thesis brings:
\begin{itemize}
  \item{I decided to.}
  \item{I devised.}
  \item{I laid out.}
  \item{I calculated.}
  \item{I derived.}
  \item{I simplified.}
  \item{I improved.}
  \item{I designed.}
  \item{I found out.}
  \item{I researched.}
\end{itemize}

Sometimes it's difficult to separate new ideas and implementation. Programmer confuses ``designed'' with ``programmed''.
It's also easy to confuse ``I improved'' with ``The results are''. But seperating these chapters is correct.

Theses in many non-IT fields are structured as research theses called ``scientific methods''. In our environemnt (lets assume engineer's decree IT study), theses don't follow this structure. Our theses (not entirely wrong in my opinion) are more similar to project documentation. It still is the correct way to separate hypothesis from a draft, how to validate them, and from their actual validation or assessment.


\subsection*{Advice on draft and description of your algorithm from associate professor Černocký}

If the purpose of your thesis is ``science'', this part will probably be the longest one and it's advised to split it into multiple chapters -- for example \textbackslash chapter\{Basis\} \textbackslash chapter\{Innovation, ...\} \textbackslash chapter\{Results and discussion\}. On the other hand, if the purpose of your thesis is to try something existing/new, this chapter can be very brief. The length of this chapter should be roughly 10 pages and contain:
\begin{itemize}
  \item{what specifically did you do with the theory described above -- block scheme, setting constants, technical simplification of complicated theory etc.,}
  \item{draft -- can be a simple block scheme or a full object draft, but it should be clear that your SW has structure.}
  \item{choice of OS, programming language and libraries. The point of thesis is not to write everything on your own, you can use any and all free and commercial programs, libraries, modules, etc -- essentially anything -- it's a standard engineer's decree thesis -- the goal is to \textbf{make it}, not to \textbf{write everything alone}. You do, however, need to describe everything in detail and cite sources, not plagiarize work of others!. Libraries have a good, accurate specification -- where from, which version, if it needed to be paid for, how much did it cost.}
\end{itemize}

\subsection*{Advice on draft of a solution from assistant professor Beran}

What it comes drafting a solution, it depends on the specification -- the following points are not general. The length of this part should be about a third of the pages.
\begin{itemize}
  \item{Write from the perspective of a well paid expert on thoughts, that drafts a solution to problem, innovative solution, solution full of interesting thoughts.}
  \item{The ``draft'' part can be perceived as a procedure/tutorial to a solution, this draft is then passed to a team of programmers and testers, that realise and test the draft.}
  \item{If possible, it should be a general draft, without considering implementation on iOS or Android, Linux or Windows, MySQL or Postgres, HTML5 or Flash.}
  \item{Detailed specification analysis, detailed specification and formulation of goal and it's parts.}
  \item{Description of solution application, situations and problems that the project solves.}
  \item{Work procedure or steps leading to the goal, split the whole project into parts.}
  \item{Draft of the entire solution as well as it's parts, including references to theoretical part.}
  \item{Analysis of the results over time (measuring, observing, testing).}
  \item{Draft development and updates.}
  \item{If you update the draft based on the results of tests -- include references to tests and their results.}
\end{itemize}

\section{User interface}

This chapter is only useful in some theses and it should only be a handful of pages long. In some theses, it just doesn't fit at all. If it's necessary, it should be included (even prior to a draft) and contain: \cite{Cernocky}:
\begin{itemize}
  \item{UI concept -- what was the inspiration (existing programs, classic mechanical device...) -- write about it,}
  \item{mockup -- if you made any hand drawings, don't hesitate to include them!}
  \item{how did you choose the final version,}
  \item{if a UI went through more development, for example you weren't quite happy with the first version and made changes based on user reviews, write about it!}
\end{itemize}


\section{Implementation}
\label{implementace}

This part of a thesis should be about the work itself. It should contain information about ``what you actually did'' and it should be about 40\,\% of the entire thesis. It should be clear what the basis of the thesis is, how was it made, what tools were used and what results were achieved.

\subsection*{Advice on describing your own work from professor Zemčík}

When writing this part of thesis, it is necessary to avoid technical details that could distract the reader or even worse, bore them. Important things are concept of thesis, outlines of solution and what lead you to that solution.
Another important aspect of this part is that it must describe how the solution is used, while not being a manual.
Any and all technical details, that are not vital to understanding the thesis (and disrupt the ``flow of the text'') belong in the appendixes part and not the text itself. This mostly applies to long snippets of source code, instructions, tables of results etc. If you include snippets of your code in this part, keep them short and well distinguished from the text. A typical issue with this part is that it just isn't ``feasible'' for the reader due to a high number and depth of the solved details and the attempts to reflect how tough some thing were to deal with, in the text (usually in a way that emphasizes how much effort the author had to put in to finally overcome an obstacle), but the reader just doesn't care. On the other hand, images, photos or even screenshots are welcome. This part can be split in multiple chapters or it can exist as a whole. It is advised to split this part in multiple chapters if the realisation consists of vastly different parts (e.g. server programmed in C++ and a client in HTML etc.). The outline heavily depends on the individual in this case, although the basics can still be identified and should be in an order specified below.
\bigskip

\begin{samepage}
\noindent Typical outline: 
\begin{itemize}
  \item{What is the basic concept of the work,}
  \item{How does it work as a whole (and what is it good for), description of the functionality of individual parts of the solution (there's no need to emphasize everything equally -- some things are more important than others, the ``routine'' can be reduced to a~minimum),}
  \item{How do you use it including a good example, ``case study'' approach, ``screenshots'', procedures (instructions are not desired).}
\end{itemize}
\end{samepage}


\subsection*{Advice on implementation from assistant professor Szöke}
This chapter (there can be more of them) is where you describe your problem from implementation perspective. Which development environment have you chosen, which libraries, class design, communication design, protocol etc. Don't bother with the details. Instead of explaining how a button is implemented, explain how you have implemented artificial intelligence, communication or an interesting function, that's what the reader is interested in. \bf There has to be a clear reason for every single decision you've made. \rm Again, your grandmother does not need to fully understand this chapter, but it's good to have multiple levels. Someone who is not completely lost in IT should understand what and why you have implemented, experienced programmer should understand even the details (how exactly have you implemented it).

\subsection*{Advice on implementation from associate professor Černocký}

If the goal of your thesis is to create a ``production'' SW, describe it here. This part should contain: 
\begin{itemize}
  \item{implementation comments -- for example list of classes and what they represent, you don't have to describe minor things in detail (command line parameters), focus on the key functionality. Don't include full source codes here, those belong to the CD appendix. If a snippet of your code is vital here, you should include it and explain it's importance,}
  \item{if your program is supposed to communicate with the outside world in real time, write about the time sync, conflicts and how you deal with them. It's not expected that you make a 100\,\% ``foolproof'' SW, but you should know about the commonly occuring issues,}
  \item{if you implemented something in Matlab for example, write about it,}
  \item{what were the results -- preferably comparison with existing results from someone else,}
  \item{if the goal was to build on something preexisting and explore new options, here's the place to write about it in detail.}
\end{itemize}

\subsection*{Advice on implementation from assistant professor Beran}

This chapter should be separated from testing, especially if the nature of the thesis is implementation heavy. The recommended length is about a third of the thesis.

Write from the perspective of a poorly paid programmer, that got specification of a~prototype (your draft) and they're supposed to implement it. The chapter should contain:
\begin{itemize}
  \item{target platform and technology specifications,}
  \item{tools used for implementation of the solution prototype.}
\end{itemize}


\section{Testing}
\label{testovani}

This text should provide a clear picture of how the functionality of the software was verified. Through mathematical or experimental means, or a study conducted on users etc. What were the results of verification? The length should be roughly 10 pages.

This part can have a single or multiple chapters. It's advised to split this part into multiple chapters if an extensive testing and evaluation was conducted. The outline depends heavily on the individual, but the basics can still be identified \cite{Zemcik}: 
\begin{itemize}
  \item{Methods and results of verification, that can contain mathematical proofs, testing procedures, testing procedudes involving humans,}
  \item{interpretation of results and possible application in practice (TODO things included).}
\end{itemize}

\subsection*{Advice on testing from associate professor Černocký}

This chapter chapter can be wildly different -- only use things that are useful \cite{Cernocky}.
\begin{itemize}
  \item{Offline data testing -- same or better results as the published ones? If not, why? Worse results don't necessarily mean that the thesis is wrong (access to less data, worse algorithm developed over the course of one semester, etc.), but you should know why.}
  \item{Offline data testing -- are results from implementation in C/C++ and from the original Matlab algorithm the same?}
  \item{How demanding is it when it comes to HW? -- CPU, memory, parallel behaviour, behaviour when using GP-GPU, etc.}
\end{itemize}

\subsection*{Advice on experimentation from assistant professor Beran}
This chapter contains experiments and evaluation -- if the nature of thesis is not implementation heavy, it can be included in the implementation chapter.

Write from the perspective of a poorly paid tester, that got prototype specification (your draft) and it's implementation and they have to test it. This chapter should contain:
\begin{itemize}
  \item{specifications on the platform used,}
  \item{experimental data used for the prototype -- description of data, source, conditions under which the data were obtained, what are the data like,}
  \item{data annotation -- annotation format, source and usage,}
  \item{description of measurements, description of experiments/tests and their conditions,}
  \item{measured data,}
  \item{results -- discussion and interpretation of measured data.}
\end{itemize}

\subsection*{User testing}

Again, only relevant at times, but in some cases it is vital \cite{Cernocky}.
\begin{itemize}
  \item{Testing subject selection process (naive, experienced).}
  \item{``Testing protocol'' -- what did they actually test, what did you ask them -- questions, evaluation, \ldots}
  \item{Testing results -- answers subject by subject and summary.}
  \item{Conclusions -- Were the users satisfied? What were/weren't they satisfied with? Is it good? Is it bad? Can it be improved? Did you make any improvements during your thesis or is it a matter of future?}
\end{itemize}


\subsection*{Advice on experimentation and testing from assistant professor Szöke}
In this chapter, your results should be subjected to extensive testing. Not just you, let independent users test your solution. The most important thing is to collect all the feedback and formulate a relevant conclusion. For example improve GUI, make some sections of code faster or choose a completely different approach. If you have enough time, you can even try an addition iteration and work on the biggest shortcomings of your work.


\section{Conclusion}
\label{zaverPrace}

The final chapter -- Conclusion contains evaluation of achieved results with extra emphasis on student's contribution. A mandatory part of conclusion is also evaluation from the perspective of further advancement of the project. Student states ideas and suggestions based on experience with the project and also states how it relates to other finished projects (other bachelor's theses that year or projects at external workplaces).

The conclusion of a thesis should contain facts summarizing the thesis and give the reader (even without reading other parts of the thesis) information about what the subject of this thesis was and it's success. The conclusion should contain your personal opinions and impressions, and even better a summary of options going forward. Conclusion should be at most one page of text long. Don't include references to thesis text or literature in the conclusion.

Make sure there are no new breakthroughs, new numbers or a new chart in the conclusion.

\subsection*{Advice on conclusion from professor Zemčík}

I highly recommend the following outline:
\begin{itemize}
  \item{Brief summary of the conclusion (for example ``The goal of this thesis was \ldots''.),}
  \item{Stating that the goal was achieved (preferably without self-criticism, save it for the reviewer),}
  \item{Summary of all satisfied requirements of the formal specification of your thesis, either directly as a ``reaction to the specification'' or ``hidden reaction''; either way, state one sentence summarizing the answer to the question ``How did you meet the requirement X of specification?'',}
  \item{Healthy balance of qualitative and quantitative summary of the thesis, e.g 3--5 most notable information (numbers, data, etc.) about the thesis (recommended length is 5--10 lines),}
  \item{your observations (``I learned \ldots''),}
  \item{What are your future plans, preferably split in parts ``I would like to continue my work and \ldots'' and ``My work could be expanded on \ldots'' -- different parts based on what you would like to try and what could be done, but you won't be the one to do it.}
\end{itemize}

Please, keep in mind that that conclusion is a part of thesis, that people will read the most. If anyone reads your thesis in the future, they'll only read introduction, conclusion, or introduction and then conclusion, introduction, description of the work and conclusion, but it is very rare that people read the whole thesis. Each of the possibilities mentioned should be ``feasible'' for the reader.


\subsection*{Advice on conclusion from professor Herout}
The functionality of a conclusion:

\begin{itemize}
  \item{Author looks back at their work: ``The main accomplishments are. The important results are. I managed to.''}
  \item{Author states ideas they did not have time to realise as a way to continue their work: ``There are things that can still be done. If I knew then what I know now, I would.''}
  \item{Author summarizes steps taken to satisfying the requirements of specification.}
\end{itemize}

\begin{samepage}
\noindent Two things.
\begin{itemize}
  \item{First: ``Discussing ways to continue your work'' sounds easy and safe. I would not take this brief statement lightly. Things like ``It would be a good idea to increase the speed and precision'' show, that the author does not really bother thinking about their work and that they don't really have any ideas. Other ``ways to continue your work'' can leave the reviewer thinking that you should have done it to meet the requirements of the specification, which means worse evaluation results.}
  \item{Second: Conclusion is the very last thing the reviewer reads just a few moments before they start writing their evaluation. Therefore the conclusion should put them in the right mood to write the best evaluation possible. It's a good idea to leave impression such as ``I'm a good student, that met any and all requirements of the specification and did a good job.'' There's a thin line between that and ``I'm a bad student, that can't think and program, so I have to scream that I'm the best out loud.''}
\end{itemize}
\end{samepage}

\subsection*{Advice on conclusion from associate professor Černocký}

The conclusion contains:
\begin{itemize}
  \item{Summary of work -- I did this and this, this didn't work, this worked, i didn't have time to finish this because \ldots}
  \item{Future work
    \begin{itemize}
      \item{\uv{short term} -- things that you are certain you could do alone or with the help of 1--2 people within a couple weeks -- months and that it would help.}
      \item{\uv{long term} -- this is entirely a thing of your imagination.}
    \end{itemize}}
\end{itemize}

\subsection*{Advice on conclusion from assistant professor Beran}
Conclusion should only be one page long and contain:
    \begin{itemize}
      \item{what was the goal of the thesis,}
      \item{what steps did you take to get to your solution (see ``brief content of the thesis'' in introduction),}
      \item{what did you manage to create,}
      \item{evaluation of the solution based on results,}
      \item{other possible solutions, future possiblities,}
      \item{I do not recommend selfevaluation and things like ``I chose a specification, learned to program and achieved the goals of thesis''.}
    \end{itemize}

\section{Appendices}

This section should contain descriptive parts of the thesis (e.g. user manual, snippets of source code, detailed schemes, descriptions of designed solutions etc). All appendices need to be numbered. If there are more appendices, there needs to be a list of them at the end of the thesis. \cite{fitWeb}

There is no limit to number of appendices pages. However, it is necessary to keep it purposeful, concise and consider the signifficance of an appendix for the review of the thesis and for follow-up theses in the future. Needlesly large volume of appendices (i.e. not well justified) can negatively affect the assessment from at least environmental and economic perspective. \cite{fitWeb}

Appendices are the right place to include instructions, detailed descriptions of designed protocols and formats, tables, most images and other elements that would disrupt the ``flow of the text'' of thesis. The length of appendices does not count towards the length of the entire thesis. The number of pages should not be too high -- it is important that all the appendices in paper form serve their purpose.


\subsection*{Advice on appendices from associate professor Černocký}

\paragraph{Appendix one -- The cookbook}

Instructions for those who would like to redo your work, less than 10 pages
\begin{itemize}
  \item{What needs to be downloaded, compiled, how to ``hack the OS'' to make everything run smoothly \ldots}
  \item{directories, scripts, launch order, where to look for the results,}
  \item{etc.}
\end{itemize}

\paragraph{Other appendices}

Anything that would disrupt the flow of the text of thesis -- a perfect example is a two pages long derivation of something, three pages long table of specifications, etc. \ldots

The included data medium should contain:
\begin{itemize}
  \item{All the source codes -- Matlab, C, LaTeX, etc.,}
  \item{All the parameters of a model -- HMM, neural network, transformation, basically everything,}
  \item{Everything necessary to be able to launch your software -- external libraries, modules, etc.,}
  \item{All the data -- unless they're under some sort of license,}
  \item{Detailed results -- table summarizing the results belongs in a thesis, but you can include multiple MB of automatically generated tables in the appendices}
\end{itemize}

\subsection*{Advice on appendices from assistant professor Beran}

Assistant professor Beran recommends you include:
\begin{itemize}
  \item{all used libraries, source codes and build instructions,}
  \item{Your application (including binaries), ie. solution executable directly from the CD,}
  \item{Video -- a decent presentation of your software, you can even include a clip of a~working final version of your software,}
\end{itemize}

\subsection*{Poster}
\begin{itemize}
  \item{Burn it to CD and print it too (pdf is probably your best bet),}
  \item{A2 papersize,}
  \item{print it in the library or have it printed by a commercial printing company,}
  \item{A healthy balance of the contents:
  \begin{itemize}
    \item{preferably eye-catching and clear format (what is it, what can it do, what is it good for, why is it good),}
    \item{a bit of technical description (procedures and methods used).}
  \end{itemize}}
  \item{A poster should have:}
  \begin{itemize}
  	\item{student's first and last name}
    \item{student's email}
    \item{supervisor's first and last name}
    \item{academic year}
  \end{itemize}
\end{itemize}

\chapter{Rules for bibliographic citations}
\label{citace}

These rules were taken from the faculty web pages \cite{citace}.

\section{Definitions}

\begin{itemize}
  \item{\bf Bibliographic citation \rm is a summary of details on the work cited, or its part, allowing for its identification and search.}
  \item{\bf Reference to a bibliographic citation \rm is an in-text reference to a bibliographic citation in a different part of the work.}
\end{itemize}

\section{Using works of others}

When using works of other authors, it is necessary to properly differentiate them from your own text. Otherwise you present the text as your own, which is an unacceptable violation, especially when it comes to theses and dissertations. There are two ways to incorporate works of other authors into one's own:

\begin{itemize}
  \item{\bf Quotation \rm -- the text used matches the source document word for word. Short quotations are enclosed within double quotation marks, long ones in an indented free-standing paragraph. 
The quotes are also often typed in italics.}
  \item{\bf Paraphrasing the original text \rm -- the author of the presented work puts the original text into his/her own words, while maintaining the meaning of the original statement. Paraphrases are typed in the same typeface as the rest of the text. In order to distinguish which part of the text is borrowed and which is your own, you must use other means (for example, by a reference to a citation at the end of the sentence or paragraph the author indicates that the sentence or paragraph was cited or paraphrased; when the paraphrase is more extensive, you can specify that at the beginning of a~subchapter, for example: ``This subchapter has been adopted from [1]''.)} 
\end{itemize}

Quotations are suitable for definitions, parts of laws, regulations or rules, or to discuss the opinions of other authors. Quotations are therefore not that frequent in technical texts. If there are no valid reasons for a quotation, use a paraphrase. Also, think about the scope of the quotation. Too many and too long quotations indicate that the thesis is of poor quality (usually finished in the last minute, and the quotations only serve to reach the minimum length required).

\section{Basic principles of citation}

Bibliographic citations are provided to make it clear to the reader what works the author used as a basis for his/her own, to familiarise the reader with a wider context (for example, 
so that a reader less knowledgeable than the author's target audience could find more on what the author did not explain in his/her work) and also to comply with the terms of the 
Copyright Act (see Section 31). When citing, the following rules must be followed:

\begin{itemize}
  \item{\bf Cite all the sources you have used! \rm If you fail to cite any of the sources, you are presenting someone else's work as your own! Make notes of all the sources you have found since the very beginning of your work on the project. It is more difficult to look up all the sources once you are done.}
  \item{\bf Cite only the works that you actually used! \rm Readers who actually know the works cited (or look them up) will soon realise that even though you use them as reference, you do not really know them.}
  \item{\bf Only cite primary sources! \rm Only cite the works that you actually worked with \texttt{physically} (or on a screen). Otherwise you risk using a quotation with mistakes.}
  \item{\bf Cite precisely! \rm You will make it easier for the readers to find the original source avoid the suspicion that you intend to make the identification of the source difficult or impossible by providing an incomplete citation, so that the readers cannot check the scope of the quoted text.} 
\end{itemize}

When writing your text, pay special attention to these citation principles, as their violation means, for apparent reasons, far  more serious consequences for the overall assessment than non-compliance with the formal requirements for bibliographic citations.

\section{Citation standards}

The standards for creating bibliographic citations and their reference in professional publications are governed by ČSN ISO 690:  Information and documents -- Rules of bibliographical references and citations of 
information sources of 2011. The standards are available at the FIT Library. There is a citation generating tool based on these standards, available at \url{http://www.citace.com/} (in Czech, but simple to use -- write ISBN, DOI or name and search it). and you can find a brief summary of the standards with citation examples 
on the same website \cite{biblio}.

Although the standards allow to place citations in professional publications in different ways (at the end of the text, at the end of the individual chapters, directly in the text, in the footnote, partly in the text and partly in the footnote), it is required for theses and dissertations at FIT BUT that the list of works cited is provided at the end of the work.

The reference to the citation is in the form of a serial number, under which the citation is listed in the Works Cited section at the end of the work. The serial number is stated in the text in square brackets.

\noindent Example: \textit{The SMTP protocol is specified in RFC 5321 [1].}

Citations are in alphabetical order. Ordering of names containing letters with diacritics can be changed using the \texttt{key} element, the value of which can be set to a last name without diacritics. If a record does not have the author element, the citation is ordered to the beginning of the list, which is not ideal. In this case, we can change the ordering by setting the key element to a desirable value.

\textbf{Example:}
\begin{verbatim}
    @Article{Cech:2020:Citace,
    author               = "Čech, Jan",
    key                  = "Cech",
    ...
\end{verbatim}

If you quote the same publication several times in a row, you can use the 
term \uv{Ibid.} instead of repeating the entire reference, supplemented 
by a page number.

Examples of citations are in appendix \ref{priloha-priklady-citaci}.

\section{Using electronic sources}

Finding high-quality sources to use as a basis for your work is essential. The quality of a~classic printed work may be recognised quite easily. When working with electronic sources, examine their credibility and quality thoroughly. Always verify who the author of the material is, and evaluate the depth of the information provided (a number of electronic sources presents merely superficial and incomplete information, and therefore the authors copy eachother's mistakes) critically.
Even if you find high-quality electronic sources, do not rely solely on those, and try to find a printed work as well.

Whenever you cite electronic sources, always state when the information was found and used as the entire website may not be available several days later. When using BibTex, this can be achieved by adding the following key: (\verb|cited = "yyyy-mm-dd"|). You can use the \verb|@website| (entire domain) or \verb|@webpage| (a single page within the domain) record types to cite electronic sources. If you cite a magazine article or a paper from a conference, do not cite them as electronic sources and use \verb|@article| instead.

\section{Evergreen: citing web vs. paper}

Yes, everything is on the web these days. Thank god.

Web however does not have an archive. It changes every minute. If the references to literature are all URLs, this part of is somewhat alive, ungraspable, fluid -- links refer to an environment, that changes every second. This part of thesis should ideally be rigid, constant -- only references to paper version of literature.

It's easy to cite a web source, when it comes to a downloaded .pdf file of a magazine or conference article. Don't do it in your thesis.
\bigskip

\noindent Don't include this in the list of referrenced literature

\noindent \it BAY, H., ESS, A., TUYTELAARS, T. and GOOL, L. V.: Speeded-Up Robust Features (SURF) [online]. 2006, updated 2008 [cit. 2010-07-13]. Retrieved from: \url{https://www.vision.ee.ethz.ch/en/publications/papers/articles/eth_biwi_00517.pdf}
\bigskip
\rm

\noindent when you can use this instead

\noindent \it BAY, H., ESS, A., TUYTELAARS, T., GOOL, L. V.: \uv{SURF: Speeded Up Robust Features}, Computer Vision and Image Understanding (CVIU), Vol. 110, No. 3, pp. 346–359, 2008.
\bigskip
\rm

There's a chance that the web no longer works in a month, whereas a~conference proceedings can be found retroactively until at least the end of this civilization.


\section{Typesetting of citations}

The template for BUT FIT theses and dissertations uses the BibTeX system to typeset citations.

The following text is taken from ČSN ISO 690 standard and condensed. Individual parts of a citation are separatedy by the dot symbol ``.''.
Each item in the cited document should be stated as it appears in the original cited document. The usual order of items in a bibliographic citation is as follows:

\begin{enumerate}
    \item \textbf{name(s) of the author(s)/creator, if they are available(s)}
    \begin{itemize}
        \item Persons or institutions responsible for creation of the contents should be stated as authors.
        \item First names and other parts of names should follow after the last name, e.g. \texttt{GORDON, D}.
        \item If a cited source has multiple authors, list them (using the ``and'' separator), and if there are more than three authors, we do not list all of them and use ``et al.'' (names followed by ``and others'') instead.
    \end{itemize}
    \item \textbf{title}
    \begin{itemize}
        \item Title is typed in \textit{italics}. A title too long can be cut short and missing words replaced with \ldots{} (three dots).
    \end{itemize}
    \item \textbf{type of medium}
    \begin{itemize}
        \item A type of medium is a medium, where the cited material is available (e.g. [DVD]) as well as the form in which it is available (e.g. [Braille]). This information is stated in square brackets. [online] is used for online materials.
    \end{itemize}
    \item \textbf{edition}
    \begin{itemize}
        \item Edition should be stated with symbols, e.g. ``2nd ed., (re-edition)''. If it is an updated version, it should be apparent from the citation, which version was used.
    \end{itemize}
    \item \textbf{place of publication, publisher, date of publication}
    \begin{itemize}
        \item If the place of publication is not stated in the cited document, but is known, we state it in square brackets.
        \item A publisher should be a person or organisation that is the most emphasized in the cited document.
        \item Date of publication should always be stated. A year is sufficient for paper publications, however it is necessary to also state a month and a day for electronic sources.
    \end{itemize}
    \item \textbf{numbering within a unit, page or page range}
    \begin{itemize}
        \item Citation should idenfity the specific part of a cited document, that is used for citation.
    \end{itemize}
    \item \textbf{title and volume in series, if they are available}
    \item \textbf{standard identifiers}
    \begin{itemize}
        \item If a cited document has an international standard number (ISBN, ISSN, \ldots) or DOI, that serves as the unique identifier, it must be stated.
    \end{itemize}
    \item \textbf{dostupnost, přístup nebo umístění informací}
    \begin{itemize}
        \item The location of source should always be stated for electronic sources. Stated as: ``Available at:'' followed by URI or URL.
    \end{itemize}
    \item \textbf{additional general information}
\end{enumerate}

\chapter{Formal aspect of a thesis}
\label{formality}

The formal requirements for writing a bachelor's or diploma thesis are based on rector's directive no. 72/2017 \cite{smernice} and FIT guideline no. 7/2018 \cite{smerniceFIT}. Formal aspect of a thesis is an important part of reveiwer's  assessment and should be given some attention (it is necessary to familiarize yourself with the mentioned directives). Other instructions and recommendations are listed on faculty web pages \cite{formalniBP}, \cite{formalniDP}.

Required length of the thesis text without appendices is shown in table \ref{rozsah}: 

\begin{table}[hbt]
\centering
\caption{Required thesis text length in standard pages}
\label{rozsah}
\begin{tabular}{|l|c|l|l|}
\hline
 & Minimal length & Usual length & Length should \\
 &  &  & not exceed  \\ \hline
Bachelor's thesis (9 cr.) & 30 & 40--50 & 60 \\ \hline
Bachelor's thesis (13 cr.) & 40 & 60--80 & 100 \\ \hline
Semester project (SEP) & 20 & 30--40 & 50 \\ \hline
Dissertation & 50 & 80--100 & 120 \\ \hline
\end{tabular}
\end{table}

The length of typeset pages will be roughly 1/2 of the extend in standard pages. The term {\it standard page} applies to evaluating the volume of a thesis, not the number of printed sheets of paper. From a historical standpoint it's approximately the number of pages of handwriting, that used to be typed on a typewriter to pre-printed forms at the usual line length of 60 symbols and 30 lines per page. Considering the need for proofreading marks, the line spacing value used to be 2 (every second line). This data (number of symbols per line, number of lines and the spacing between them) are in no way related to the final printed product. Their only use is to assess the length. \textbf{Therefore, one standard page means $\mathbf{60\cdot 30 = 1800}$ symbols including whitespaces.} Images inserted in text are counted towards the length of the thesis by estimating the amount of text it replaces in the final document.

For a rough estimate of the number of standard pages when using the \LaTeX{} system, you can use the sum of source file sizes of the thesis and divide it by about 2000 (usually we'd divide by 1800, but source files contain other symbols and commands that do not count towards the length of the thesis). For a more accurate estimate, you can extract the plain text from a PDF (e.g. use cut-and-paste or {\it Save as Text...}) and divide it's size by 1800. You can also use the Detex\footnote{\url{https://www.ctan.org/pkg/detex}} application (on Linux available in it's distribution, for Windows, however, you need to install it separately\footnote{\url{http://urchin.earth.li/~tomford/detex/}}), that removes special symbols and commands from source text and then you can divide it's size by 1800. \textbf{The Detex application uses the \texttt{Makefile} in this template to count the number of standard pages in the core of the thesis too} -- command \verb|make normostrany|.

In Microsoft Word the approximate length of the thesis in standard pages can be computed using the {\it Word Count} function in {\it Tools} menu if you divide {\it Symbols (spaces included)} by 1800. Only the core of the thesis is counted towards it's length. Parts such as abstract, keywords, declaration, table of contents, references or appendices do not count towards the length of the thesis. It is necessary to select the core of the thesis first and then have the software count the number of symbols. Assess the length of images manually. The same procedure can be applied when using OpenOffice or LibreOffice. 

Original text dealing with the assignment, where the solution is core of the thesis, must be at least a third of the entire thesis text. A mere compilation of available sources is unacceptable.

The subject of reviewer's assessment is primarily the text and final product.
Needlesly large number of pages is the evidence of poorly processed topic and a burden for the reviewer. Theses, where the length of text report is equal to amount of work done can be longer and enriched with explanatory text. The explanation of the essence of solved problem and the procedures used to solve it does not have to increase at a linear rate with the amount of work. Well structured and comprehensive text report can be of relatively small length. Detailed descriptions of significant parts of project, that are more of a documentary (rather than explanation), can be included in appendices and referenced from the main text.

If the length of a text report is close to the minimal required length, reviewer will focus heavily on whether or not individual parts of the report for understanding the thesis are really necessary. The inclusion of foreign texts, that only vaguealy relate to the topic of the thesis or ones that are questionable, in attempts to reach at least the minimal required length (for example not enough time just before the submission deadline), can lead to a~much worse overall review of the thesis.

When you insert an image, choose the dimensions carefully so that they do not overlap the are for text printing area (i.e. text margins on all sides). Put large images on a~separate page. Images or tables that are larger than A4 dimensions can be folded (so-called Engineering fold, similar to Z-fold but doesn't make issues with binding) and included in the main text (if it is crucial for the thesis), appendices or fold it and put to the pocket on the back side of the binding.

Images and tables use their own, independent numerical series. This means that references in the text must state if it is an image or a table as well as the number (for example ``... {\it see table 2.7} ...''). It's rather natural to comply with this principle.

When it comes to references to pages, chapter and subchapter numbers, image and table numbers and many other examples, we use special tools provided by the desktop publishing program, that can generate the correct numbers even if the text moves due to changes made to the text itself or changes to typesetting parameters.

Equations referenced in the text will be provided with serial numbers on the right side of their respective line. These ordered numbers are surrounded by round brackets. Equations numbering can be in order within the whole text or within individual chapters.

If you are in doubt when printing a mathematical text, try to follow the LaTeX-defined printing system. If your thesis contains a large number of mathematical formulas, we highly encourage you to use LaTeX system.

There is no space between a number and letters that form a single word or a symbol -- for example {\it 25x}. In a paragraph, it is better to spell both out though -- for example {\it ten times}.

Punctuation symbols such as a dot, comma, semicolon, colon, question mark and exclamation mark, and even closing brackets and quotation marks are adjacent to the preceeding word, without a space. The space comes after the punctuation symbol itself. This, however, does not apply to decimal point. Opening brackets and quotation marks are adjacent to the following word and the space in front of them is excluded -- (like this) and ``this''.

We do not use the same symbol for connecting and separating dash, and a regular dash. There is a special (longer) character for a dash. In \TeX (\LaTeX ) system, the connecting dash is a single ``dash'' symbol (e.g. ``Brno-center''), when typing text for intervals or pairs, opponents and others we use two ``dash'' symbols (e.g. ``price 23--25 crowns'', ``match England -- Belgium''), to separate a section of a sentence, to separated an inserted sentence expressing unspoken thoughts and in other situation (see orthography) we use the longest type of dash represented by three ``dash'' symbols (e.g. ``Another term --- no matter how pointless it seems --- will be defined informally in the following paragraph.''). When typesetting a~mathematical minus symbol, we need to use a different symbol again. In TeX system, the source text, the regular minus symbol is used (i.e. ``dash'' symbol). Typesetting in math mode, however, is done by surrounding the formula with dollar symbols to generate the correct output.

Rules for the use of abbreviations in different languages are listed in their respective orthography books (e.g. Pravidla českého pravopisu \cite{Pravidla}). There are other reasons to always have one of these close.

\section{Common errors}
\label{chyby}

This chapter contains a selection of the most common errors as well as advice on how to avoid them, taken from professor Herout's blog \cite{Herout} and from the list of common errors that assistant professor Szöke posted on their blog \cite{chyby}.

\subsection*{Minor things that notoriously ruin reading}

\begin{itemize}
	\item{
		\textbf{Using hyphens instead of dashes} \\
		A dash is long and there should be whitespace surrounding it. Dash is very often used instead of a dot in sentences: ``This book -- published before the war -- is amazing.'' It is also used for ranges: 	``page 23--26'' or ``success rate of 3--5\,\%.'' More examples can be found in the language handbook \cite{prirucka}.

Hyphens occur in our IT theses very rarely (at least they should). A good example would be phrases or joining subjects ``Rh-factor'', ``real-time'', ``propane-butane''.
	}
    \item{
    	\textbf{Brackets surrounded by whitespaces} \\
        There always is a whitespace in front of the opening parenthesis or brace (when referencing literature). There is no space after the closing parenthesis or brace, if they're immediately followed by a dot, comma, exclamation mark or question mark. There are no whitespaces inside the brackets.
    }
\end{itemize}

\noindent Here's a brief overview of common stylistic and language sins.

\begin{itemize}
	\item{
    	Correct in english:
		\begin{itemize}
  			\item{\uv{by using the OpenGL library}}
  			\item{\uv{in the MVC model}}
  			\item{\uv{all UI elements}}
  			\item{\uv{from the JSON string}}
  			\item{\uv{call it from C\# code}}
		\end{itemize}
        
        Incorrect in czech:
        \begin{itemize}
          \item{\uv{s použitím OpenGL knihovny}}
          \item{\uv{v MVC modelu}}
          \item{\uv{všechny UI prvky}}
          \item{\uv{z JSON řetězce}}
          \item{\uv{volat ji z C\# kódu}}
        \end{itemize}
        
        Correct in czech:
        \begin{itemize}
          \item{\uv{s použitím knihovny OpenGL}}
          \item{\uv{v modelu MVC}}
          \item{\uv{všechny prvky UI} -- nebo ještě radši \uv{všechny prvky uživatelského rozhraní}}
          \item{\uv{z řetězce ve formátu JSON}}
          \item{\uv{volat ji z kódu v jazyce C\#}}
        \end{itemize}
    }
    \item{
    	\textbf{Sentences without verbs} \\
        Every single sentence needs a verb. Sentences without verbs are sometimes used for artistic purposes in poetry. Over the course of the last two weeks, I have read countless sentences without verbs in them and it never quite worked, always lead to a negative outcome. Make sure every sentence of your thesis has a verb in it.
    }
\end{itemize}

\subsection*{Figures (almost) without captions}

Figure (similar with a table) consists of the image itself and it's caption (\texttt{\textbackslash caption} in LaTeX). The caption of a figure or a table is there to make sure the final object works as a~standalone object -- reader checks the figure often even before reading the surrounding text and therefore it is expected that they can understand the figure even without reading the text.

I wouldn't worry about five to seven lines long figure captions. Two lines will do just fine at times. Sometimes -- not very often -- captions consisting of just three words are the best ones. If all the captions in a thesis consist of only three to four words, the reader will most likely be frustrated, because the figures make no sense whatsoever.

If a figure contains a colorful code (some objects are red, some are blue and others thick green), proper explanation of code must be a part of the caption. If a figure consits of parts (for example top right, top left and bottom), naming and explanation for each part belongs in the caption.

Whenever I tell my students this, they get anxious about the whole thing: ``But that means I have to move entire sentences from text to the captions!'' Yes, move them, there's nothing wrong with that. The basic explanation will be a part of the figure and that's how it should be. More detailed explanation, reasoning and interpretation should remain in the text itself. Twenty lines long caption for a figure or table is too long, but five lines long caption is a standard.


\subsection*{Grammatical person}

Using \bf second-person \rm (addressing the reader with ``You/you'') is almost always wrong and it gets annoying very quickly.

Incorrect:
\begin{itemize}
  \item{``If you look at figure 5, you'll see ...''}
  \item{``If you work with library X, I'm sure you'll stumble upon ...''}
  \item{``If you want to switch to settings, select the respective option in menu.''}
\end{itemize}

Correct:
\begin{itemize}
  \item{``Figure 5 shows ...''}
  \item{``A commonly occuring thing when using library X is ...''}
  \item{``Settings can be accessed by selecting the respective option in menu.''}
\end{itemize}

Using \bf first-person plural \rm (``we'') is not always wrong either and ``classic'' literature about writing technical text sometimes even recommends it as so called ``author's plural'':
\begin{itemize}
  \item{``We found out ...''},
  \item{``We focused on ...''},
  \item{``We designed a solution, that ...''}.
\end{itemize}

Less experienced writers often switch from author's plural to some weird and incorrect language mode, that has a common occurence in kindergarten. This is how teachers in kindergarten speak (nothing against them otherwise): ``Alright kids, let's glue a bead right in the middle of the flower. Now we push out a bit of glue, yeees, and we push the bead into it with a finger, juuust like that.''

It's not funny and cute, if diploma student uses this language to describe their life's work: ``First we need to link library X. Then we create objects of selected classes and send them to a server one by one. When the server responds with an error code, we must reset the connection.'' Do not use this language mode if you wish for your thesis to be perceived as a work of someone who is not on the mental level of a child in kindergarten.

Using \bf first-person singular \rm (``I``) is correct, when it comes to a matter of subjectivity:
\begin{itemize}
  \item{``I focused on ...''},
  \item{``I created ...''},
  \item{``I measured ...''},
  \item{``I addressed several respondents ...''}
\end{itemize}

It is incorrect (and unfortunately quite common) to use first-person in the description of procedures and phenomena:
\begin{itemize}
  \item{``In the first step of the algorithm, I reset counters.''}
  \item{``If the pointer points to null, I allocate new object.''}
  \item{``It's clear from the graph that the cache size I used is too small.''}
\end{itemize}

\subsection*{Other errors}

\bf Don't cite stupid things: \rm Information in introduction about the fact that internet is available even on Mount Everest can seem like a thing, that needs to be confirmed. Nevertheless, this pointlessly inflates Literature with a lot of irrelevant sources (not within the scope of your thesis) -- that can be a bad thing for you. Just don't write anything like that in your thesis, or don't cite it at least.

\bf Abbreviations (and notes) in footnotes should make sense on their own. If your text contains an abbreviation, specify what it stands for when you first use it: \rm If an abbreviation occurs two chapters later, you need to specify what it stands for again when you first use it there, alternatively write a footnote addressing the abbreviation. It is important that the footnote contains the abbreviation too. If it's an abbreviation of a~phrase, feel free to translate it to the desired language.

\noindent Incorrect:
\begin{enumerate}
  \item{large-vocabulary continuous speech recognition}
\end{enumerate}
Correct:
\begin{enumerate}
  \item{LVCSR -- large-vocabulary continuous speech recognition}
\end{enumerate}

\bf Don't leave text between a section and subsection out: \rm There should be a text between the titles of chapters and subchapter. This calls for 1 -- 2 paragraphs, where you explain what the chapter is about and what can the reader learn here.

\bf Hand drawn sketches and images: \rm It's easier to quickly draw something, take a~picture of it and continue writing, especially when it comes to a work draft. Usually the initial sketch of a scheme is wrong and re-doing it takes precious time. I'd be careful with ``hand work'' in the final text. Some reviewers could see it as a false indication, that you did not have enough time, and so you decided to save time by doing some quick hand drawing.

\bf Introduction and conclustion: \rm Typos, nonsensical sentences, errors, ungrammatical words, \ldots Neither one of these belong in a thesis. An evasive typo can hopefully get lost deep in a thesis. Everyone reads the introduction and conclusion though. Making errors here is shameful. When you try to convince the committee members, that you did a big chunk of work, how is the fact that you can't read a single page of text going to affect the thesis review?

\bf Cite figures: \rm If you ``borrow'' an image from a source, don't cite it and it is later discovered, you receive an F. Citation referring to literature at the end of the figure description is (probably) OK. Nonetheless, many reviewers prefer stating the source explicitly in brackets: (taken from literature [xy]).

\bf Don't overdo it with chapters: \rm If a chapter has 1 -- 2 paragraphs in individual subchapters, try to think of a better way of doing it, like bullet points.

\bf Notes in images: \rm Don't describe complicated images with ``A is on the left side, B is at the top, C is in the middle, D is under that and next to it is XY.'' This results in a half of a page worth of text. And if the figure is not on the same side, the reader has to flip the pages til they die. Add captions to figures and it becomes a standalone figure.

\bf Details belong to appendices: \rm Don't try to describe a complete class diagram, table diagram, object diagram, ... at any cost. Include one complete scheme and describe the basic components. Then you can focus on the core of the system (for example the three highlighted objects). Those are the key and they're the basis of your thesis. Description of other ``supporting'' objects (e.g. data loading and rendering) can be a part of an appendix. You can simply reference the appendix by writing: \it Detailed description of all classes and their methods can be found in Appendix 1. \rm

\bf Avoid compound sentences that are too long: \rm If your sentence is the length of a paragraph (6.5 lines), by the time the reader reads third line, they already forgot the beginning of the sentence. Don't be afraid to split the compound sentence into smaller parts. Fire short sentences at the reader rather than suffocating them with loads of words. Alternatively, you can present information using the \tt itemize \rm LaTeX environment.

\bf Unbreakable spaces: \rm A typographical error such as leaving ``s'' at the end of a line as early as the title of your thesis is extremely shameful. There is something called non-breaking space, in other words a space, that prevents breaking the end of a line. In LaTeX, this space is generated using a tilde \textasciitilde. It is described in most \LaTeX{} books.

\chapter{Thesis submission}
\label{odevzdani}
Bachelor's and diploma theses are submitted in both printed and electronic version. The electronic version needs to be on the data medium included in the printed version of thesis and uploaded to FIT IS. It is only considered submitted when both versions of a thesis are submitted properly. This chapter is mostly taken from the official instructions that can be found on the web \cite{formalniBP}, \cite{formalniDP}.

Those that have decided to keep some of the parts of their thesis a secret need to submit an application at least a month prior to the submission of thesis. Based on the changes of university law no.~111/1998 with law no. 137/2017, \S 47b, paragraph 4 -- in case of a delayed publishing, one print of full version of the thesis must be sent to Ministry of Education, Youth and Sport of the Czech Republic -- as of 2017. Student must submit an additional print of the full version in a proper binding. This second print will be identical to the first one, except it will have a copy of specification and it won't have a declaration. Both prints will have a paper stating that it is a print for delayed publication (at most 3 years), reason for the delayed publication, and the thesis will then be published when the agreed upon date of delay has passed. 

Before you submit your thesis, go through the checklist (appendix \ref{checklist}) and make sure everything is okay. Then make sure that the thesis is in compliance with directives \cite{smernice} and \cite{smerniceFIT}. Important details that everyone seems to forget from time to time:
\begin{itemize}
	\item \textbf{Title page:}  Don't forget to set the correct year (of submission) and department according to the specification.
	\item \textbf{Specification:} Don't forget to download electronic version of the specification (template expects a file named \texttt{zadani.pdf}).
    \item \textbf{Declaration:} Don't forget to sign the declaration in both prints of the thesis.
    \item \textbf{Bibliographic citations:} Make sure you cite all used sources in text.
\end{itemize}

Print the thesis and have it bound. As soon as you open the bound thesis on a random page, you'll find an error. Honestly, don't worry about it, that's just how it is. Nothing is perfect, so \uv{just ship it}\footnote{\url{http://blog.igor.szoke.cz/2011/08/zkoumejte-bezte-za-bod-odkud-neni.html}} \cite{rady}.

\section{Submission of printed verions of thesis}
Bachelor's thesis or dissertation is submitted in two prints, both of them contain signed declaration of authorship. The archived print must be bound in an undeconstructable way. A file made out of dark (blue, grey) bond paper is recommended. There will be an envelope with a CD/DVD or other allowed media glued to the inner side of the file back so that it can be easily accessed. The print that will be available in FIT library and serve as a~preview before the presentation can be bound in a deconstructable way (for example comb binding).

Student also submits the following parts of thesis in electronic format:
\begin{itemize}
  \item{text report in PDF format (and in FIT IS),}
  \item{source form of the text report (including everything necessary for editing and printing again),}
  \item{complete documentation (installation instructions, user manual, schemes, etc.),}
  \item{source codes of programs (binary programs must be compilable from submitted source files),}
  \item{all programs in executable form capable of running in FIT Computer Center environment. If this cannot be achieved (machines in Computer center do not have the required software installed, or the required hardware is not available, or if the result of thesis is a new hardware), after discussing the situation with their supervisor, the student will present a functional product to the reviewer.}
\end{itemize}

You can convert document from LaTeX to PDF using the pdflatex application or from dvi using dvipdf application, or even from postscript using ps2pdf application. The final document should be under 10 MB large. If it is larger, something is wrong. Documents like this usually contain needlessly large images.

A complete electronic form of thesis must be included on a non-rewritable media CD-R, DVD-R, DVD+R in ISO9660 format (with RockRidge and/or Jolliet extension) or UDF, or SD (Secure Digital) memory card in FAT32 or exFAT format,and card is set to write-protected mode.


\section{How to submit thesis in FIT IS}
The full thesis text in PDF must be submitted to FIT IS -- in the Final exams and theses section. You also need to fill in the abstract and keywords, both in Czech (Slovak) and English (be sure to replace non-breaking spaces when copying from \LaTeX{}), and give the permission for publication. The only reason to not make the thesis text public is protection of intellectual property, which must be approved by the supervisor. In this case, student submits both written and electronic full version of thesis for review and following the guidelines on faculty web pages or guidelines of the supervisor, potentially even version of thesis without the sensitive information for immediate publication. Thesis is considered submitted only when all the requirements are met.

The semester project for master's thesis is also submitted to the FIT IS. Bachelor's term project (term thesis) is only submitted if the supervisor requires it.

\chapter{Conclusion}
\label{zaver}

This text summarized the formal requirements for a technical report of a bachelor's thesis or a dissertation. It described the usual procedures used when writing a text of technical nature and offered additional information and independent useful hints and tips for creation of a technical report of a dissertation. It was also explained that a bachelor's thesis also a~dissertation and needs to be approached as such.

It is necessary to point out that dissertation is a unique individual work, that is developed under the supervision of an experienced expert. Regardless of what this template says, you're only obliged to comply with the official guidelines stated on the faculty web pages. You always need to consider which things in the text above are relevant for a specific dissertation and which are not. Most importantly, you should listen to your supervisor, who understands the given problem the most and is therefore able to provide the best advice that you can get.

Despite the effort, it is not possible to include all the elements needed for developing a thesis in this template and guarantee that once the text, images, literature and others are added, that everything will be alright for every single dissertation. A longer text than expected will break to two lines, en entry in list of references that the style was not tested with, and in other cases the result can be hardly satisfying. It could require a modification of the template to account for an error that occurs once in hundred projects. The final PDF and consequently the printed version needs to be thoroughly checked, don't let thoughs like \uv{this was generated by the template, therefore it must be correct} cloud your judgement. If you find errors in the template or you have suggestions on how to improve it, contact us via email at \texttt{sablona@fit.vutbr.cz} and help us improve it. Any and all comments and suggestions are welcome.

Your supervisor can help you significantly when it comes to correcting errors. However, do not expect them to read through your work the night before submission deadline. For that reason, it is necessary to have everything ready in advance and consult your supervisor as you write your dissertation. Supervisor's critical viewpoint can allow for a better result and the extra effort will have a positive effect on their evaluation of the work.


Lastly, on behalf of all the authors, I would like to wish everyone currently in development of their own dissertation and those who are getting ready to start developing it a~successful completion and presentation of their work.

%=========================================================================

  \else
    % Tento soubor nahraďte vlastním souborem s obsahem práce.
\chapter{Úvod}

Tento text slouží jako ukázkový obsah šablony a současně rekapituluje nejdůležitější informace z předpisů a poskytuje další užitečné informace, které budete potřebovat pro tvorbu technické zprávy ke svojí práci. Než se šablonou budete dále pracovat, je třeba vědět, jak ji správně použít. To je stručně uvedeno v~příloze \ref{jak}.

I když některým studentům pro napsání dobré diplomové práce (bakalářská práce je také diplomová -- dostává se za ni diplom) stačí znát a dodržovat oficiální formální požadavky uvedené ve směrnicích a typografické zásady, často je výhodné před započetím psaní zjistit, jaké jsou osvědčené postupy pro psaní odborného textu a jak si práci usnadnit. Někteří vedoucí svým studentům připravili popisy osvědčených postupů, které vedly k desítkám úspěšně obhájených prací. Výběr nejzajímavějších postupů, které měli autoři této šablony k~dispozici ve chvíli její tvorby, je v níže uvedených kapitolách. Má-li Váš vedoucí svoji stránku s doporučenými postupy, tyto kapitoly můžete vynechat a řídit se pokyny svého vedoucího. Pokud takovou stránku nemá, může být přečtení níže uvedeného textu vhodnou přípravou na konzultaci o plánované struktuře a náplni textu práce.

Diplomová práce je rozsáhlé dílo a tomu odpovídá i technická zpráva. Ne každý je schopen si sednout a jednoduše ji napsat. Je třeba vědět, kde začít a jak postupovat. Jedním z možných přístupů je začít psaním klíčových slov a abstraktu, abyste si ujasnili, co je v~práci nejdůležitější. O tom pojednává kapitola \ref{abstrakt}.

Po sepsání abstraktu se lze pustit do psaní samotného textu technické zprávy. Typicky si nejprve připravíme základní strukturu práce, kterou pak budeme plnit textem. Kapitola \ref{struktura} se zabývá základními informacemi a radami pro psaní odborného textu, které Vám pomohou vyhnout se začátečnickým chybám, a stanovením nadpisů kapitol a přibližných rozsahů jednotlivých částí práce. V závěru kapitoly je pak uveden přístup, kterým si lze psaní technické zprávy značně usnadnit.

Diplomové práce v oblasti informačních technologií mají určitou typickou strukturu. Po~úvodu bude následovat kapitola či kapitoly zabývající se shrnutím současného stavu, který bude v následujících kapitolách zhodnocen a bude navrženo řešení, které bude implementováno a otestováno. V závěru pak budou výsledky vyhodnoceny a bude navržen budoucí vývoj. I když se názvy a rozsahy kapitol v různých pracích liší, vždy tam lze najít kapitoly odpovídající této struktuře. Kapitola \ref{kapitoly} se zabývá obsahy typických kapitol, které se v diplomových pracích z oblasti IT vyskytují. Většina studentů ve svojí práci pravděpodobně využije pouze určitou podmnožinu popsaných kapitol, která je pro jejich práci relevantní. Uvedené popisy a rady mohou pomoci jak s~rozhodnutím, zda danou kapitolu uvést, tak i~s~vnitřní strukturou a samotným obsahem kapitoly.

Za závěrečnou kapitolou práce vždy následuje seznam použité literatury. Citacemi, které tento seznam tvoří, a odkazy na ně se zabývá kapitola \ref{citace}. Byť to tak nezkušený student nemusí vnímat, je seznam použité literatury a odkazy na něj pro práci zcela zásadní. Hodnocení práce s literaturou a citací tvoří jednu z důležitých částí posudku oponenta a bude-li chybět jediná položka, může to vést k hodnocení stupněm F, následnému disciplinárnímu řízení za plagiátorství a k vyloučení z nedokončeného studia. Nesprávná práce se zdroji může mít i další důsledky -- v roce 2018 stála křesla dva členy české vlády. Proto prosím citacím věnujte odpovídající pozornost.

Po dokončení textu je nutné zjistit, jaké požadavky jsou kladeny na vysokoškolskou kvalifikační práci na FIT VUT v~Brně, a dořešit případné nedostatky. Formální požadavky jsou uvedeny ve směrnicích a na webových stránkách, které jsou zmíněny v kapitole \ref{formality}. Tato kapitola obsahuje i požadované rozsahy jednotlivých typů prací a další vybrané informace z~předpisů a doporučení. V závěru kapitoly je uveden přehled nejčastějších chyb, se kterými se oponenti setkávají a kterým byste se měli vyhnout. Hodnocení formální úpravy práce je pak další z důležitých součástí posudku oponenta.

Po odstranění formálních nedostatků lze práci odevzdat. Před odevzdáním práce si můžete projít kontrolní seznam (tzv. \uv{checklist}) uvedený v příloze \ref{checklist}. Samotné odevzdání listinné i elektronické verze práce je pak popsáno v kapitole \ref{odevzdani}.

V závěrečné kapitole \ref{zaver} je pak uvedeno shrnutí toho, co se lze přečtením tohoto textu dozvědět, a to nejdůležitější, na co je třeba myslet před odevzdáním práce.


\chapter{Abstrakt}
\label{abstrakt}
Pod nadpisem Abstrakt je uvedeno shrnutí práce zabírající prostor maximálně 10 řádků. Z~dobrého abstraktu by mělo být i přes jeho malý rozsah patrné, jaký problém se řešil, jaký přístup k jeho řešení byl v práci použit a jakých výsledků bylo dosaženo. Účelem abstraktu je, aby potenciální čtenář práce již po přečtení abstraktu věděl, zda v práci najde to, co hledá \cite{fitWeb}. Zbytek této kapitoly byl převzat z blogu prof. Herouta \cite{Herout}.
\bigskip

\noindent Za prvé – na abstraktu záleží. Za druhé – není těžké ho napsat. Pojďme na to.

\subsection*{K čemu je abstrakt}
Abstrakt slouží k \bf vyhledávání\rm, společně s názvem dané vědecké práce a seznamem klíčových slov. Tyto části (snad s výjimkou názvu) nejsou přímo součástí textu a nečeká se, že někdo, kdo by zasedl ke čtení dané vědecké práce, bude číst je. To, že práci čte, znamená, že už se dostal za fázi čtení abstraktu. Abstrakt mu slouží ve chvíli, kdy se ještě rozhoduje, \bf zda vůbec \rm text číst.

Když někdo tam venku hledá odpověď na svůj problém, zadá knihovnici nebo dnes spíše vyhledávacímu serveru klíčová slova, která se jeho potíží dotýkají. Na základě shody těchto klíčových slov a klíčových slov uvedených autory dostane seznam názvů prací, které by mu mohly nabízet řešení. Dobře sestavený název práce badateli pomůže vytipovat takové texty, které by mohly mít vztah k jeho problému a mohl by se zajímat o jejich přečtení.

A tady právě přichází na scénu abstrakt. Badatel si čte abstrakt vytipovaných prací a~rozhoduje se, zda práci skutečně chce číst, nebo jestli se v tomto případě jeho filtr založený pouze na jednořádkovém nadpisu zmýlil.

V tuto chvíli obvykle ještě nemá stažené nějaké PDF s celým textem, natož aby měl v ruce vytištěný fascikl. Abstrakty jsou určeny k tomu, aby byly \bf mimo text\rm , aby ležely na serverech agregujících vědecké texty. Proto první pravidlo je, že abstrakt musí fungovat samostatně -- pokud obsahuje odkazy do literatury nebo se odvolává na text (\uv{Výkonnost metody je přehledně shrnuta na straně 51.}), nedělá badateli dobrou službu, což badatel ocení tím, že si o autoru nepomyslí nic hezkého, práci si nepřečte a autora neocituje.

\subsection*{Kdy a jak psát Abstrakt}
Může dávat smysl psát abstrakt na závěr celého psaní -- jako shrnutí a skutečné anotování sepsaného díla. Já jsem vyznavačem opačného přístupu -- abstrakt píšu na samém začátku. Když píšu vědecký článek, začínám sepsáním velkého počtu klíčových slov, jež se textu dotýkají. Bývá jich více, než potom uvedu jako ona charakteristická klíčová slova používaná k indexování. Ujasňuji si tím prostor, kde se článek pohybuje -- o čem je třeba hovořit, co je v textu podstatné, čeho se dotýká. Hned po ujasnění klíčových slov formuluji nadpis a~právě abstrakt.

Považuji za mimořádně užitečné ujasnit si právě ony čtyři části abstraktu -- Jaký problém se řeší? Jaké řešení práce nabízí? Jaké jsou přesně výsledky? Jaký je jejich význam? Když je toto jasné, text se píše skoro sám. Pokud toto má být nejasné, jak u všech všudy je možné vůbec dát dohromady smysluplnou větu v samém textu?

\subsection*{Doporučená struktura abstraktu}
Abstrakt vědecké práce se může skládat ze čtyř částí a pak být opravdu užitečný. Každá část se bude skládat z nějakých dvou, tří vět, někdy postačí jedna.

V byznysu se vžil slovesný útvar \uv{elevator pitch} -- představení ve výtahu. Ne náhodou jeho struktura připomíná právě doporučovanou strukturu abstraktu. Opravdu, autor odborného textu má do abstraktu napsat právě to, co by říkal o své práci, kdyby na to měl nejvýše dvě minuty a nemohl použít žádných slajdů, obrázků, textu. O čem by tedy měl mluvit?

\paragraph{První část -- Jaký se řeší problém? Jaké je téma? Jaký je cíl textu?}
\begin{itemize}
  \item{Tato práce řeší.}
  \item{Cílem této práce je.}
  \item{Zaměřil jsem se na.}
\end{itemize}
Nepatří sem úvodní pohádky charakteristické pro špatný odborný sloh: \uv{Naše poslední pětiletka staví před nás nové a smělé cíle}, \uv{S rozvojem výpočetní techniky a zejména zobrazovacích zařízení je stále důležitější \ldots} Ty nepatří do dobrého textu nikam, ale do~abstraktu tím méně. Pokud dokážete vyjádřit účel svého textu v jedné větě o pár slovech, udělejte to a nepřidávejte nic navíc. Stručnější zde vždy znamená lepší.

\paragraph{Druhá část -- Jak je problém vyřešen? Cíl naplněn?}
\begin{itemize}
  \item{Zvolený problém jsem vyřešil pomocí toho a toho.}
  \item{V řešení bylo použito metody té, postupu toho a analýzy oné.}
  \item{Práce představuje algoritmus takový, který.}
  \item{Data jsem zpracovával pomocí těch a těchto nástrojů a provedl vyhodnocení takové.}
  \item{Podstatou našeho algoritmu je.}
\end{itemize}

Pokud je podstatou sepisovaného odborného textu nová metodologie (= \uv{jak něco dělat}), patří přesně sem její popis. Pokud se představované řešení skládá ze tří částí, pravděpodobně v této části abstraktu budou tři věty, z nichž každá se bude věnovat jedné části řešení. Dobrý abstrakt v této části bude upřímný a přesný -- nebude si schovávat \uv{odhalování svých tajemství} až do textu. Vágní formulace podstaty řešení v abstraktu obvykle znamená, že autoři buď neumí psát a nebo vlastně nemají o čem -- ani jedno není zrovna výzva ke stažení a čtení mnoha stran textu.

\paragraph{Třetí část -- Jaké jsou konkrétní výsledky? Jak dobře je problém vyřešen?}
\begin{itemize}
  \item{Podařilo se dosáhnout úspěšnosti 87,3\,\%.}
  \item{V práci jsme vytvořili systém, který.}
  \item{Vytvořené řešení poskytuje ty a ty možnosti.}
  \item{Provedeným výzkumem jsme zjistili, že.}
\end{itemize}

Není špatným zvykem uvést v této části konkrétní číslo -- \uv{existující metodu XY jsme zrychlili pětkrát}. Pokud přínos práce není možné shrnout do dvou nebo tří vět, někde je něco velmi špatně a celý text pravděpodobně nestojí za psaní.

\paragraph{Čtvrtá část -- Takže co? Čím je to užitečné vědě a čtenáři?}
\begin{itemize}
  \item{Přínosem této práce je.}
  \item{Hlavním zjištěním je.}
  \item{Hlavním výsledkem je.}
  \item{Na základě zjištěných údajů je možné.}
  \item{Výsledky této práce umožňují.}
\end{itemize}

Při psaní vědeckých článků já sám obvykle bojuji s podobností části třetí a čtvrté. Vskutku, obě hovoří o tom, co jsou výsledky a přínosy textu. Účelem třetí části je jmenovitě a konkrétně jmenovat dosažené výsledky, úkolem části čtvrté je interpretovat jejich význam. Asi ničemu nevadí, když tato dvě sdělení do jisté míry splynou a část třetí a čtvrtá nejen že nemají každá vlastní odstavec, ale prolínají se dokonce ve společných větách.

\paragraph{Nultá část -- O co jde? Kde jsme?}
\begin{itemize}
  \item{Práce je řešena v kontextu tom a tom.}
  \item{Nauka ta a ta se zabývá studiem toho a toho.}
  \item{Stavíme na těchto a oněch nedávných pokrocích v naší oblasti.}
\end{itemize}

Někdy je nutné na sám začátek abstraktu vložit kratičké uvedení kontextu, ve kterém se~celá záležitost vlastně odehrává. Může to být přínosné~u vskutku obskurního a esoterického oboru, který leží stranou hlavního proudu. Obvykle tato část ovšem nebývá nutná a~věty v~ní obsažené bývají prototypy ohavné, rádobyodborné vaty. Je dobrou praxí zapomenout, že se tato část v abstraktu může vyskytovat. Když někdo, kdo je odborníkem v~oboru práce, přece po přečtení abstraktu zavrtí hlavou: \uv{Vůbec nevím, o čem tady můžete psát,} pouze tehdy je vhodné vložit nějaké věty s uvedením kontextu.

\subsection*{Inovace není Ignorance}

Popisuji v tomto textu jakýsi obecný model obecné diplomky. Ještě ke všemu se na začátku zaklínám, že to je můj názor a vkus a jsem zvědavý na názory a vkusy alternativní (což jsem!). Každý diplomant (Mgr. i Bc.) přitom cítí, že jeho diplomka je speciální a výjimečná. Tudíž se nebude držet nějakého schématu, které slouží pro běžné a průměrné diplomky -- tj. pro ty ostatní. Setkávám se s dobrými důvody, proč se od výše naznačeného schématu odchýlit a každoročně některým studentům odchýlení od schématu sám doporučuji. Vskutku, každá diplomka je jedinečná a zvláštní. Kdyby ne, nemusely by se psát, stačilo by je kopírovat. Ovšem vždycky před tím, než vybočíte ze standardního a kanonického způsobu organizování odborného textu, dejte si tu práci ho poznat, pochopit a zvládnout. Způsob vědecké práce, strukturování odborného textu, nebo třeba citování pramenů, může vypadat rigidně a neohrabaně, je to ale zatím ten nejlepší způsob, který jsme jako lidstvo dokázali vymyslet. Pokud ho ovládnete, pochopíte jeho výhody a nevýhody a inovujete ho, je to v pořádku a jste vítáni. Pokud se jím odmítnete zabývat, pravděpodobně neprovedete hodnotnou inovaci, ale vytvoříte \uv{paskvil}.


\chapter{Příprava základní struktury práce}
\label{struktura}

V této kapitole jsou nejprve uvedeny obecné zásady pro psaní odborného textu a po nich následuje detailnější popis doporučeného postupu přípravy struktury a základní osnovy práce.

Před začátkem psaní textu práce je vždy vhodné zeptat se svého vedoucího, co Vám poradí a zda nemá nějakou svoji aktuální stránku s radami a pokyny. Jeho zaměření bude pravděpodobně odpovídat zaměření Vaší práce a poradí Vám tu nejvhodnější strukturu, které byste se měli držet. Dozví-li se autoři tohoto souboru o další sbírce užitečných rad, jistě sem v budoucnu budou zařazeny.

Tento text se zaměřuje na obecná doporučení a obecnou strukturu práce, kterou je vždy potřeba  modifikovat a popřemýšlet o ní na základě konkrétního zadání \cite{Cernocky}.

\section{Užitečné rady pro psaní odborného textu}

Následující pokyny jsou dostupné též na školních webových stránkách~\cite{fitWeb}. Přehled základů typografie a tvorby dokumentů s využitím systému \LaTeX{} je uveden v~knize od~Jiřího Rybičky~\cite{Rybicka}.

Hodnocenou součástí potenciálního inženýra je mimo jiné i jazyková kvalita a čistota. Naším cílem je vytvořit jasný a~srozumitelný text. Vyjadřujeme se proto přesně, píšeme dobrou češtinou či slovenštinou (případně angličtinou) a~dobrým slohem podle obecně přijatých zvyklostí. Předpokládá se dodržování pravopisných norem zvoleného jazyka práce a dodržování odborného názvosloví. Slangové výrazy jsou nepřípustné. Při pochybnostech o~překladu či přepisu cizích pojmů využijte literatury dostupné v knihovně FIT.

Text má upravit čtenáři cestu k~rychlému pochopení problému, předvídat jeho obtíže a~předcházet jim. Dobrý sloh předpokládá bezvadnou gramatiku, správnou interpunkci a~vhodnou volbu slov. Snažíme se, aby náš text nepůsobil příliš jednotvárně používáním malého výběru slov a~tím, že některá zvlášť oblíbená slova používáme příliš často. Pokud používáme cizích slov, je samozřejmým předpokladem, že známe jejich přesný význam. Ale i~českých slov musíme používat ve správném smyslu. Např. platí jistá pravidla při používání slova {\it zřejmě}. Je {\it zřejmé} opravdu zřejmé? A~přesvědčili jsme se, zda to, co je {\it zřejmé}, opravdu platí? Pozor bychom si měli dát i~na příliš časté používání zvratného se. Například obratu {\it dokázalo se, že \ldots{}} zásadně nepoužíváme.

Za pečlivý výběr stojí i~symbolika, kterou používáme ke {\it značení}. Máme tím na mysli volbu zkratek a~symbolů používaných například pro vyjádření typů součástek, pro označení hlavních činností programu, pro pojmenování ovládacích kláves na klávesnici, pro pojmenování proměnných v~matematických formulích a~podobně. Výstižné a~důsledné značení může čtenáři při četbě textu velmi pomoci. Je vhodné uvést seznam značení na začátku textu. Nejen ve značení, ale i~v~odkazech a~v~celkové tiskové úpravě je důležitá důslednost.

S tím souvisí i~pojem z~typografie nazývaný {\it vyznačování}. Zde máme na mysli způsob sazby textu pro jeho zvýraznění. Pro zvolené značení by měl být zvolen i~způsob vyznačování v~textu. Tak například klávesy mohou být umístěny do obdélníčku, identifikátory ze~zdrojového textu mohou být vypisovány {\tt písmem typu psací stroj} a~podobně.

Uvádíme-li některá fakta, neskrýváme jejich původ a~náš vztah k~nim. Když něco tvrdíme, vždycky výslovně uvedeme, co z~toho bylo dokázáno, co bude dokázáno v~našem textu a~co přebíráme z~literatury s~uvedením odkazu na příslušný zdroj. V~tomto směru nenecháváme čtenáře nikdy na pochybách, zda jde o~myšlenku naši nebo převzatou z~literatury.

Abychom mohli napsat odborný text jasně a~srozumitelně, musíme splnit několik základních předpokladů:
\begin{itemize}
\item Musíme mít co říci,
\item musíme vědět, komu to chceme říci,
\item musíme si dokonale promyslet obsah,
\item musíme psát strukturovaně.
\end{itemize}

\subsection*{Musíme mít co říci}
Nejdůležitějším předpokladem dobrého odborného textu je myšlenka. Je-li myšlenka dost závažná, tak přetrvá, i když je neobratně a zmateně podaná. Chceme-li však myšlenku podat co nejvýstižněji a ušetřit tak čtenáři čas, musíme dodržet určité zásady, o kterých pojednáme dále.

\subsection*{Musíme vědět, komu to chceme říci}
Dalším důležitým předpokladem dobrého psaní je psát pro někoho. Píšeme-li si poznámky sami pro sebe, píšeme je jinak než výzkumnou zprávu, článek, diplomovou práci, knihu nebo dopis. Podle předpokládaného čtenáře se rozhodneme pro způsob psaní, rozsah informace a~míru detailů.

\subsection*{Musíme si dokonale promyslet obsah}
Musíme si dokonale promyslet a~sestavit obsah sdělení a~vytvořit pořadí, v~jakém chceme čtenáři své myšlenky prezentovat.
Jakmile víme, co chceme říci a~komu, musíme si rozvrhnout látku. Ideální je takové rozvržení, které tvoří logicky přesný a~psychologicky stravitelný celek, ve kterém je pro všechno místo a~jehož jednotlivé části do sebe přesně zapadají. Jsou jasné všechny souvislosti a~je zřejmé, co kam patří.

Abychom tohoto cíle dosáhli, musíme pečlivě organizovat látku. Rozhodneme, co budou hlavní kapitoly, co podkapitoly a~jaké jsou mezi nimi vztahy. Diagramem takové organizace je graf, který je velmi podobný stromu, ale ne řetězci. Při organizaci látky je stejně důležitá otázka, co do osnovy zahrnout, jako otázka, co z~ní vypustit. Příliš mnoho podrobností může čtenáře právě tak odradit jako žádné detaily.

Výsledkem této etapy je osnova textu, kterou tvoří sled hlavních myšlenek a~mezi ně zařazené detaily.

\subsection*{Musíme psát strukturovaně}
Musíme začít psát strukturovaně a~současně pracujeme na co nejsrozumitelnější formě, včetně dobrého slohu a~dokonalého značení.
Máme-li tedy myšlenku, představu o~budoucím čtenáři, cíl a~osnovu textu, můžeme začít psát. Při psaní prvního konceptu se snažíme zaznamenat všechny své myšlenky a~názory vztahující se k~jednotlivým kapitolám a~podkapitolám. Každou myšlenku musíme vysvětlit, popsat a~prokázat. Hlavní myšlenku má vždy vyjadřovat hlavní věta a~nikoliv věta vedlejší.

I k~procesu psaní textu přistupujeme strukturovaně. Současně s~tím, jak si ujasňujeme strukturu písemné práce, vytváříme kostru textu, kterou postupně doplňujeme. Využíváme ty prostředky DTP\footnote{Desktop publishing (DTP) -- tvorba tištěného dokumentu na počítači.} programu, které podporují strukturovanou stavbu textu (předdefinované styly pro nadpisy a~bloky textu).

\subsection*{Nikdy to nebude naprosto dokonalé}
Když jsme už napsali vše, o~čem jsme přemýšleli, uděláme si den nebo dva dny volna a~pak si přečteme sami rukopis znovu. Uděláme ještě poslední úpravy a~skončíme. Jsme si vědomi toho, že vždy zůstane něco nedokončeno, vždy existuje lepší způsob, jak něco vysvětlit, ale každá etapa úprav musí být konečná.

\section{Komu se píše diplomka}
Tato podkapitola byla převzata z blogu prof. Herouta \cite{Herout}.

\bigskip
\noindent \bf Pište svou diplomku pro studenta, který má na Vaše dílo navázat. \rm
\bigskip

Představte si, že na Vaší práci bude dál pracovat student Franta, asi tak stejně chytrý jako Vy sami. Máte teď čtyři hodiny na to, abyste mu svou práci ukázali, zasvětili ho do~všeho, co je potřeba, a on pak bude pokračovat sám. Franta je studentem stejné školy jako Vy a~ví asi tolik, co průměrný student, není žádným super odborníkem na obor Vaší diplomky, ale rozhodně není hloupý a řešeného tématu se neštítí. Franta, tak jako Vy, se~o~tom, že bude po Vás pokračovat, zrovna dozvěděl, takže ještě neměl čas si něco k~tématu nastudovat.

Bude dobré začít tím, že se Franta dozví, co je cílem práce, proč se to celé dělá, co mají být výsledky.

Nikdo soudný by hodinu z vyměřeného času s Frantou nestrávil řečněním typu: \uv{\mbox{Internet} byl vytvořen americkou armádou v roce 1962, pak v roce 1991 v CERNu udělali www, a~nyní se používá v nejrůznějších oblastech lidské činnosti.} (to vše na šesti stranách s~mnoha odkazy a obrázky).
Franta obvykle nepotřebuje několikastránkové skriptum o detailech barevných modelů pro reprezentaci obrázků, historii a detaily Houghovy transformace, kompletní popis vrstev referenčního modelu ISO/OSI, ani řadu koláčových grafů o~zastoupení jednotlivých mobilních platforem na trhu za posledních deset let.
Franta potřebuje nasměrovat na~cenné zdroje, které Vám při řešení pomohly, a chce letmý popis nástrojů a~algoritmů, které byly nutné pro řešení: \uv{Je potřeba nástroj XY, který slouží k tomuhle a~tamtomu, hlavně jeho modul PQ, který se používá tehdy a~tehdy. Nejlepší je k tomu tato dokumentace.}

Řekněte Frantovi hodně o tom, co se při řešení osvědčilo a co pomáhalo, ale upozorněte ho i na to, co nejdřív vypadalo jako dobrý nápad, ale pak se ukázalo jako zbytečné nebo nefunkční.

Dobře dávkujte úroveň detailu. Nějakou optimalizační fintu rozeberete ve zdrojovém kódu řádek po řádku, nějaký pomocný modul přejdete jedním odstavcem s popisem vstupů a výstupů a odkazem na použitou knihovnu.

Představte si průběh toho čtyřhodinového sezení s Frantou.
\begin{itemize}
  \item{O čem byste asi mluvili na začátku, kdy se Franta teprve začíná orientovat?}
  \item{Co jsou věci, které by rozhodně měly zaznít?}
  \item{Jaké obrázky byste v průběhu sezení malovali?}
  \item{Na co by se Franta vyptával, protože je to důležité a přitom to není samozřejmé?}
  \item{Na jaká omezení a nedodělanosti byste Frantu potřebovali upozornit, aby neupadl do~nějaké pasti?}
  \item{Jak vlastně Franta může pokračovat? Co jsou otevřené záležitosti, které by ještě stálo za to vyzkoušet a vylepšit?}
  \item{Co byste říkali úplně první (úvod) a úplně poslední (závěr) minutu sezení?}
\end{itemize}


\section{Struktura diplomové práce -- Pět kapitol}
Není-li dále uvedeno jinak, tato podkapitola byla převzata z blogu prof. Herouta \cite{Herout} (částečně inspirovaného knihou, kterou napsal Jean-Luc Lebrun \cite{Lebrun2011}) a z dokumentu na osobní stránce prof. Zemčíka~\cite{Zemcik}.
\bigskip

Diplomová práce je činnost, kterou student vyvíjí po dva semestry studia a pak o ní sepíše knížečku. Rozšířená terminologická chyba je, že se té knížečce, která je o činnosti sepsaná, říká diplomová práce. Ta knížečka je ve skutečnosti technická zpráva o provedené roční činnosti a diplomová práce je ta roční činnost.

Diplomantova roční činnost zahrnuje za prvé studium: \uv{Co už v oblasti mého zadání existuje? Jak to dělají jiní?} V rámci diplomky člověk dále nějaké věci vymyslí a navrhne: \uv{Zadaný problém lze řešit tak a nebo tak, já k němu přistoupím tímto způsobem, protože na~zvolené platformě je to nejefektivnější.} To, co řešitel navrhl, by měl po sobě ověřit tím, že to implementuje a vyhodnotí: \uv{Pro implementaci jsem zvolil ty a ty nástroje, celý systém rozvrhl do takových modulů. Výsledek je takhle rychlý, má takovou úspěšnost a~reakce uživatelů jsou takové a takové.}

Základní struktura diplomové práce podle prof. Herouta tedy je:
\begin{enumerate}
  \item{Úvod (1 strana)}
  \item{Co bylo třeba vystudovat (vč. zhodnocení současného stavu; 40\,\% rozsahu)}
  \item{Nové myšlenky, které tato práce přináší (30\,\%)}
  \item{Implementace a vyhodnocení (30\,\%)}
  \item{Závěr (1 strana)}
\end{enumerate}

Není chybou, když text má právě 5 takových kapitol, není ani chybou, když je některá z~nich rozdělena na dvě části -- o tom dále. Obvykle je velkou chybou, když tam některá část chybí, nebo má nápadně odchylný rozsah. Názvy kapitol nemusí kopírovat tuto strukturu. Samozřejmě, samotný obsah práce je nadřazen všem zásadám a pokud tedy bude dobrý důvod strukturu porušit, tak to udělejte.

Na této základní struktuře se řada vedoucích shoduje, byť různí vedoucí doporučují rozdílné názvy kapitol a např. zhodnocení současného stavu lze umístit nejen do 2., ale i~do~3. kapitoly jak to doporučuje prof. Zemčík:
\begin{enumerate}
  \item{Úvod (1--2 strany)}
  \item{Shrnutí dosavadního stavu (40--50\,\% celkového rozsahu)}
  \item{Zhodnocení současného stavu a návrh řešení (3--5 stran)}
  \item{Popis vlastní práce (cca 40\,\% celkového rozsahu)}
  \item{Závěr (max. 1 strana)}
\end{enumerate}

Názory vedoucích se dle zaměření práce liší i v rozsazích, jak je vidět např. v~doporučeních dr. Berana \cite{Beran}:
\begin{enumerate}
  \item{Úvod (1 stránka)}
  \item{Teorie (1/3 stran)}
  \item{Návrh řešení (1/3 stran)}
  \item{Realizace, experimenty a vyhodnocení (1/3 stran)}
  \item{Závěr (1 stránka)}
\end{enumerate}

U prakticky zaměřených prací, pro která jsou zásadní data a uživatelská rozhraní, lze využít doporučení od doc. Černockého \cite{Cernocky}:
\begin{enumerate}
  \item{Úvod (jednotky stran)}
  \item{Teoretická část (cca 10 stran)}
  \item{Data (jednotky stránek)}
  \item{Popis Vašeho algoritmu a jeho testování (cca 10 stran)}
  \item{Návrh a implementace (pár stran)}
  \item{Uživatelské rozhraní (pár stran)}
  \item{Testování (cca 10 stran)}
  \item{Závěr (jednotky stran)}
\end{enumerate}

\section{Diplomka -- komiksová edice}
Tato podkapitola byla převzata z blogu prof. Herouta \cite{Herout}.

Diplomka (bakalářka je taky diplomka) je poměrně komplexní dílo. Skládá se z velkého počtu písmenek. A ta písmenka nejsou jen tak za sebou, ale jsou uspořádána hierarchicky do~kapitol. Celé to musí mít nějakou logiku, nějaký sled -- čtenář se musí nejdřív dozvědět jedny věci, aby mu pak šlo přesvědčivě předat věci jiné. Musí tam být obrázky, tabulky, vzorce; zároveň si tyto ne-textové věci musí s okolním textem povídat a navzájem se doplňovat. Musí se zcela pokrýt oficiální zadání, a vše musí být doručeno k nějakému přesnému datu, vytištěné a svázané. Pokud, nad to všechno, má být diplomka dobrá, musí to všechno být uděláno dobře. Když se peče koláč z deseti surovin a jedna z nich je zkažená, celý koláč bude hnusný. Musí klapnout všechno.

\subsection*{Jak to všechno pohlídat? Odkud zaútočit?}
Když s kolegy píšeme článek (poslední dobou asi tak pořád), v hodně rané fázi připravíme něco, čemu říkáme \uv{Comics Edition}. Děláme to jednak proto, že já na tom trvám, a dvak proto, že nám to dosti pomáhá. Třeba to pomůže i Vám s diplomkou.

Nejprve si ujasněte odpovědi na následující otázky:
\begin{itemize}
  \item{Jak byste vystihli podstatu svého řešení ve třech až pěti krátkých větách?}
  \item{Jaké jsou silné stránky Vašeho řešení?}
  \item{Jaké jsou konkrétní argumenty, že to, co jste udělali, je dobré?}
  \item{Kdyby chtěl někdo být na Vaši práci zlý -- co by vytkl?}
  \item{Co byste mu odpověděli?}
  \item{Jaká klíčová slova by měl člověk zadat do vyhledávače, aby Vaše diplomka byla relevantní odpovědí?}
\end{itemize}

Pokud máte, můžeme jít na věc \ldots

\subsection*{Hned založit TEN dokument}
Občas vídám postup, že někdo píše \uv{předběžnou} verzi diplomky do nějakého poznámkovadla. Je to za prvé práce navíc a za druhé zbytečná. Je třeba hned založit dokument, ve~kterém svůj boj dokončíte, a ze kterého výsledek vytisknete.

\subsection*{Nadpisy kapitol}
Důležitou součástí komixového vydání, o něž se tu snažíme, jsou nadpisy kapitol. Vložte je do dokumentu. Vložte je tak, jak budou z hlediska formátování -- žádné provizorní seznamy: \uv{Já to pak předělám}. Chcete vidět, jak přesně to bude vypadat, jak se vyloupne celý automaticky sestavený obsah. Vložte je jak budou z hlediska jejich znění. Nadpis kapitoly říká, co v kapitole bude. Nadpisy kapitol tvoří kostru celého díla, již pak obalujete masem a kůží textu a obrázků.

Ze všech slov, která jsou v diplomce, jsou slova v nadpisech \textbf{ta nejdůležitější}. Věnujte jim opravdu mimořádnou pozornost.

\subsection*{Obrázky}
Obrázek vydá za tisíc slov. Prošel jsem 8 posledních článků, jež jsem spoluautoroval. Dohromady mají 80 stran, obsahují 87 obrázků a 17 tabulek, tj. 1,3 vizuálního sdělení na~stránku (včetně stránek obsahujících reference do literatury, úvodních stránek s abstrakty a vůbec všeho). Mnohé obrázky (tak půlka) se ve skutečnosti skládají z několika podobrázků, zvlášť odkazovaných. Těch jsem v řečených 8 článcích napočítal 221, tj. v průměru 3 vizuály na~stranu. Taková je moje představa o roli obrázků v odborném textu. Nemyslím si, že by mohla existovat vážně míněná diplomka, která by měla \uv{příliš mnoho obrázků}.

Již v rané verzi komixového vydání diplomky hleďte promyslet, kde se obrázky budou vyskytovat a jaké. Obrázky ještě nemusíte mít hotové. Zdaleka. Ještě nevíme, co přesně na~obrázku bude. Ještě nevíme, jaký pod ním bude popisek. Co víme je, že tu nějaký takový obrázek bude a že se bude skládat z více podobrázků a tak ho hned vložíme. Zabere to tak minutu (vektorovou podobu \uv{TODO Image} máme už dávno uloženou a je součástí této šablony) a hned se víc rýsuje, jak text bude vypadat.
U některých obrázků už dokonce máme představu, jak budou vypadat -- konceptuálně. Nakreslíme na papír nebo na tabuli, vyfotíme mobilem, obrázek vložíme, aby držel místo tomu, jenž přijde po něm a bude nakreslený pořádně (vektorově v Inkscape\footnote{\url{https://inkscape.org}} nebo vygenerovaný Gnuplotem\footnote{\url{http://www.gnuplot.info/}}).

Jen tak na okraj: Obrázek vydá za tisíc slov. Hloupý obrázek vydá za tisíc hloupých slov. A když už jsme u těch obrázků: Kdo vkládá věci, které by měly být vektorové (schémata, grafy, nákresy, diagramy, prakticky všechno kromě fotek a~snímků obrazovek), jako rastrové obrázky, a kdo vkládá snímky obrazovek (a podobné věci, které mají být přesně) se ztrátovou kompresí (obvykle JPEG), nemůže očekávat pozitivní hodnocení práce.

\subsection*{Objem textu}
Zrovna tak jako vkládáme obrázky, které ještě nemáme, vkládáme i text, který ještě nemáme. V \LaTeX{}u je na to krásný příkaz \texttt{\textbackslash Blindtext}\footnote{Stručný tutorial: \url{https://texblog.org/2011/02/26/generating-dummy-textblindtext-with-latex-for-testing/}}. Kdo ho (ke své škodě) nechce nebo neumí používat, použije \url{http://lipsum.com}. Pomůže pisateli tušit přibližný rozsah celého díla, hustotu obrázků v textu a další charakteristiky textu při pohledu shůry. Vytvořit si takový odhad trvá třeba 5 minut. Pro orientaci v rozdělaném textu je rozumné tento nijaký text vybarvit šedou barvou (inteligent si na to udělá příkaz, ať se pak barva snadno jednotně změní pro celý dokument). Zkušenost říká, že bez barev v tom člověk začne mít slušný nepořádek -- co už je hotové, co ne, na čem je potřeba pracovat. Je radno investovat pár minut do zprovoznění balíčku pro barvení textu. V rámci této šablony můžete použít příkaz \verb|\todo|, např. \todo{Toto je třeba dokončit}.

Geneze každé kapitoly ať začíná tím, že obsahuje třeba 3-5 kusů TODO a nějaké to Lorem ipsum. Každé TODO se pak postupně transformuje na větší počet TODO menšího rozsahu, nebo na text, na obrázek, další podkapitoly, cokoliv. TODO střídavě přibývají a~ubývají, práce se přitom vždycky o kousek pohne.

\subsection*{Jak s tím celým pracovat}
Když na to člověk sedne a \LaTeX{} zrovna nemá špatný den, za hodinu je hotová diplomka (bakalářka je taky diplomka) o správném počtu stran, obsahující představu, co kde bude. Začíná se podobat výsledku, který má přijít až za nějakou dobu, po nějaké práci.

Dokument pak už v zásadě nenarůstá, ale transformuje se. Je velký rozdíl sednout si před prázdnou bílou obrazovku a \uv{psát diplomku} a vzít si jedno TODO a napsat místo něj odstavec. To první je těžké a někdy to prostě nejde. To druhé jde: má to svůj začátek a konec. Ví se, co se má udělat.

Pořád platí, že diplomka se nenapíše sama, ale jde to lépe a výsledek má spíše hlavu a~patu.

\section{Jak pojmenovat kapitoly}
Dobře publikovat neznamená jen posypávat co nejvíc papíru co nejvíce písmenky. Renomé vědce vzniká vlastně až tím, že jinému vědci je jeho práce natolik užitečná, že ho cituje ve~své práci. Proto je potřeba, aby svůj článek napsal dobře: nikdo nebude citovat práci, která je humpolácky napsaná, protože by se sám shodil.
Humpolácky napsaný článek ovšem nikdo nebude citovat už z toho důvodu, že \bf ho nenajde\rm . Už dávno před nějakým internetovým SEO vědci používali při psaní různé \uv{fligny} tak, aby je další vědec, když dělá svoji rešerši, zařadil mezi svůj materiál, jejž přečte, z něhož si nadělá poznámky a který -- nakonec a~logicky -- ocituje ve svém díle.

Na světě je moře článků. Když vědec Tonda hledá materiál relevantní pro svou práci, zadá klíčová slova (kdysi papírově do knihovny, nyní elektronicky do příslušného vyhledávače) a vypadne mu hromada nálezů -- tj. názvů. První krok, aby článek někdo ocitoval, je mít dobrý název. Tak dobrý, aby Tondu zaujal a on si název rozkliknul ve snaze zjistit více. Název je \bf prvním filtrem\rm .

Články, jež prošly prvním filtrem, si Tonda rozklikne. To znamená, že vidí abstrakt článku. Abstrakt je \bf druhým filtrem \rm a hodně na něm záleží. Je to jako dostat se do~druhého kola pohovoru k vysněné práci.

Když tedy Tondu zaujme nadpis i abstrakt, stáhne si celé PDF článku a rychle ho proskroluje, aby si udělal představu: vytiskne ho, nebo okno s článkem zavře a bude se věnovat desítkám jiných? To je \bf třetí filtr \rm a je to jako být mezi pár posledními uchazeči o~práci snů. Na co se Tonda dívá ve třetím kole, během svého skrolování? Na vizuály, tj. na~obrázky, tabulky, vzorce, a právě na nadpisy kapitol. Projde Váš článek třetím filtrem? Bude s ním Tonda pracovat? Na obrázky se zaměříme jindy, tato podkapitola je o nadpisech.

S diplomkami to může být trochu jinak. Ne každý pisatel diplomky stojí o to, aby ji lidé četli. Tušíme, že existují tací, kteří si přejí spíše pravý opak. Pojďme ale v tomto návodu pracovat s hypotézou, že pisatel chce napsat dobrý text, který by mohl být lidstvu užitečný a stojí za čtení. Tj. za který je možné dostat rozumnou známku.

\subsection*{Klíčová slova -- půl úspěchu}
Jedna z nejlepších rad pro psaní článků (a obecně odborného textu), kterou jsem kdy slyšel, není úplně intuitivní a samozřejmá.
\bf Napište si klíčová slova, jež by člověk měl napsat do vyhledávače, aby mu vypadlo Vaše dílo jako relevantní odpověď. \rm
Popusťte uzdu fantazii, klidně to vezměte ze široka. Přemýšlejte o aplikacích Vaší práce. O souvislostech. Sepište všechna klíčová slova, bude to na několik řádků. Klíčové slovo je i sousloví -- typicky dvou nebo tří slov. Vyberte z nich ta důležitá. K tomu je potřeba intuice a zkušenost. Kde ty vzít, nevím, ale vždycky se můžete s někým (např. vedoucím práce) poradit. Až budete psát svůj třicátý odborný text, půjde to celkem hladce.

\bf Všechna důležitá klíčová slova se musí objevit v nadpisu článku nebo v nadpisech kapitol. \rm

\subsection*{Příliš obecný nadpis, příliš specifický nadpis}
Proč to s těmi klíčovými slovy? Protože nadpisy jsou směrovníky, ukazatele, které ukazují: \uv{Co hledáš, je tady!} Aby někdo o text stál, musí se v něm zorientovat. Potřebuje vědět, že text nabízí odpovědi na některé jeho otázky. Nadpisy mu v tom můžou pomoct, nebo ho přesvědčit o tom, že o text vlastně nestojí. Kdo si při stěhování napíše na všechny svoje krabice od banánů: \uv{RŮZNÉ VĚCI}, bude mít pravdivé a formálně správné popisky, ale nemusel nic psát. Obecný popisek k ničemu není.

Jednoslovný nebo dvouslovný nadpis kapitoly jsou obvykle podezřelé z toho, že jsou příliš obecné -- s výjimkou úvodu a závěru, kde názvy kapitol jsou kanonické (dané vyhláškou). Máte-li ve své práci jednoslovné nadpisy kromě dvou řečených, pravděpodobně je máte špatně. Název kapitoly, který by šel použít u jiné práce stejného oboru, třeba \uv{Implementace systému}, \uv{Základ zpracování obrazu}, \uv{Principy uživatelských rozhraní}, je podezřelý z~toho, že je příliš obecný. Lepší obvykle bude \uv{Implementace systému pro sledování pohybu much}, \uv{Algoritmy pro detekci objektů a sledování jejich trajektorií}, \uv{Principy uživatelských rozhraní jednoduchých webových systémů}.

Název kapitoly, který by šel použít na úplně rozdílných školách, je prakticky vždycky špatně -- příliš obecný. Nadpis \uv{Teorie} by mohl být použit na lesnické univerzitě, v IT, na~právech, na vysoké škole mlékárenské a sýrárenské. Je špatně. Nadpis \uv{Studium existujících řešení} je špatně. \uv{Průzkum dostupných technologií} je špatně. Ještě jsem neviděl nadpis, který by byl příliš specifický a nemyslím, že by něco takového mohlo existovat. Může být špatně -- tedy nevystihovat, co se v kapitole nachází. Pokud ale vystihuje, nemůže být příliš specifický.

Nevybízím tím k nadpisům na pět řádků. Většina dobrých a specifických nadpisů bude na jediném řádku a budou mít v průměru kolem pěti slov. Tu a tam nadpis přeteče na~druhý řádek a bude k tomu dobrý důvod. Sestavit dobré nadpisy -- dosti specifické a přitom ne~příliš dlouhé -- není lehké, ale vyžaduje to přemýšlení. Jako každá lidská činnost, která má za něco stát.

\subsection*{Zkratky v nadpise}
Zkratky v nadpise nemají co dělat, pokud nejsou úplně super-notoricky známé (ČR, AIDS, IT).

Je možné v první kapitole vysvětlit nějaký pojem a uvést, že dále se bude vyskytovat ve~zkratce. Je možné tuto zkratku používat dále v textu druhé kapitoly bez dalšího vysvětlení. Není ale možné tuto zkratku používat v nadpise druhé kapitoly, protože nadpisy čte vědec Tonda už ve třetím filtru při rozhodování, jestli se vůbec do první kapitoly pustí. Pokud Tonda při rychlém skenování článku narazí na něco, co v něm vzbudí dojem, že text je nějaký divný, nesrozumitelný, vlastně neví, co se tam píše (to je případ zkratky v~nadpise), článek zavře a už ho neotevře.

Odkazy do literatury a odkazy na další objekty v článku (obrázky, nadpisy, \ldots) do~nadpisu nepatří a ještě jsem neviděl případ, kdy by tam byly potřeba (a to už jsem viděl hodně případů, kdy se tam vyskytovaly).

\section{Obecné rady zkušených vedoucích}

Tato podkapitola obsahuje vybrané rady od dalších zkušených vedoucích, pod jejichž vedením již byly obhájeny stovky prací a kteří věnovali nemalé úsilí sepsání svých rad a jejich vystavení na web. Pro kompletní texty neváhejte navštívit jejich stránky, které naleznete v~literatuře \cite{Beran}, \cite{BeranPDF}, \cite{Cernocky} a \cite{Zemcik}.

\subsection*{Obecné rady dle dr. Berana}
Tato podkapitola byla převzata ze stránek dr. Berana \cite{Beran}, \cite{BeranPDF}.

Jak psát/nepsát
\begin{itemize}
  \item{Kapitoly číslujte maximálně do druhé úrovně, nadpisy nižších úrovní volte jako nečíslované a neuvádějte je v obsahu, výsledná práce i obsah budou mnohem přehlednější.}
  \item{Logické členění -- každý celek -- celá práce, každá kapitola, každá podkapitola má: úvod, stať a závěr:
    \begin{itemize}
      \item{úvod -- sděluje, co je obsahem celku, co se dočteme, co se řeší, uvádí do kontextu,}
      \item{stať -- řeší kontext, problém, detailně specifikuje problém, způsob řešení, postup řešení, výsledek řešení,}
      \item{závěr -- rekapituluje úlohu, shrnuje dosažené výsledky a jejich podstatu, uzavírá celek.}
    \end{itemize}}
  \item{Jedná se o technický text, nedoporučuji příliš mnoho osobních pocitů a \uv{povzdechů}.}
  \item{Nepoužívejte množné číslo \uv{MY} jsme udělali, chtěli apod.
    \begin{itemize}
      \item{používejte buď trpný rod, \uv{testy byly provedeny} namísto \uv{my jsme provedli testy} -- zejména v teoretické části, kdy jde o převzaté myšlenky,}
      \item{tam, kde chcete zdůraznit, že se jedná o Váš přínos, Vaši práci, Váš nápad apod. použijte \uv{já} -- návrh řešení, experimenty, realizace,}
      \item{protože MY (já-Vy, Vy-čtenář, Vy-svět) jsme nic neudělali, VY jste udělal(a).}
      \item{(To, že někde používáte nápady vedoucího, neřešte, to se očekává, je to Vaše práce na~jeho téma.)}
    \end{itemize}}
  \item{U převzatých obrázků/myšlenek/tabulek použijte \bf citaci zdroje\rm }.
  \item{Každý \bf nadpis \rm (kapitoly či podkapitoly) by měl být následován odstavcem textu, který čtenáře informuje, co se dočte v následující části, a který čtenáře uvede do~následující problematiky.}
  \item{Nepodceňujte úvod a závěr.}
\end{itemize}


\subsection*{Obecné rady od doc. Černockého}

Tato podkapitola je převzata ze stránek doc. Jana Černockého \cite{Cernocky}.

\begin{enumerate}
  \item{Přečtěte si pár dobrých DP/BP a snažte se vstřebat, jak taková dobrá práce vypadá. Příklady Vám rád dá Váš vedoucí.}
  \item{\textbf{Čeština/Slovenština nebo English?}
    \begin{itemize}
      \item Pokud umíte slušně anglicky a Vaše práce má potenciál být čtena někde jinde než na FIT VUT (součást mezinárodního projektu, práce pro nadnárodní firmu, popis SW, který chcete dát na GooglePlay, atd. atd.), velmi doporučuji psát anglicky. Můžete si 100x říkat, že to pak z češtiny přeložíte, ale už na to nikdy není čas. Navíc nemusíte psát háčky a čárky.
      \item  Pokud pracujete na BP/DP lokálního významu a víte, že Vaše angličtina za~moc nestojí, doporučuji naopak ušetřit sobě i vedoucímu a oponentovi utrpení a psát česky nebo slovensky. Rady pro psaní práce v angličtině a chyby, kterých se studenti často dopouštějí, najdete v~příloze~\ref{anglicky}.
    \end{itemize}}
  \item{Nemějte strach z toho, že budete mít málo stran! Nerozepisujte se zbytečně, nekopírujte zbytečné věci z Wikipedie -- jen tím naštvete Vašeho vedoucího a oponenta. Pokud budete následovat doporučenou strukturu a budete psát o tom, co jste skutečně udělali, budete mít na závěr dost kvalitního materiálu. }
  \item{Pište průběžně -- nemusíte psát přímo text technické zprávy se vším možným formátováním, ale mějte aspoň nějaký soubor README, kam si budete značit, na čem děláte, průběžné výsledky, co jste četli, co jste použili, o čem to zhruba je atd. Důrazně varuji před systémem \uv{hrnu práci, pak to celé napíšu} -- za půl roku už vůbec nebudete vědět, co jste dělali, a budete si na to (v lepším případě) horko těžko vzpomínat, v horším případě si to budete muset zopakovat. Průběžné psaní Vám také pomáhá strukturovat Vaší technickou zprávu.}
  \item{Používejte kontrolu pravopisu (spell-checker). Ušetřete vedoucího a oponenta opravování hloupých chyb (překlepů apod.). MS Word má kontrolu pravopisu dobrou, v~Linuxu je slušný ispall/aspell/hunspell (volá se např. z populárního textového editoru Emacs). Některé nástroje jsou nepoužitelné, např. ten v PSPadu propouští chyb jak ryb.}
	\item{Vedoucího průběžně seznamujte s aktuálním stavem práce a sdílejte s ním pracovní verze. Vedoucí by měl práci vidět zavčas! -- nejpozději přibližně 14 dní před odevzdáním, nejlépe průběžně po kapitolách (tabulky s výsledky/závěry třeba nemusí být ještě úplně hotové). Vedoucí Vám ji poškrtá, Vy se na něho naštvete, co si to dovoluje nad mým (přece tak krásným!) dílkem, za~půl dne vychladnete a zjistíte, že má vlastně pravdu, opravíte a další verze již bude mnohem kvalitnější. Pokud toto neuděláte a~odevzdáte verzi, kterou jste četl(a) jen Vy, případně nikdo, dostane \uv{plnou palbu} Vašich chyb oponent a podle toho bude vypadat jeho hodnocení.}
\end{enumerate}


\subsection*{Obecné rady od prof. Zemčíka}
Tato část textu je převzata z dokumentu na stránce prof. Zemčíka \cite{Zemcik}. Normálním písmem jsou uvedeny vybrané důležité obecně akceptované zásady, kurzivou jsou uvedena osobní doporučení \uv{navíc}.

Obsah práce by měl mít na jednu stranu textu práce nejvýše jeden \uv{řádek}. Členění práce do podkapitol by mělo být (s výjimkou úvodu a závěru) rovnoměrné. Tato zásada má přednost dokonce před standardním \uv{desetinným řazením} práce\footnote{Desetinné třídění dle ČSN ISO 7144 a ČSN 01 6910 -- výtah viz \url{http://web.ftvs.cuni.cz/hendl/metodologie/doporuceniupravydizprace.pdf}}. Pokud se týká písma, je třeba typy písma \uv{neplýtvat}. Méně je v tomto případě lépe. Orientačně byste měli kromě základního písma, nadpisů, popisů obrázků a tabulek a rovnic, používat jen minimum fontů, například italiku nebo tučné písmo pro zvýraznění textu (a to nejlépe ani ne obojí najednou) a vhodné písmo s \uv{konstantním krokem} (pevnou šířkou) pro úryvky zdrojových textů. Kromě toho je třeba dodržet formální šablonu předepsanou pro práci, která je uvedena na~fakultním webu.

\it Pokud se týká úvodu a závěru, velmi doporučuji je vůbec nečlenit do podkapitol a nechat je jako kompaktní bloky textu. Z výše uvedeného v podstatě taky vyplývá, že je vhodné kapitoly členit do podkapitol jen do \uv{1. úrovně}. Pokud byste cítili potřebu častějších nadpisů než v~průměru jednoho na
stranu (zamyslete se ale v tom případě nad tím, jestli to je skutečně potřeba), dá se použít v textu \uv{pomocný} nadpis bez číslování a bez zařazení do obsahu. U~nadpisů je též rozumné se zamyslet nad tím, zda jsou pro čtenáře rozumným vodítkem o textu a pokud tomu tak není, je asi rozumné je přejmenovat. Je též dobré za každým nadpisem zařadit alespoň kousek textu (tedy za nadpisem kapitoly třeba \uv{2.~Stav práce} nepokračovat hned \uv{2.1 Počátek stavu}, ale zařadit kus textu a pak pokračovat. Není také dobré končit kusy textu obrázkem nebo rovnicí, ale je vhodné zakončovat nějakým shrnujícím textem.

Typograficky podstatnými prvky jsou samozřejmě rovnice, obrázky, grafy, nadpisy apod. V~práci je ale jejich formát v podstatě \uv{tvrdě} upraven šablonou, takže se jimi zde nemá smysl zabývat příliš detailně. Přesto je třeba říci pár slov k jejich začlenění do práce. Především dbejte na to, aby byly
tyto \uv{grafické} prvky dobře graficky odděleny od textu a aby byl výsledek \uv{graficky pěkný}. Výmluvy typu \uv{toto udělal \LaTeX}, \uv{toto udělal Word} neobstojí. U obrázků je vhodné dodržet jednotný styl jejich začleňování do textu a přitom dbát na to, aby obrázky byly buď centrovány nebo aby byly (pokud jsou užší než stránka) umisťovány u~vnější strany stránky (vnější vzhledem k budoucí vazbě, při jednostranném tisku tedy v~zásadě vpravo).

\textit{Poznámka:} Pokud byste nebyli spokojení s tím, co je na začátku této kapitolky napsáno o obsahu a chtěli čtenáře \uv{lépe navést}, udělejte rejstřík -- to je jiný typografický útvar než obsah a tam můžete uvést hesel kolik se Vám jen zlíbí. Je to ale dost pracné a proto to příliš nedoporučuji.
\rm

Při psaní práce je třeba používat zásadně spisovný jazyk a vyvarovat se hovorových výrazů, případně slangu (včetně slangu odborného).

\it Při psaní práce velmi doporučuji se zaměřit zejména na následující body, v nichž se, bohužel, často chybuje:
\begin{itemize}
  \item{Využívání 1. osoby jednotného čísla (\uv{já}) je potřeba velmi omezit. Využívání 1. osoby množného čísla, i když se často využívá v beletrii, je potřeba téměř eliminovat úplně. Existují následující výjimky:
    \begin{enumerate}
      \item{1. osobu jednotného čísla je možno využívat v úvodu a závěru práce jako prostředek sdělení \uv{osobního dojmu} (například \uv{Obvykle se to dělá takto\ldots , ale já jsem zvolil jiný postup\ldots}), lze toho využít i ve zhodnocení současného stavu, ale v žádném případě ne ve shrnutí současného stavu.}
      \item{1. Osobu množného čísla je možno využívat v případě, že označujete část práce, kterou jste nedělali sami, ale dělali jste ji v nějakém kolektivu. Vzhledem k tomu, že BP/DP by měla být v zásadě Vaším dílem, je tím potřeba velmi šetřit a z práce by mělo být jasné, že alespoň 90\,\% práce jste dělali sami (například \uv{Program jsem napsal sám, ale pro testování programu jsem požádal o pomoc spolužáky a~spolu jsme provedli experiment\ldots}). Předcházejte otázkám typu \uv{Vy jste tu práci nedělal sám?}}
      \item{Výjimkou z obou výše uvedených pravidel jsou matematické texty, kde se tradičně užívá 1. osoby množného čísla (například \uv{Mějme krychli se stranou A\ldots}), tam 1. osobu samozřejmě užívejte bez omezení.}
      \item{Další možnou výjimkou jsou řečnické otázky, pokud je tedy v práci používáte. Celkem doporučuji 1. osobu jednotného či množného čísla v práci použít maximálně cca 10$\times$ (mimo případ 3, tam se neomezujte), dolní limit není, ale zase tak 1-2$\times$ se to i hodí.}
    \end{enumerate}}
  \item{Anglické výrazy by se v práci, obecně vzato, neměly moc vyskytovat. Vzhledem k tomu, že v našem oboru se jim ale nevyhneme, doporučuji takový postup, při kterém při prvním výskytu nějakého odborného pojmu uvedete obě verze (českou a anglickou) s tím, že tu, kterou nadále nebudete využívat, uvedete do závorky třeba i s komentářem (například \uv{\ldots octree (oktalový strom, nadále bude využívána jen anglická forma, protože to je zvykem i mezi odborníky v oboru)\ldots}).}
  \item{Zkratky by se při prvním užití v textu měly vysvětlit. Alternativou (asi lepší, ale pracnější) je uvedení seznamu zkratek, kde se na jednom místě přehledně vysvětlí všechny zkratky.}
  \item{Budoucí/minulý/přítomný čas by měl být v práci pojat tak, že obecně vzato je práce popisem obecně platných skutečností (pak používejte přítomný čas) v kombinaci s popisem Vaší práce (ta už proběhla,
tak používejte čas minulý). V plánech budoucí práce samozřejmě používejte čas budoucí. Nejvíc ze
všeho ale v tomto případě použijte \uv{cit} a přizpůsobte jazyk situaci tak, aby se práce dobře četla.}
\end{itemize}
\rm


\chapter{Jednotlivé kapitoly práce}
\label{kapitoly}

V této kapitole je detailně popsán význam a doporučený obsah jednotlivých kapitol technické zprávy a jsou zde uvedena doporučení od zkušených vedoucích.

Struktura práce je ovlivněná jejím zaměřením, průběhem práce a dosaženými výsledky. Je pravděpodobné, že ve svojí technické zprávě nebudete mít všechny níže uvedené kapitoly nebo že tam budete mít nějakou další kapitolu, která zde není zmíněna. Vždy je dobré plánovanou strukturu textu včas konzultovat se svým vedoucím.

Není-li uvedeno jinak, zbytek této kapitoly je převzat z osobních stránek prof. Zemčíka~\cite{Zemcik}, blogů prof. Herouta~\cite{Herout} a dr. Szökeho~\cite{rady} a dále ze stránek doc. Černockého~\cite{Cernocky} a~dr. Berana~\cite{Beran}.

Jednotlivé kapitoly by na sebe měly logicky navazovat. \bf Vyžívejte odkazů, např.: \rm \uv{\it Ve~funkci XY implementujeme matematickou formuli, kterou jsme odvodili v sekci 3.2, rovnice 7.\rm} Pokud se odkazujete dopředu, popište dvěma větami to, na co odkazujete. Čtenář by se neměl ztratit, nenuťte ho listovat. \bf Každá kapitola by měla být (v rámci možností) samonosná. \rm Pokud se celou Vaší prací prolíná pár důležitých pojmů, zkratek nebo úvah, vždy je na znovu vysvětlete (jakmile je použijete v dané kapitole). Pokud čtenář otevře Vaší práci uprostřed (např. kapitola 4), neměl by se okamžitě utopit ve zkratkách a~výrazech.



\section{Obsah}
\label{obsah}

Delší text -- jakým je diplomka -- je opatřen automaticky sestaveným obsahem. U diplomky se MUSÍ vejít na jednu stranu. U sedmisetstránkové knihy možná ne, i když by mohl a bylo by to fajn. U diplomky se MUSÍ vejít vždycky.

U diplomky ať obsah uvádí pouze nadpisy první a druhé úrovně, třetí úroveň ať tam není, je moc podrobná. Čtvrtá úroveň -- byť by se v obsahu neuváděla a byť by byla použitá jen v některých kapitolách -- je obvykle špatně sama o sobě.

Pokud jsou nadpisy kapitol správně sestaveny, pohledem na obsah (bez dalších informací, bez čtení abstraktu) musí člověk, který se v oboru aspoň letmo orientuje, přesně poznat, co se v práci nachází. Dokáže odhadnout, co je cílem práce. Ví, z jakých modulů se~celé řešení skládá a k čemu tyto slouží. Řekne, kolik a jakých experimentů řešitel provedl. Dokáže říct, kdo je cílovým \uv{zákazníkem} práce -- komu a k čemu je dobrá. Pokud to ze~samotného obsahu poznat není, nadpisy kapitol jsou asi špatně a buď je pisatel předělá, nebo má špatně napsanou práci. Třetí možnost není.

\section{Úvod}
\label{uvod}

První kapitola práce má vždy název Úvod. Slouží k zasazení řešené problematiky do širšího kontextu a v podobě stručného obsahu jednotlivých kapitol definuje strukturu písemné práce \cite{fitWeb}.

Rozsah úvodu práce by měl být cca 1-2 strany textu. Očekává se, že úvod bude čitelný i~pro \uv{náhodného kolemjdoucího}, který je gramotný, žije v~naší době a naší zemi a to je tak vše. Nemusí být odborníkem v dané technické oblasti. Přesto by měl úvodu porozumět. Je zapotřebí úvod napsat tak, aby byl samostatně čitelným literárním útvarem a pokud čtenář přečte jen úvod, měl by pochopit, o co v práci hlavně jde. Také se celkem často stává, že~lidé čtou jen ten úvod.

\subsection*{Rady pro úvod od prof. Zemčíka}
Není vhodné úvod dále členit na podkapitoly a není vhodné do něj uvádět odkazy na~literaturu -- měl by to být literární útvar dobře čitelný a \uv{stravitelný} a měl by se pěkně a~příjemně číst. Pokud se týká vnitřní struktury úvodu, je vhodné, aby byla následující:
\begin{itemize}
  \item{Cca 5 řádků obecný úvod do tématu pokud možno založený na obecně známých slovech nejlépe úplně bez odborných výrazů (tedy třeba \uv{počítač, video} ano, \uv{stromová struktura} ne),}
  \item{kousek textu o tom, proč je práce v dané oblasti důležitá, jaký má význam pro svět a~pro Vás, jaký má vztah ke studované oblasti, případně další důležité skutečnosti,}
  \item{kousek i o tom, co se v dané oblasti dělo v minulosti, případně \uv{v~jakém stavu je dnes}, případně jestli jsou nějaké vyhlídky do budoucna,}
  \item{je třeba napsat i o tom, \uv{proč se tomu vlastně chcete věnovat}, tedy co Vás vedlo k~výběru tématu a co na tom vidíte perspektivního -- pokud možno pravdivě a věrohodně,}
  \item{taky je třeba napsat, co je cílem práce (vlastními slovy, ne opsat formální zadání),}
  \item{nakonec je třeba popsat strukturu práce, aby čtenář věděl, kde co hledat (například \uv{v následující kapitole je uveden\ldots, v kapitole \uv{XXX} je popsáno, v kapitole 3 je\ldots} apod.) -- nezapomeňte, že úvod má být i návodem ke čtení práce. }
\end{itemize}

\begin{samepage}
\subsection*{Rady pro úvod od prof. Herouta}
Nemá se jednat o \uv{úvod do problematiky}, ale o~\uv{úvod do knížečky} (technické zprávy). Po jeho přečtení tedy má čtenář
\begin{enumerate}
  \item{mít představu o~čem knížečka bude,}
  \item{těšit se na to, že si ji přečte.}
\end{enumerate}
\end{samepage}

\subsection*{Rady pro úvod od doc. Černockého}

Úvod je vlastně \uv{rozšířené zadání}. Zahrnuje:

\begin{itemize}
  \item{Proč se na té práci dělalo -- co je Vaše motivace?}
  \item{Pozadí tohoto projektu -- je to součást nějakého výzkumného projektu? popsat. Nějaké průmyslové spolupráce? popsat.}
  \item{Pokud je BP/DP součást kolektivní práce, jasně popsat, kdo byl zodpovědný za co.}
  \item{Co jsou konkrétně věci, kterými byste se chtěli pochlubit (\uv{claims}).}
  \item{Kde se čtenář o~čem dočte. \texttt{\textbackslash subsection\{Obsah kapitol\}} atd.}
\end{itemize}

\subsection*{Rady pro úvod od dr. Berana}

Úvod by měl být stručný (1 stránka). Měl by obsahovat:
\begin{itemize}
  \item{úvod do problematiky (že se pohybujeme v~IT, např. zpracování obrazu a~ne výroba čipů),}
  \item{cíl práce -- jeden jasný cíl práce a k němu vedoucí kroky -- cíl je jeden, kroků k dosažení cíle (jako např. knihovna funkcí, vytvoření datasetu) je více,}
  \item{stručně obsah celé práce (nejdříve udělám přehled existujících řešení, z~nich vyjdu a~představím návrh svého řešení, otestuji, vyhodnotím, \ldots).}
\end{itemize}



\section{Shrnutí dosavadního stavu}
\label{stav}

Rozsah této části práce by měl být cca 40--50\,\% celkového rozsahu práce. Smyslem této části práce je seznámit čtenáře s~dosavadním stavem oblasti techniky, která je předmětem práce, a~seznámit ho s~aparátem použitým v~práci (matematickým, fyzikálním, elektronickým, IT apod.). Nelze předpokládat, že ve shrnutí dosavadního stavu bude popsáno úplně vše související s~prací, ale mělo by být uvedeno vše podstatné tak, aby čtenář, alespoň trochu znalý oblasti práce, byl schopen práci pochopit. Tuto část je dost často vhodné dělit na~více kapitol, zejména pokud se jedná~o \uv{mezioborové téma}. Tato část by se měla \uv{velmi intenzivně} odkazovat na literaturu. Rozsah závisí na druhu práce, pokud je práce spíše teoretická, je tato část podstatně delší než u prakticky zaměřené práce.

\subsection*{Rady pro shrnutí dosavadního stavu od prof. Zemčíka}

Je vhodné na začátku této části uvést, co obsahuje a proč a taky že \uv{není encyklopedickým přehledem} daného oboru -- to proto, aby někdo práci nemohl vyčíst, že \uv{v přehledu něco chybí}. Text je vhodné psát vlastními slovy, ale může v podstatě obsahově kopírovat citovanou literaturu. Pokud byste museli nutně převzít kompletní kusy textu v rozsahu cca více než jedné věty, je třeba tuto skutečnost řádně označit a ocitovat zdroj. Takových úseků by měl v práci být jen nezbytně nutný počet a maximální délka takového textu je (orientačně) 1/2 strany textu.

Ideální je se v této části práce zcela vyvarovat soudů ohledně technického obsahu -- dosavadní stav je třeba popsat, ale ne hodnotit. V podstatě je vhodné řídit se tím, že když někdo něco napsal v literatuře, můžete to napsat do dosavadního stavu. Důvodem je to, že~pokud dostane práci do ruky odborník v dané oblasti, měl by mít možnost tuto část práce přeskočit (je odborník, zná ji) a nepřijít přitom o nic podstatného ze samotné práce autora BP/DP. Je velmi vhodné tuto část práce sepisovat průběžně, dokud je přečtená literatura \uv{v~živé paměti} -- pak se nemusí číst znovu.


\subsection*{Rady pro shrnutí dosavadního stavu od prof. Herouta}

Kapitola popisující co bylo třeba vystudovat by měla zabírat asi 45\,\% rozsahu.

Záměrně nepoužívám duchaplně znějící slovo \uv{Teorie}, které by na tuto část diplomky v~zásadě sedělo. Je to proto, že slovo teorie má zvláštní magickou schopnost spouštět nutkání psát samoúčelné slohové cvičení na nejrůznější témata více nebo méně blízká tématu diplomové práce. I na témata poměrně dost vzdálená.

Nad každým odstavcem v této části diplomky doporučuji se ptát: \uv{Je tato informace potřebná k pochopení toho, co jsem vymýšlel a implementoval?} Není-li, bez milosti pryč s~ní!

Tato část textu diplomky se může často skládat ze dvou kapitol. Kdyby někdo v rámci diplomky vytvářel třeba webový účetnický systém, asi by měl jednu kapitolu o účetnictví a~jednu o bezpečných webových systémech. To je typický případ: u spousty IT řešení je potřeba studovat jednak obor činnosti, kde bude systém pomáhat, a jednak nástroje a~postupy vytváření příslušných systémů.

\subsection*{Rady pro shrnutí dosavadního stavu od dr. Szökeho}

V této kapitole (může jich být i víc) byste měli ukázat, \uv{že víte, která bije}. Máte prostudováno, jak se Váš problém obvykle řeší, a v čem je Vaše řešení jiné a lepší. Tato kapitola je dobrý zdroj položek do bibliografie. Měli byste ukázat, že jste něco nastudovali. Je dobré se~občas vynořit z matematiky a lidsky v jednom odstavci shrnout, co ty složité derivace vlastně znamenají. Je to užitečné, protože pokud se čtenář ztratí a přestane chápat, vrátíte ho do děje.

\subsection*{Rady pro teoretickou část od doc. Černockého}

Teoretická část by měla být na cca 10 stran, u teoretičtějších práci o něco více.
V této kapitole popisujete nezbytnou teorii pro Vaši práci.
\begin{itemize}
  \item{Pozor, opravdu pouze tu, kterou potřebujete, nechceme vidět opsaná skripta, knihy, Wikipedii \ldots}
  \item{Pokud uvádíte rovnice, je nutné v nich vysvětlit všechny symboly a musí být jasné, k~čemu je ta která rovnice ve Vaši práci použita. Nedávejte tam tedy rovnice \uv{jen tak pro ilustraci} nebo \uv{proto, aby to vypadalo vědečtěji} nebo \uv{abych si zkusil, jak se to v \LaTeX u sází \ldots}}
  \item{Pokud je nějaká teorie hodně těžká (například HMM a spol.), nepopisujte ji celou, ale dejte úplně základy, citujte vhodný zdroj a \uv{vypíchněte} jen to, co je ve Vaší práci opravdu důležité.}
\end{itemize}


\subsection*{Rady pro teoretickou část od dr. Berana}
Teorie by měla obsahovat:
\begin{itemize}
  \item{existující řešení z pohledu Vašeho zadání,}
  \item{co již existuje v oblasti mé práce, jaká jiná řešení mého zadání existují,}
  \item{co existuje za postupy/nástroje, které mohu využít k řešení,}
  \item{veškerá teorie by měla být \bf zdůvodněna \rm (uvedeno, proč je s~ní čtenář seznámen a~jak souvisí s řešením práce).}
\end{itemize}

\section{Data}

Pro projekty zabývající se rozpoznáváním či zpracováním přirozeného jazyka jsou data naprosto klíčová a~tato kapitola by jim neměla chybět. Doporučený rozsah je v~jednotkách stránek. Popište:
\begin{itemize}
  \item{kde se data vzala (producent, katalogové číslo atd.),}
  \item{technický popis -- např. Fs, bitová šířka, počet mluvčích, délka audia atd.,}
  \item{dělení na sub-sety -- trénování, vývoj, testování/vyhodnocení -- a~kdo ho dělal (nejlépe použít již existující dělení).}
\end{itemize}

Tato kapitola ve Vaší diplomce asi nebude, pokud např. děláte na zpracování hudebních zvuků pro real-time hraní.

Tato kapitola tedy bude především u~prací, které pracují s~datovými sadami, se kterými se běžně pracuje na různých pracovištích a umožňují srovnání výsledků, či u~prací, k~nimž byla školou či třetí stranou poskytnuta určitá datová sada pro testování.


\section{Zhodnocení současného stavu a~plán práce (návrh)}
\label{navrh}

Hlavním cílem této části práce je popsat zhodnocení současného stavu a návrh Vašeho inovativního řešení.

\subsection*{Rady pro zhodnocení současného stavu a~plán práce od~prof. Zemčíka}

Tuto část práce je velmi vhodné uvést. Její rozsah je \uv{podle potřeby}. Smyslem je na~základě zhodnocení současného stavu určit cíl práce a vytvořit tak vlastně \uv{detailní zadání}, případně určit předpokládané parametry řešení, ale ne jeho způsob. Osnova této části může být například:
\begin{itemize}
  \item{Kritické zhodnocení dosavadního stavu (co je dobře, co je špatně, co případně není řešeno vůbec a~případně cenové parametry, dostupnost řešení, potřebný výpočetní výkon apod.), }
  \item{návrh, co by bylo vhodné vyřešit na základě znalostí dosavadního stavu a~také osobních preferencí, zadání práce, požadavků z praxe apod.,}
  \item{podle možností i~specifikace práce ve smyslu \uv{co to má dělat}, \uv{jaké to má mít parametry}, \uv{jaké prostředky budou použity}, \uv{jak se to bude vyhodnocovat}, \uv{jak se pozná, že se to povedlo}.}
\end{itemize}
Je vhodné do této části práce napsat pravdivou úvahu, zejména v~bodě \uv{návrh} tak, aby zbytek práce byl uvěřitelný a~aby po přečtení této kapitoly a~poté zbytku textu byl čtenář přesvědčen, že bylo uděláno, co se udělat mělo.


\subsection*{Rady pro návrh řešení od prof. Herouta}

Tato část obsahuje nové myšlenky, které práce přináší:
\begin{itemize}
  \item{Rozhodl jsem se.}
  \item{Vymyslel jsem.}
  \item{Rozvrhl jsem.}
  \item{Vypočítal jsem.}
  \item{Odvodil jsem.}
  \item{Zjednodušil jsem.}
  \item{Vylepšil jsem.}
  \item{Navrhl jsem.}
  \item{Zjistil jsem.}
  \item{Vyzkoumal jsem.}
\end{itemize}

Někdy je těžké od sebe oddělovat nové myšlenky a implementaci. Programátorovi se míchá \uv{navrhl} s~\uv{naprogramoval}. Míchá se: \uv{vylepšil jsem} s \uv{výsledky jsou}. Ale je správně od sebe tyto kapitoly oddělit.

V mnoha ne-IT oborech se diplomky strukturují jako výzkumné práce podle tzv. vědecké metody. V~našem prostředí (řekněme inženýrské studium IT) se struktura diplomky nedrží přesně tohoto schématu. Naše diplomky (a~myslím, že to není úplně špatně) se vedle výzkumných prací podobají projektové dokumentaci. Pořád je ale správná cesta oddělit formulaci hypotéz od návrhu, jak je ověřit, od jejich samotného ověření/vyhodnocení.

\subsection*{Rady pro návrh a popis Vašeho algoritmu od~doc. Černockého}

Pokud je úkolem práce \uv{věda}, bude tato kapitola asi nejobsažnější a bude vhodné ji rozdělit -- třeba na \texttt{\textbackslash chapter\{}Základ\texttt{\}}, \texttt{\textbackslash chapter\{}Zlepšení, zpřesnění, \ldots\texttt{\}} a \texttt{\textbackslash chapter\{}Výsledky a~diskuse\texttt{\}}. Pokud je naopak úkolem spíše vyzkoušet něco existujícího/nového, může být táto kapitola docela stručná. Rozsah této části by měl být asi 10 stran. Obsahuje:
\begin{itemize}
  \item{co jste konkrétně udělal s teorií popsanou výše -- blokové schéma, nastavení konstant, technické \uv{přitesání} složité teorie (zjednodušování atd.),}
  \item{návrh -- může být jednoduché blokové schéma nebo plný objektový návrh, ale mělo by být jasné, že má Váš SW nějakou strukturu,}
  \item{volbu použitého OS, programovacího jazyka a knihoven. V BP/DP není úkolem napsat všechno sám, můžete použít jakékoliv volné i komerční podpůrné programy, knihovny, moduly atd. atd. -- prostě cokoliv -- je to standardní inženýrská práce -- cílem je ji \textbf{udělat}, ne \textbf{napsat všechno sám}. Musíte to v práci pořádně popsat, ne vydávat dílo jiných za vlastní! U knihoven je dobrá přesná specifikace -- odkud, která verze, pokud ji bylo potřeba platit, tak kolik to stálo.}
\end{itemize}

\subsection*{Rady pro návrh řešení od dr. Berana}

U návrhu řešení záleží na typu zadání -- následující body nejsou obecné. Rozsah by měl být asi 1/3 stran.
\begin{itemize}
  \item{Pište z pohledu velmi dobře placeného experta na myšlenky, který navrhuje řešení problému, inovativní řešení, řešení zajímavých myšlenek.}
  \item{Část \uv{návrh} se dá vnímat jako samostatný popis/návod k tomu, jak problém vyřešit, tento návrh se pak předá týmu programátorů a~testerů, který návrh zrealizuje a~otestuje.}
  \item{Je-li to možné, návrh by měl být obecného charakteru, bez ohledu na to, bude-li realizován na iOS nebo Androidu, Linuxu či Windows, MySQL nebo Postgres, HMTL5 nebo Flash.}
  \item{Detailní rozbor zadání práce, detailní specifikace a formulace cíle a jeho částí.}
  \item{Popis použití řešení, situace/problémy, které projekt řeší.}
  \item{Postup práce/kroky vedoucí k cíli, rozdělení celku na podčásti.}
  \item{Návrh celého řešení i jeho částí, s odkazy na teoretickou část.}
  \item{Analýza (mezi)výsledků (měření, pozorování, pilotní testy).}
  \item{Vývoj a aktualizace návrhů.}
  \item{Aktualizujete-li návrh podle průběžných testů -- uvádět odkazy na testy a výsledky.}
\end{itemize}

\section{Uživatelské rozhraní}

Tato kapitola se hodí pouze do některých diplomek a měla by mít rozsah pár stran. U~některých o ně vůbec nejde. Pokud o ně jde a je zásadní, měla by být uvedena (může být i~před návrhem) a obsahovat \cite{Cernocky}:
\begin{itemize}
  \item{koncepci UI -- asi jste se něčím inspiroval(a) (existující programy, klasické mechanické zařízení\ldots) -- napište o tom,}
  \item{mockup -- pokud si kreslíte nějaké obrázky ručně, dejte je sem!}
  \item{jak proběhl výběr konečné varianty,}
  \item{pokud UI prodělalo nějaký vývoj, např. jste nebyl spokojený s 1. verzi a na základě uživatelů to úplně předělal, napsat.}
\end{itemize}


\section{Implementace}
\label{implementace}

Tato část práce by se měla zaměřit na vlastní popis práce. Mělo by zde být popsáno, \uv{co se vlastně udělalo}, a tato část by měla společně s testováním tvořit cca 40\,\% celkového rozsahu práce. Z~textu by mělo být zřejmé, co tvoří podstatu vlastní práce uchazeče o titul, jak bylo dílo vytvořeno, jakými prostředky a s jakými výsledky.

\subsection*{Rady pro popis vlastní práce od prof. Zemčíka}

Při psaní této části práce je třeba se zejména vyvarovat technických detailů, které by mohly čtenáře odradit či unudit. Důležité je uvést celkovou koncepci práce, podstatné rysy řešení, případně co k těm podstatným rysům vedlo. Dost podstatné taky je, že tato část práce musí popsat, jak se dílo využívá, ale nemůže být návodem k použití. Pozor, veškeré technické detaily, které nejsou podstatné k pochopení podstaty práce (a narušovaly by tak \uv{tok textu} této části práce), patří do příloh a ne do samotného textu, týká se to zejména výpisů dlouhých kusů zdrojového textu, návodů, dlouhých tabulek výsledků atd. Pokud v této části práce uvádíte výpisy zdrojových textů, tak jen úryvky a v nezbytné míře a graficky dobře oddělené od textu. Typickou vadou těchto částí prací bývá to, že jsou pro čtenáře \uv{nestravitelné} díky velikému počtu a~hloubce řešených detailů a také kvůli popisům věcí, které se lidem nedařily a pak se~to \uv{nedaření se} snaží reflektovat v textu (většinou tak, aby bylo jasné, jak se nadřeli a~jak se to nakonec povedlo), ale čtenáře to obvykle vůbec nezajímá. Vítány (a čtenářsky vděčné) jsou naopak obrázky, fotografie, případně i \uv{screenshoty}. Tato část práce může mít jednu nebo více kapitol. Více kapitol je vhodných zejména pokud realizace byla složena ze dvou nebo více tématicky odlišných celků (například server programovaný v C++ a klient v HTML nebo tak \ldots). Osnova v~tomto případě je silně individuální, ale přesto se dají identifikovat základní části, které by asi měly být v níže uvedeném pořadí.
\bigskip

\begin{samepage}
\noindent Typická osnova:
\begin{itemize}
  \item{Popis základní koncepce díla,}
  \item{popis fungování díla jako celku (co na co je), podle potřeby více či méně detailní popis fungování jednotlivých částí řešení (ale pozor, není potřeba na všechny dávat stejný důraz -- důraz je třeba dát na neobvyklé či náročné části, \uv{rutinní} části je možné a~taky vhodné zredukovat na minimum),}
  \item{způsob a velmi vhodně také příklad použití díla, vhodné jsou \uv{case study} přístupy, \uv{screenshoty}, postupy (pozor, nevhodné jsou návody).}
\end{itemize}
\end{samepage}

\subsection*{Rady pro popis vlastní práce od dr. Szökeho}
V této kapitole (může jich být víc) popisujete váš problém z implementačního hlediska. Jaké prostředí jste zvolili, jaké knihovny, návrh tříd, komunikace, protokol atd. Nezacházejte do~detailů. Čtenáře nezajímá jak se~implementuje tlačítko. Spíš ho bude zajímat, jak jste implementovali umělou inteligenci, komunikaci nebo zajímavou funkci. \bf Ze všeho by mělo být patrné, proč to tak děláte. \rm  Tuto kapitolu nemusí chápat Vaše babička, ale je dobré mít několik úrovní. IT znalému by mělo být jasné co a proč implementujete, zkušený programátor by měl pochopit i detaily (jak přesně to implementujete).

\subsection*{Rady pro implementaci od doc. Černockého}

Pokud je úkolem práce udělat \uv{produkční} SW, tak jej zde popisujete. Tato část obsahuje:
\begin{itemize}
  \item{komentáře k implementaci -- např. seznam jednotlivých tříd a co dělají, nemusíte detailně popisovat rutinní věci (vstupní parametry z příkazové řádky), ale zaměřte se na klíčové funkce. Nedávejte sem plné zdrojové kódy, ty jsou na CD. Naopak, pokud je pár řádků zdrojového kódu životně důležitých, je dobré je sem vykopírovat a říci, co dělají,}
  \item{pokud má Váš program v reálném čase komunikovat s vnějším světem, napište o~časování, možných konfliktech a zda jsou řešeny. V BP/DP se neočekává, že uděláte 100\,\% \uv{blbuvzdorný} SW, ale o možných problémech byste měli vědět,}
  \item{pokud tady proběhla třeba implementace v Matlabu, popište,}
  \item{výsledky, co to dalo -- nejlépe srovnání s nějakými již existujícími, které na podobném setu dostal někdo jiný,}
  \item{pokud bylo úkolem \uv{odpíchnout se} od něčeho existujícího a prozkoušet něco nového, tady je prostor na to to popsat a řádně prodiskutovat.}
\end{itemize}

\subsection*{Rady pro popis implementace od dr. Berana}
Tuto kapitolu je vhodné oddělit od testování, zejména je-li práce implementačního charakteru. Doporučený rozsah je asi 1/3 stran.

Pište z~pohledu špatně placeného programátora, který dostal od nadřízeného specifikaci prototypu (Váš návrh řešení) a má za úkol ho implementovat. Kapitola zahrnuje:
\begin{itemize}
  \item{specifika cílové platformy a technologií,}
  \item{nástroje použité k realizaci prototypu řešení.}
\end{itemize}


\section{Testování}
\label{testovani}

Z~textu této kapitoly by mělo být zřejmé, jak se ověřila správnost/funkčnost vytvořeného díla, jestli matematicky, experimentem, na uživatelích nějakou studií apod. Také, jaké mělo ověřování výsledky. Rozsah by měl být cca 10 stran.

Tato část práce může mít jednu nebo více kapitol. Více kapitol je vhodných zejména pokud proběhlo rozsáhlé vyhodnocování, pokud se práce nějak nasadila v praxi (může do extra kapitoly) apod. Osnova v tomto případě je silně individuální, ale přesto se dají identifikovat základní části \cite{Zemcik}:
\begin{itemize}
  \item{metodika a výsledky ověřování díla, které mohou zahrnovat matematické důkazy, postupy testování, postupy ověřování na lidech,}
  \item{interpretace výsledků a možnosti nasazení v praxi (včetně toho, co by se třeba ještě mělo dodělat).}
\end{itemize}

\subsection*{Rady pro obsah kapitoly o testování od doc. Černockého}

Tato kapitola je hodně variabilní -- použijte jen body, které se hodí.
\begin{itemize}
  \item{Testování na off-line datech -- dalo to stejné/lepší výsledky než publikované? Pokud ne, proč? Horší výsledky nemusí nutně znamenat, že jde o špatnou diplomku (mohl(a) jste mít méně dat na trénování, horší algoritmus udělatelný za 1 semestr atd.), ale měl(a) byste vědět, proč.}
  \item{Testování na off-line datech -- dala implementace v C/C++ to stejné, co původní zápis algoritmu v Matlabu?}
  \item{Jak je to celé HW náročné -- CPU, paměť, chování při paralelizaci, chování při použití GP-GPU atd.}
\end{itemize}

\subsection*{Rady pro experimentování od dr. Berana}
Tato kapitola obsahuje experimenty a vyhodnocení -- není-li práce implementačního charakteru, může být v jedné kapitole s implementací.

Pište z pohledu špatně placeného testera, který dostal od nadřízeného specifikaci prototypu (Váš návrh řešení) a implementaci a má za úkol ji otestovat. Kapitola zahrnuje:
\begin{itemize}
  \item{specifika využité platformy,}
  \item{data použitá k experimentování s prototypem řešení -- popis dat, zdroje, podmínky získávání dat, charakter dat,}
  \item{anotace dat -- formát anotací, zdroj anotací, využití anotací,}
  \item{popis měření, popis a podmínky experimentů/testů,}
  \item{naměřená data,}
  \item{výsledky -- diskuse a interpretace naměřených dat.}
\end{itemize}

\subsection*{Uživatelské testování}

Opět je relevantní pouze někde, ale pro některé práce je zásadní.
\begin{itemize}
  \item{Výběr subjektů na testování (naivní, zkušení).}
  \item{\uv{Testovací protokol} -- co vlastně testovali, na co jste se jich pak ptal(a) -- otázky, hodnocení, \ldots}
  \item{Výsledky testování -- odpovědi po subjektech a pak sumární.}
  \item{Závěry -- Jsou uživatelé spokojeni? S čím ano, s čím ne? je to dobré? Je to špatně? Dá se něco zlepšit? Zlepšil(a) jste to ještě během diplomky nebo je to na budoucí práci?}
\end{itemize}

\subsection*{Rady pro experimentování a testování od dr. Szökeho}
V této kapitole byste měli podrobit Váš výsledek kritickému názoru. Nejen Vašemu, ale i~nezávislých uživatelů. Podstatné je, abyste z posbíraných zpětných vazeb učinili relevantní závěr. Například vylepšit GUI, zrychlit některé části kódu nebo zvolit úplně jiný přístup. Pokud máte čas, můžete v rámci diplomky zkusit udělat ještě jednu iteraci a zapracovat na~těch největších nedostatcích.

\section{Závěr}
\label{zaverPrace}

Závěrečná kapitola -- Závěr obsahuje zhodnocení dosažených výsledků se zvlášť vyznačeným vlastním přínosem studenta. Povinně se zde objeví i zhodnocení z pohledu dalšího vývoje projektu, student uvede náměty vycházející ze zkušeností s řešeným projektem a uvede rovněž návaznosti na právě dokončené projekty (řešené v rámci ostatních bakalářských prací v~daném roce nebo na projekty řešené na externích pracovištích).

Závěr práce by měl obsahovat shrnující fakta o práci a čtenáři dát (i bez čtení jiných částí práce) informaci o tom, co bylo předmětem práce a jak se práce vydařila. Závěr by měl obsahovat i osobní dojmy z práce a nejlépe i shrnutí možností dalšího pokračování práce. Délka závěru by neměla přesáhnout jednu stranu textu. Do závěru se nehodí odkazy do~textu práce či literatury.

V závěru ať nepřicházejí žádné nové poznatky, neobjeví se tam nové číslo nebo nový graf.

\subsection*{Rady pro Závěr od prof. Zemčíka}

Velmi doporučuji následující osnovu:
\begin{itemize}
  \item{Jednou větou shrnutý záměr práce (například \uv{Cílem této práce bylo \ldots}),}
  \item{konstatování, že záměr byl splněn (nejlépe bez zbytečných sebekritických poznámek, ty případně nechte na oponenta),}
  \item{přehled splnění jednotlivých bodů formálního zadání práce, a to buď přímo jako \uv{otevřená reakce na zadání} nebo \uv{skrytá reakce}; v každém případě uveďte (třeba jako návod pro sebe jak odpovídat) jednu větu shrnující možnou odpověď na otázku \uv{Jak jste splnili bod X zadání?},}
  \item{vhodné kvalitativní a kvantitativní shrnutí práce, například obsahující 3--5 nejvýstižnějších číselných údajů o práci (v rozsahu cca 5--10 řádků textu),}
  \item{nějaký pěkný postřeh k práci (\uv{Práce mi dala \ldots naučila \ldots}), }
  \item{výhled do budoucna, nejlépe i rozdělený do částí \uv{V práci bych chtěl pokračovat tak, že \ldots} a \uv{V práci by ještě někdo mohl pokračovat tak, že \ldots} -- do částí rozdělených podle toho, co byste případně ještě chtěli zkusit a co by se ještě taky dalo, ale do čeho asi \uv{nepůjdete}. }
\end{itemize}

Prosím, při psaní závěru si uvědomte, že to je část, kterou si asi přečte z práce nejvíce lidí. Pokud do práce bude v budoucnu někdo nahlížet, přečte si buď jen úvod, nebo jen závěr, případně úvod a hned potom závěr, úvod, popis vlastní práce a závěr, ale jen velmi výjimečně kompletně celou práci. Každý z výše uvedených \uv{průchodů} by měl být pro čtenáře \uv{stravitelný}.


\subsection*{Rady pro Závěr od prof. Herouta}
Funkce závěru:

\begin{itemize}
  \item{Autor se ohlíží za tím, co udělal: \uv{V práci je. Hlavní úspěchy jsou. Důležitými výsledky jsou. Podařilo se.}}
  \item{Autor uvede nápady, které nestihl realizovat v podobě možností pokračování: \uv{Ještě by bylo možné zkusit. Kdybych byl na začátku věděl, co vím teď, dělal bych.}}
  \item{Autor (ve vlastním zájmu) rekapituluje, jak bylo naplněno zadání práce.}
\end{itemize}

\begin{samepage}
\noindent K tomu dvě poznámky.
\begin{itemize}
  \item{Za prvé: \uv{diskutovat možnosti pokračování} zní jednoduše a bezpečně. Nebral bych ale toto krátké sdělení na lehkou váhu. Obecné texty typu \uv{do budoucna by bylo dobré to zrychlit a zpřesnit} ukazují, že se pisatel neobtěžuje skutečně přemýšlet nad svou prací a že nemá nápady. U některých jiných \uv{možností pokračování} zase oponenta právem napadne, že to přece pisatel měl dělat už podle zadání, takže body dolů. Opravdu bych nebral toto krátké sdělení na lehkou váhu.}
  \item{Za druhé: Závěr čte oponent úplně na závěr. Chvíli před tím, než se pustí do sepisování oponentského posudku. Závěr ho tedy tak říkajíc ladí na to, jak má posudek napsat. Je dobré ho ladit na notu: \uv{Jsem dobrý student, který naplnil celé zadání a udělal kus zajímavé práce.} To je třeba nepřehnat do polohy: \uv{Jsem špatný student, který neumí přemýšlet a programovat, ale umí hlasitě vykřikovat, že je nejlepší.}}
\end{itemize}
\end{samepage}

\subsection*{Rady pro Závěr od doc. Černockého}

Závěr obsahuje:
 \begin{itemize}
   \item{shrnutí práce -- udělal jsem to a to, toto nefungovalo, toto fungovalo, toto jsem nestihl udělat, protože \ldots}
   \item{Budoucí práce
     \begin{itemize}
       \item{\uv{krátkodobá} -- co je Vám naprosto jasné, že byste sám/sama nebo s pomocí 1--2~lidí dokázal(a) v horizontu několika týdnů -- měsíců udělat a pomohlo by to,}
       \item{\uv{dlouhodobá} alias \uv{vzdušné zámky} -- je na Vás, jak moc popustíte fantazií na~procházku.}
     \end{itemize}}
\end{itemize}


\subsection*{Rady pro Závěr od dr. Berana}
Závěr by měl mít rozsah 1 stránka a obsahuje:
    \begin{itemize}
      \item{co bylo cílem práce,}
      \item{jak se postupovalo při řešení (viz \uv{stručně obsah celé práce} v úvodu),}
      \item{co se podařilo vytvořit,}
      \item{hodnocení řešení podložené výsledky,}
      \item{další možnosti řešení, výhled do budoucna,}
      \item{nedoporučuji sebehodnocení a věty typu \uv{vybral jsem si zadání, abych se naučil programovat, a cíle se podařilo splnit}.}
    \end{itemize}

\section{Přílohy}

Do příloh umístíme ty části práce, které mají výrazně popisný charakter (například příručka pro použití vytvořeného systému, fragmenty zdrojového textu, detailní schémata a detailní popisy řešených částí projektu a podobně) \cite{fitWeb}. Všechny přílohy musí být očíslovány, přičemž se využívá samostatný styl číslování (písmena A--Z). Je-li příloh více, jejich seznam se uvádí na konci práce za seznamem použité literatury.

I když počet stránek příloh není omezen, je nutné dodržet stručnost a účelnost a přihlédnout k významu přílohy pro hodnocení práce a pro případné navazující práce v budoucnu. Zbytečně velký objem příloh bez řádného odůvodnění může být negativně hodnocen přinejmenším z důvodů ekologických a ekonomických. Více viz \cite{fitWeb}.

Přílohy jsou vhodným místem pro uvedení návodů, detailních popisů navržených protokolů a formátů, delších tabulek, větších obrázků (mohou být ve formě skládanky) a dalších prvků, které by narušovaly \uv{tok textu} práce. Rozsah příloh se nepočítá do rozsahu práce. Počet stran příloh by neměl být příliš velký -- je důležité, aby všechny přílohy v papírové podobě byly účelné.

\subsection*{Rady k obsahu příloh od doc. Černockého}

\paragraph{Příloha 1 -- Kuchařka}

Návod pro někoho, kdo bude chtít Vaši práci zopakovat, o rozsahu jednotek stran
\begin{itemize}
  \item{co je potřeba odkud stáhnout, zkompilovat, jak hacknout OS, aby mi věci běžely \ldots}
  \item{adresáře, skripty, co a v jakém pořadí pouštět, kam se dívat na výsledky,}
  \item{atd. atd. atd.}
\end{itemize}

\paragraph{Další přílohy}

Cokoliv, co by příliš zatěžovalo hlavní text práce -- například dvoustránkové odvození něčeho, třístránková tabulka s popisem nějakého API atd. \ldots

Přiložené datové médium by mělo obsahovat:
\begin{itemize}
  \item{všechny kódy -- Matlab, C, \LaTeX{} atd. atd.,}
  \item{všechny natrénované parametry modelu -- HMM, neuronové sítě, transformace, prostě všechno,}
  \item{všechno, co je potřeba ke spuštění Vaší práce -- externí knihovny, moduly atd.,}
  \item{všechna data -- pokud nejsou omezena licenčními podmínkami,}
  \item{detailní výsledky -- do BP/DP dáváte souhrnné tabulky, ale můžete mít třeba několik MB tabulek automaticky vygenerovaných -- dejte je sem.}
\end{itemize}

\noindent Dr. Beran doporučuje na datovém médiu přiložit:
\begin{itemize}
  \item{všechny použité knihovny (nejlépe přeložené) i zdrojové texty s popisem překladu,}
  \item{Vaše výstupní aplikace (včetně binárních souborů), tj. spustitelné řešení přímo z datového média,}
  \item{video -- jedno nějaké reprezentativní, a pak klidně záznam z běhu výsledného řešení.}
\end{itemize}

\subsection*{Plakát}
\begin{itemize}
  \item{plakát uložte na CD i vytiskněte (nejlépe PDF),}
  \item{velikost tisku A2,}
  \item{tisknout lze v knihovně nebo v komerční tiskárně,}
  \item{obsah plakátu rozumně vyvážen:
  \begin{itemize}
    \item{především poutavou a jasnou formou (co bylo vytvořeno, co to umí, k čemu to je, jak je to super),}
    \item{trošku technického popisu (použité postupy a metody).}
  \end{itemize}}
  \item{Plakát by měl určitě obsahovat:
  	\begin{itemize}
      \item jméno a příjmení studenta,
      \item e-mail studenta,
      \item jméno a příjmení vedoucího a
      \item akademický rok.
  	\end{itemize}
    }
\end{itemize}


\chapter{Pravidla pro bibliografické citace}
\label{citace}

Tato pravidla jsou převzata ze stránek fakulty \cite{citace}.

\section{Definice pojmů}

\begin{itemize}
  \item{\bf Bibliografická citace \rm je souhrn údajů o citované publikaci nebo její části umožňující její identifikaci a vyhledávání.}
  \item{\bf Odkaz na bibliografickou citaci \rm je odvolání se v textu na bibliografickou citaci uvedenou na jiném místě práce.}
\end{itemize}

\section{Přebírání cizích textů}

Při přebírání cizího textu je nutné přebíraný text řádně odlišit od vlastního textu. V opačném případě autor vydává cizí text za svůj vlastní, což je u závěrečných prací těžko tolerovatelný prohřešek. Přebíraný text lze do vlastní práce začlenit dvěma způsoby:

\begin{itemize}
  \item{\bf Doslovné přejetí \rm -- text je přejat ve stejném znění, jak byl uveden v původním zdroji. Krátké přejaté texty jsou vysázeny v uvozovkách, delší v oboustranně odsazeném odstavci. Často se doslovně přejatý text navíc sází kurzívou.}
  \item{\bf Parafráze původního textu \rm -- původní text je přeformulován autorem práce při zachování jeho smyslu. Parafráze je vysázena jako běžný text práce.  K odlišení toho, která část textu je převzata a která je vlastní, proto autor musí použít jiné prostředky (např. uvedením odkazu na citaci na konci věty popř. odstavce autor říká, že věta popř. celý odstavec byl přejat, nebo je-li parafráze rozsáhlejší, lze na začátek podkapitoly uvést např. \uv{Tato podkapitola byla převzata z [1].}}
\end{itemize}

Doslovné přejímání textů se hodí například pro uvedení definic, částí zákonů, předpisů či norem nebo pro polemiku s názory jiných
autorů. Doslovné přejímání není proto v technických pracích příliš časté. Pokud nejsou jednoznačné důvody pro doslovné přejetí textu,
použijte parafrázi. Dobře také zvažte rozsah přebíraného textu. Příliš mnoho přebraného textu svědčí o nízké kvalitě závěrečné
práce (obvykle dokončované na poslední chvíli, kdy přebíraný text slouží pouze pro dosažení minimálního požadovaného rozsahu).

\section{Základní principy citování}

Bibliografické citace uvádíme proto, aby bylo čtenáři jasné, z jakých prací vycházíme, abychom uvedli čtenáře do širších souvislostí (například aby si méně znalý čtenář, než pro
kterého jsme práci psali, mohl dostudovat látku, kterou jsme v práci nevysvětlili), a také proto, abychom splnili podmínky autorského zákona (viz paragraf 31). Při bibliografických
citacích dodržujte následující pravidla:

\begin{itemize}
  \item{\bf Citujte všechna díla, ze kterých jste čerpali! \rm Pokud některé dílo neuvedete, vydáváte cizí práci za svou vlastní! Již od počátku práce na projektu je vhodné si zaznamenávat všechny zdroje, se kterými jste přišli do styku. Na konci práce je vyhledání všech zdrojů pracnější.}
  \item{\bf Citujte pouze díla, která jste skutečně použili! \rm Čtenář, který citovaná díla opravdu zná (nebo si je vyhledá), brzy odhalí, že pořádně nevíte, co v nich vlastně je, přestože je citujete.}
  \item{\bf Citujte pouze z primárních pramenů! \rm To znamená, citujte pouze díla, která jste měli \uv{v ruce} (nebo na obrazovce). Jinak hrozí, že citaci převezmete s chybami.}
  \item{\bf Uvádějte citace přesně! \rm Čtenáři tím usnadníte vyhledání původního zdroje a vyhnete se podezření, že nepřesnou citací chcete identifikaci původního díla ztížit nebo zcela znemožnit, aby čtenář nemohl ověřit rozsah přebíraného textu.}
\end{itemize}

Při psaní technické zprávy věnujte těmto citačním principům zvláštní pozornost, protože jejich porušení mívá z pochopitelných důvodů mnohem vážnější následky na celkové hodnocení práce než nedodržení formálních požadavků na bibliografické citace.

\section{Citační normy}

Pravidla pro vytváření bibliografických citací a jejich uvádění v odborných publikacích upravuje norma ČSN ISO 690: Informace a dokumentace - Pravidla pro bibliografické odkazy a citace informačních zdrojů z roku 2011. Norma je dostupná v knihovně FIT. Nástroj pro generování citací podle těchto norem
najdete na \url{http://www.citace.com/} a~stručný výtah z normy s příklady citací najdete na stejných stránkách \cite{biblio}.

Přestože norma připouští umístit citace do odborných publikací různým způsobem (na~konec textu, na konce jednotlivých kapitol, přímo do textu, do poznámky pod čarou, částečně do textu a částečně do poznámky pod čarou), u závěrečných prací na FIT VUT v~Brně je požadováno, aby soupis citací byl uveden na konci práce.

Odkaz na citaci bude ve tvaru pořadového čísla, pod nímž je uvedena citace v seznamu literatury na konci textu. Pořadové číslo uvedeme v textu v hranatých závorkách.

\noindent \textbf{Příklad}: \textit{Protokol SMTP je definován dokumentem RFC 5321 [1].}

Citace jsou seřazeny podle abecedy. Řazení jmen s písmeny s diakritikou můžeme ovlivnit prvkem \texttt{key}, jehož  hodnotu nastavíme na příjmení bez diakritiky. Pokud není vyplněn autor, citace se řadí na začátek seznamu, což není vhodné. Řazení v tomto případě můžeme taktéž ovlivnit vhodně nastaveným prvkem key.

\newpage
\textbf{Příklad}:
\begin{verbatim}
   @Article{Cech:2020:Citace,
	   author               = "Čech, Jan",
	   key                  = "Cech",
	   ...
\end{verbatim}

Citujeme-li tutéž publikaci vícekrát bezprostředně za sebou, můžeme místo opakování celé citace použít výrazu \uv{tamtéž}.

Příklady citací naleznete v příloze \ref{priloha-priklady-citaci}.

\section{Využití elektronických zdrojů}

Výběr kvalitních pramenů, ze kterých při práci vycházíte, je pro vlastní práci velice důležitý. U klasických tištěných materiálů se dá kvalita díla poznat poměrně dobře. U elektronických zdrojů ale zvláště důkladně zkoumejte jejich věrohodnost a kvalitu. Ověřujte si proto, kdo je autorem daného materiálu, a kriticky hodnoťte hloubku předložených informací (řada elektronických zdrojů předkládá jen povrchní a neúplné informace a chyby mezi sebou autoři často přejímají). I když se vám podaří najít dobré elektronické zdroje, nespoléhejte se pouze na ně, ale snažte se využít i tištěnou literaturu.

Při citaci elektronických zdrojů vždy uveďte, kdy byla informace převzata, protože za několik dní už může být celý web nedostupný. Při použití nástroje BibTeX toho můžete docílit přidáním klíče \verb|cited = "yyyy-mm-dd"|. K citování elektronických zdrojů využívejte \verb|@website| (celá doména) nebo \verb|@webpage| (jedna stránka v rámci domény). Pokud v rámci práce citujete článek z časopisu nebo konference, necitujte jej jako elektronický zdroj, ale jako \verb|@article|.

\section{Evergreen: citování webu vs. papíru}

Tato podkapitola byla převzata z blogu prof. Herouta \cite{Herout}.

Ano, dnes je všechno na webu. Bohu díky. Web ale nemá žádný archiv. Za minutu bude vypadat jinak, než vypadá teď. Když jsou odkazy na \uv{literaturu} samá URL, je tato část diplomky jaksi živá, neuchopitelná, fluidní -- odkazy směřují do prostředí, které je každou vteřinu jiné. Ideálně by i tato část diplomky (jako její ostatní části) měla být pevná, konstantní -- samé odkazy na papír.

Je slaboduché citovat webový zdroj, když se jedná o uložené .pdf časopisového nebo konferenčního článku. Nedělejte to ve svých diplomkách.
\bigskip

\noindent Neuvádějte v seznamu literatury

\noindent \it BAY, H., ESS, A., TUYTELAARS, T. a GOOL, L. V.: Speeded-Up Robust Features (SURF) [online]. 2008 [cit. 2010-07-13]. Dostupné z: \url{https://www.vision.ee.ethz.ch/en/publications/papers/articles/eth_biwi_00517.pdf}
\bigskip
\rm

\noindent když můžete citovat

\noindent \it BAY, H., ESS, A., TUYTELAARS, T. a GOOL, L. V.: SURF: Speeded Up Robust Features, Computer Vision and Image Understanding (CVIU), sv. 110, č. 3, str. 346–359, 2008.
\bigskip
\rm

Ten web už za měsíc možná nepojede, zato konferenční sborník půjde dohledat přinejmenším do samého konce této civilizace.

\section{Sazba citací}

V šabloně pro závěrečné práce FIT VUT je pro sazbu citací použit systém BibTeX.

\subsection*{Údaje v bibliografických citacích}

Následující je převzato z normy ČSN ISO 690 a zkráceno. Jednotlivé části citace jsou odděleny znakem . (tečka). Každá položka, která se objevuje v citovaném dokumentu, by měla být uvedena tak, jak se v citovaném dokumentu nachází. Obvyklé pořadí údajů v~bibliografické citaci je následující:

\begin{enumerate}
    \item \textbf{Jméno případě jména tvůrců/autorů, pokud jsou k dispozici}
    \begin{itemize}
        \item Osoby nebo organizace zodpovědné za vytvoření obsahu by měly být uvedeny jako tvůrci.
        \item Křestní jména nebo další části jména by měla být uvedena po příjmení, například \texttt{GORDON, D}.
        \item Pokud má citovaný zdroj více tvůrců, vyjmenováváme je (oddělovač \uv{and}), pokud jsou autoři více než tři, neuvádíme všechny, ale zkracujeme pomocí \uv{et al.} (za jména píšeme \uv{and others}).
    \end{itemize}
    \item \textbf{Název}
    \begin{itemize}
        \item Název je tisknut \textit{kurzívou}. Příliš dlouhý název můžeme zkrátit a chybějící slova nahradit pomocí \ldots{} (tři tečky).
    \end{itemize}
    \item \textbf{Typ nosiče}
    \begin{itemize}
        \item Typem nosiče rozumíme jak nosič, na kterém je citovaný materiál dostupný (např. [DVD]), tak to, v jaké formě je dostupný (např. [Braillovo písmo]). Tuto informaci uvedeme do hranatých závorek. Pro materiály online uvádíme [online].
    \end{itemize}
    \item \textbf{Vydání}
    \begin{itemize}
        \item  Vydání by mělo být uvedeno včetně symbolů, např. \uv{2. vyd., (reedice)}. Pokud se jedná o aktualizovanou verzi, z citace by vždy mělo být zřejmé, z jaké verze bylo citováno.
    \end{itemize}
    \item \textbf{Místo vydání, vydavatel, datum vydání}
    \begin{itemize}
        \item Pokud místo není uvedeno přímo v citovaném dokumentu, ale je známé, doplňujeme ho v hranatých závorkách.
        \item Jako vydavatel by měla být uvedena ta osoba nebo organizace, která je v citovaném dokumentu uvedena nejvýrazněji.
        \item Datum publikování by vždy mělo být uvedeno. U papírových publikací stačí rok, u elektronických informačních zdrojů je nutné uvádět i měsíc a den.
    \end{itemize}
    \item \textbf{Číslování v rámci jednotky nebo stránkování}
    \begin{itemize}
        \item Citace by měla identifikovat tu část citovaného dokumentu, ze které se cituje.
    \end{itemize}
    \item \textbf{Název a číslo edice, pokud je k dispozici}
    \item \textbf{Standardní identifikátory}
    \begin{itemize}
        \item Pokud má citovaný dokument mezinárndní standardní číslo (ISBN, ISSN, \ldots) případně DOI, které ho jednoznačně identifikuje, musí být uvedeno.
    \end{itemize}
    \item \textbf{Dostupnost, přístup nebo umístění informací}
    \begin{itemize}
        \item U elektronických zdrojů by mělo být vždy uvedeno, kde je uvedený zdroj dostupný. Uvádíme jako \uv{Dostupné z:} následované URI nebo URL.
    \end{itemize}
    \item \textbf{Dodatečné všeobecné informace}
\end{enumerate}


\chapter{Formální stránka práce}
\label{formality}

Formální požadavky na vypracování bakalářské či diplomové práce vycházejí ze směrnice rektora č.~72/2017 \cite{smernice} a ze směrnice FIT č. 7/2018 \cite{smerniceFIT}. Hodnocení formální stránky práce je důležitou součástí posudku oponenta a je proto nezbytné věnovat jí náležitou pozornost (je nutné se seznámit se směrnicemi). Další pokyny a doporučení jsou uvedeny na webu fakulty \cite{formalniBP}, \cite{formalniDP}.

Požadovaný rozsah písemné části práce bez příloh dle směrnice FIT \cite{smerniceFIT} je uveden v~tabulce \ref{rozsah}:

\begin{table}[hbt]
\centering
\caption{Požadovaný rozsah písemné části práce v normostranách}
\label{rozsah}
\begin{tabular}{|l|c|c|c|}
\hline
 & Minimální rozsah & Obvyklý rozsah & Rozsah by neměl  \\
 &  &  & přesáhnout  \\ \hline
Bakalářská práce (9 kr.) & 30 & 40--50 & 60 \\ \hline
Bakalářská práce (13 kr.) & 40 & 60--80 & 100 \\ \hline
Semestrální projekt (SEP) & 20 & 30--40 & 50 \\ \hline
Diplomová práce & 50 & 80--100 & 120 \\ \hline
\end{tabular}
\end{table}
Přibližný rozsah ve vysázených stránkách bude asi 1/2 rozsahu v normovaných stránkách. Pojem {\it normovaná stránka} se vztahuje k~posuzování objemu práce, nikoliv k~počtu vytištěných listů. Z historického hlediska jde o~počet stránek rukopisu, který se psal psacím strojem na speciální předtištěné formuláře při dodržení průměrné délky řádku 60 znaků a~při 30 řádcích na stránku rukopisu. Vzhledem k~zápisu korekturních značek se používalo řádkování 2 (ob jeden řádek). Tyto údaje (počet znaků na řádek, počet řádků a~proklad mezi nimi) se nijak nevztahují ke konečnému vytištěnému výsledku. Používají se pouze pro posouzení rozsahu. \textbf{Jednou normovanou stránkou se tedy rozumí $\mathbf{60\cdot 30 = 1800}$ znaků včetně mezer.} Obrázky zařazené do textu se započítávají do rozsahu písemné práce odhadem jako množství textu, které by ve výsledném dokumentu potisklo stejně velkou plochu.

Orientační rozsah práce lze při použití systému \LaTeX{} odhadnout sečtením velikostí zdrojových souborů práce a podělením konstantou cca 2000 (normálně bychom dělili konstantou 1800, ale ve zdrojových souborech jsou i~vyznačovací příkazy, které se do rozsahu nepočítají). Pro přesnější odhad lze pak vyextrahovat holý text z~PDF (např. metodou cut-and-paste nebo {\it Save as Text}) a~jeho velikost podělit konstantou 1800. Využít lze i program Detex\footnote{\url{https://www.ctan.org/pkg/detex}} (v operačním systému Linux je dostupný v distribučním balíčku, pro Windows jej lze nainstalovat separátně\footnote{\url{http://urchin.earth.li/~tomford/detex/}}), který ze~zdrojového textu odstraní vyznačovací příkazy a~následně lze jeho velikost rovněž podělit konstantou 1800. \bf Program Detex pro určení počtu normostran v jádru práce využívá i \texttt{Makefile} v této šabloně \rm -- volá se příkazem \verb|make normostrany|.

V Microsoft Word lze orientační počet normovaných stran zjistit pomocí funkce {\it Počet slov} v~menu {\it Nástroje}, když hodnotu {\it Znaky (včetně mezer)} vydělíte konstantou 1800. Do~rozsahu práce se započítává pouze text uvedený v~jádru práce (od úvodu po závěr). Části jako abstrakt, klíčová slova, prohlášení, obsah, literatura nebo přílohy se do rozsahu práce nepočítají. Je proto nutné nejdříve označit jádro práce a~teprve pak si nechat spočítat počet znaků. Přibližný rozsah obrázků odhadnete ručně. Podobně lze postupovat i~při použití OpenOffice či LibreOffice.


Originální text věnující se úkolu, jehož řešení tvoří jádro kvalifikační práce, musí představovat alespoň třetinu celkového rozsahu písemné zprávy. Pouhý kompilát dostupných zdrojů je nepřijatelný.

Předmětem hodnocení oponenta kvalifikační práce je především text písemné zprávy a výsledný produkt. Neúměrně velký počet stran písemné zprávy svědčí o nekvalitním zpracování tématu a oponenta zbytečně zatěžuje. U prací, u nichž rozsah písemné zprávy odpovídá objemu vykonaných prací, se samozřejmě připouští větší rozsah vysvětlujícího textu. Vysvětlení podstaty řešeného problému a použitých přístupů k řešení však nemusí narůstat lineárně s objemem prací. Při kvalitní struktuře a ucelenosti textu písemné zprávy může být rozsah textu relativně malý. Detailní popisy významných částí projektu, které mají dokumentační (nikoliv vysvětlující) charakter, mohou být uvedeny v rámci příloh, na~které se z hlavního textu odkazujeme.

Pokud se rozsah písemné zprávy bude blížit minimálnímu požadovanému rozsahu, bude oponent zvláště pečlivě hodnotit, zda jsou jednotlivé části zprávy pro pochopení práce opravdu nezbytné. Zařazení cizích textů, které s tématem vlastní práce souvisejí pouze okrajově nebo jejichž kvalita je pochybná, s cílem dosáhnout alespoň minimálního požadovaného rozsahu (např. v časovém stresu těsně před termínem odevzdání práce), může vést k citelnému zhoršení celkového hodnocení práce.

Při vkládání obrázků volte jejich rozměry tak, aby nepřesáhly oblast, do které se tiskne text (tj. okraje textu ze všech stran). Pro velké obrázky vyčleňte samostatnou stránku. Obrázky nebo tabulky o~rozměrech větších než A4 umístěte do písemné zprávy formou skládanky (tzv. skládání do Z, angl. Engineering fold -- existuje i anglický pojem Z-fold, ale při tom by byl problém s vazbou) všité do textu (jsou-li pro práci podstatné), přílohy, nebo je poskládejte do záložek na zadní desce.

Tabulky a~obrázky používají své vlastní, nezávislé číselné řady. Z toho vyplývá, že v~odkazech uvnitř textu musíme kromě čísla udat i~informaci o~tom, zda se jedná o~obrázek či tabulku (například \uv{\ldots{} {\it viz tabulka 2.7} \ldots}). Dodržování této zásady je ostatně velmi přirozené.

Pro odkazy na stránky, na čísla kapitol a~podkapitol, na čísla obrázků a~tabulek a~v~dalších podobných příkladech využíváme speciálních prostředků DTP programu, které zajistí vygenerování správného čísla i~v~případě, že se text posune díky změnám samotného textu nebo díky úpravě parametrů sazby.


Rovnice, na které se budeme v~textu odvolávat, opatříme pořadovými čísly při pravém okraji příslušného řádku. Tato pořadová čísla se píší v~kulatých závorkách. Číslování rovnic může být průběžné v~textu nebo v~jednotlivých kapitolách.

Jste-li na pochybách při sazbě matematického textu, snažte se dodržet způsob sazby definovaný systémem \LaTeX{}. Obsahuje-li Vaše práce velké množství matematických formulí, doporučujeme při jejich psaní dát přednost použití systému \LaTeX{}.

Mezeru neděláme tam, kde se spojují číslice s~písmeny v~jedno slovo nebo v~jeden znak -- například {\it 25krát}. Lomítko se píše bez mezer. Například školní rok 2019/2020.

Členicí (interpunkční) znaménka tečka, čárka, středník, dvojtečka, otazník a~vykřičník, jakož i~uzavírací závorky a~uvozovky se přimykají k~předcházejícímu slovu bez mezery. Mezera se dělá až za nimi. To se ovšem netýká desetinné čárky (nebo desetinné tečky). Otevírací závorka a~přední uvozovky se přimykají k~následujícímu slovu a~mezera se vynechává před nimi -- (takto) a~\uv{takto}.

Pro spojovací a~rozdělovací čárku a~pomlčku nepoužíváme stejný znak. Pro pomlčku je vyhrazen jiný znak (delší). V~systému TeX (\LaTeX{}) se spojovací čárka zapisuje jako jeden znak \uv{pomlčka} (například \uv{Brno-město}), pro sázení textu ve smyslu intervalu nebo dvojic, soupeřů a~podobně se ve zdrojovém textu používá dvojice znaků \uv{pomlčka} (například \uv{zápas Sparta -- Slavie}; \uv{cena 23--25 korun}), pro výrazné oddělení části věty, pro výrazné oddělení vložené věty, pro vyjádření nevyslovené myšlenky a~v~dalších situacích (viz Pravidla českého pravopisu) se používá nejdelší typ pomlčky, která se ve zdrojovém textu zapisuje jako trojice znaků \uv{pomlčka} (například \uv{Další pojem --- jakkoliv se může zdát nevýznamný --- bude neformálně definován v~následujícím odstavci.}). Při sazbě matematického mínus se při sazbě používá rovněž odlišný znak. V~systému TeX je ve zdrojovém textu zapsán jako normální mínus (tj. znak \uv{pomlčka}). Sazba v~matematickém prostředí, kdy se vzoreček uzavírá mezi dolary, zajistí vygenerování správného výstupu.

Pravidla pro psaní zkratek jsou uvedena v~Pravidlech českého pravopisu \cite{Pravidla}. I~z~jiných důvodů je vhodné, abyste tuto knihu měli po ruce.


\section{Časté chyby}
\label{chyby}

V této části jsou vybrány a zkráceny popisy častých chyb a rady, jak se jim vyhnout, převzaté z blogu prof. Herouta \cite{Herout} a ze seznamu častých chyb, který má na svém blogu dr. Szöke \cite{chyby}.

\subsection*{Drobnosti, které notoricky kazí čtení}
V této sekci se nachází přehled drobných formátovacích zel, které se často nacházejí v~kvalifikačních pracích:

\begin{itemize}
	\item{
    	\textbf{Používání spojovníku místo pomlčky} \\
    	Pomlčka je dlouhá a před ní a za ní se píše mezera. Pomlčka se často použije ve větě místo čárky: \uv{Tato kniha -- vydaná ještě před válkou -- je opravdu úžasná.} Použije se u rozsahů: \uv{strana 23--26} nebo \uv{úspěšnost 3--5\,\%}. Další použití jsou v jazykové příručce \cite{prirucka}.

Spojovník se v našich IT diplomkách vyskytuje (tedy má vyskytovat) daleko vzácněji. Třeba ve spojeních jako \uv{říkám-li}, nebo u těsného spojení podstatných jmen: \uv{Rh-faktor}, \uv{real-time}, \uv{propan-butan}.
    }
    \item{
    	\textbf{Mezery kolem závorek} \\
        Před levou závorkou je VŽDY mezera a to platí i pro kulatou i pro hranatou (při odkazování do literatury). Za pravou závorkou není mezera, pokud je za ní tečka, čárka, vykřičník nebo otazník. Na vnitřní straně závorek (ani třeba uvozovek) mezera není nikdy.
    }
\end{itemize}

\noindent Stručný přehled častých stylistických a jazykových prohřešků:

\begin{itemize}
	\item
    {
    	Anglicky správně:
        \begin{itemize}
        	\item{\uv{by using the OpenGL library}}
  			\item{\uv{in the MVC model}}
  			\item{\uv{all UI elements}}
  			\item{\uv{from the JSON string}}
  			\item{\uv{call it from C\# code}}
        \end{itemize}

        Česky nesprávně:
		\begin{itemize}
  			\item{\uv{s použitím OpenGL knihovny}}
  			\item{\uv{v MVC modelu}}
  			\item{\uv{všechny UI prvky}}
  			\item{\uv{z JSON řetězce}}
  			\item{\uv{volat ji z C\# kódu}}
		\end{itemize}

		Česky správně:
		\begin{itemize}
  			\item{\uv{s použitím knihovny OpenGL}}
  			\item{\uv{v modelu MVC}}
  			\item{\uv{všechny prvky UI} -- nebo ještě radši \uv{všechny prvky uživatelského rozhraní}}
  			\item{\uv{z řetězce ve formátu JSON}}
  			\item{\uv{volat ji z kódu v jazyce C\#}}
		\end{itemize}

    }
    \item
    {
    	\textbf{Věty bez slovesa} \\
		Každá věta ať má sloveso. V krásné literatuře se někdy pro vytvoření spádu a pro další umělecké záměry používají věty bez slovesa. Za poslední dva týdny jsem přečetl přehršel vět bez slovesa a nikdy to nesedělo, vždycky to bylo ke zlému. V diplomkách ať má každá věta své sloveso.
    }
\end{itemize}



\subsection*{Jak popisovat plovoucí objekty (obrázky/tabulky)?}
Obrázek, a podobně i tabulka, se skládá ze samotného obrázku a z jeho popisku (v \LaTeX u \texttt{\textbackslash caption}). Popisek u obrázku či tabulky slouží k tomu, aby výsledný objekt fungoval samostatně -- čtenář se na něj často podívá ještě před tím, než si přečetl text okolo, a je žádoucí, aby obrázku i tak dokázal nějak přiměřeně porozumět.

Nebál bych se titulky obrázků mít pětiřádkové i sedmiřádkové. Někdy budou stačit jen dva řádky. Někdy -- ne příliš často -- bude titulek o třech slovech ten nejsprávnější. Pokud v některé diplomce všechny titulky mají pouze tři nebo čtyři slova, prakticky jistě bude čtenář frustrován, protože obrázky nebudou dávat samostatně smysl.

Pokud je v obrázku nějaký barevný kód (některé objekty jsou třeba červeně, některé modře a ještě jiné tlustě zeleně), vysvětlení kódu patří do titulku. Pokud se obrázek skládá z částí (třeba vpravo nahoře, vlevo nahoře a dole), pojmenování a odůvodnění částí patří do titulku.

Když to povídám studentům, někteří z nich se poděsí: \uv{Ale to přesunu celé věty z~textu do~těch titulků!} Ano, přesuňte je, ničemu to nevadí. Základní vysvětlení bude přímo u obrázků a tak to má být. Podrobnější vysvětlení, zdůvodnění, interpretace zůstanou v~hlavním textu. Dvacetiřádkový titulek u obrázku nebo tabulky, to už by bylo moc, ale pětiřádkový titulek je standard a nebojte se ho.

\subsection*{Mluvnická osoba}

Použití \bf druhé osoby \rm (oslovení čtenáře Vy/You) je skoro vždycky špatně a je to protivné.

Špatně:
\begin{itemize}
  \item{\uv{Podívejte se na obrázek 5, kde najdete \ldots}}
  \item{\uv{Když budete pracovat s knihovnou X, jistě narazíte na \ldots}}
  \item{\uv{Kdybyste se chtěli přepnout do nastavení, zvolíte příslušnou položku v nabídce.}}
\end{itemize}

Správně:
\begin{itemize}
  \item{\uv{Obrázek 5 ukazuje \ldots}}
  \item{\uv{Častým jevem při používání knihovny X je \ldots}}
  \item{\uv{Do nastavení je možné vstoupit zvolením příslušné položky v nabídce.}}
\end{itemize}

Použití \bf první osoby množného čísla \rm (my/we) nemusí být úplně vždy špatně a \uv{klasická} literatura o psaní odborného textu ho dokonce někdy doporučuje jako tzv. autorský plurál:
\begin{itemize}
  \item{\uv{Zjistili jsme \ldots}},
  \item{\uv{Zaměřili jsme se \ldots}},
  \item{\uv{Navrhli jsme řešení, které \ldots}}.
\end{itemize}

Méně zkušení pisatelé se z autorského plurálu příliš často přepnou do zvláštního a nesprávného jazykového modu, který mají zažitý už z doby, kdy navštěvovali mateřskou školu. Jde totiž o způsob řeči učitelek v mateřských školách (jinak nic proti nim): \uv{Tak děti, teď si nalepíme korálek přesně doprostřed kvítečku. Vytlačíme si trošku lepidýlka, tááák, a~prstíčkem do něj korálek natlačíme, tááák.}

Není vůbec tak komické a roztomilé, když diplomant touto řečí mateřských škol popisuje svoje životní dílo: \uv{Nejdřív si musíme přilinkovat knihovnu X. Potom si vytvoříme objekty zvolených tříd a postupně je odesíláme na server. Když nám server odpoví chybovým kódem, musíme resetovat připojení.} Nepoužívejte při psaní tento jazyk, pokud nechcete, aby vaše diplomka byla hodnocena jako dílo někoho, kdo se mentálně zasekl v mateřské škole.

Použití \bf první osoby jednotného čísla \rm (já/I) je správně, pokud se jedná o \uv{subjektivní} záležitost:
\begin{itemize}
  \item{\uv{Zaměřil jsem se na \ldots}},
  \item{\uv{Vytvořil jsem \ldots}},
  \item{\uv{Naměřil jsem \ldots}},
  \item{\uv{Oslovil jsem několik respondentů \ldots}}
\end{itemize}

Nesprávné (ovšem bohužel celkem časté) je použít první osobu v popisu jevů a postupů:
\begin{itemize}
  \item{\uv{V prvním kroku algoritmu si vynuluji čítače.}}
  \item{\uv{Pokud v ukazateli mám hodnotu null, provedu alokaci nového objektu.}}
  \item{\uv{Z grafu je patrné, že mám nastavenou příliš malou velikost vyrovnávací paměti.}}
\end{itemize}

\subsection*{Další chyby}

\bf Necitujte irelevantní \uv{blbosti}: \rm Informace v Úvodu o tom, že připojení na Internet je dostupné už i na Mount Everest může působit dojmem, že je třeba to nějak potvrdit. Nicméně toto zbytečně nafukuje Literaturu o spoustu irelevantních zdrojů (vzhledem k~tématu Vaší práce) -- což se může obrátit proti Vám v posudku. Takže to tam nepište vůbec, nebo to alespoň necitujte.

\bf Zkratky a poznámky pod čarou by měly být \uv{samonosné}. Pokud máte v~textu zkratku, tak ji při prvním použití rovnou rozepište: \rm Pokud se vám zkratka opakuje třeba o 2 kapitoly dále, tak ji opět při prvním výskytu můžete rozepsat, a nebo jen dát poznámku pod čarou. Důležité je, aby pod čarou byla vypsána i ta zkratka. Pokud se jedná o zkratku nějakého \uv{slušného} sousloví, nebojte se jej přeložit též do češtiny.

\noindent Špatně:
\begin{enumerate}
  \item{large-vocabulary continuous speech recognition}
\end{enumerate}
Správně:
\begin{enumerate}
  \item{LVCSR -- large-vocabulary continuous speech recognition (rozpoznávač spojité řeči s~velkým slovníkem)}
\end{enumerate}

\bf Nevynechávejte text mezi sekcí a podsekcí: \rm Mezi názvem kapitoly a podkapitoly by neměl chybět text. Toto místo přímo volá po 1-2 odstavcích, kde vysvětlíte, o čem ta kapitola je a co se v ní čtenář dozví.

\bf Ručně dělané skicy a obrázky: \rm Obzvláště pro pracovní verzi práce je jednodušší rychle něco namalovat, vyfotit a pak psát dál. Většinou je ta skica schématu napoprvé špatně a tak zbytečně ztrácíte čas předěláváním. Do finálního textu bych byl s \uv{ruční prací} opatrný. Pro některé oponenty/přísedící to může být falešný indikátor, že jste nestíhali, a~tak jste to odbyli rukou za 5 minut.

\bf Úvod a Závěr: \rm Překlepy, nesmyslné věty, hrubky, nespisovná slova, \ldots  To vše do diplomky nepatří. Někde hluboko v práci se ten jeden jediný překlep snad ztratí. Ale úvod a~závěr čtou úplně všichni. A mít tam chyby je ostuda. Až u obhajoby budete přesvědčovat komisi, že jste odvedli velký kus práce, jak asi bude vypadat fakt, že si po sobě neumíte přečíst jednu stranu textu?

\bf Citujte obrázky: \rm Pokud si \uv{vypůjčíte} obrázek z nějakého zdroje, nepřiznáte se a~přijde se na to, máte za F. Citace odkazem do literatury na konci popisku obrázku je (asi) OK. Přesto má řada oponentů raději explicitní uvedení do závorky: (převzato z literatury [xy]).

\bf Nepřehánějte to se sekcemi: \rm Pokud má mít sekce 1-2 odstavce v samostatných podsekcích, zkuste se zamyslet, jestli by nebylo lepší to vyřešit třeba odrážkami.

\bf Popisky v obrázku: \rm Nepopisujte komplikovanější obrázky stylem Vlevo je A, nahoře je B, uprostřed vidíte C, pod tím se nachází D a vedle je XY. Výsledek bude půl strany textu. A pokud nebude obrázek hned na stejné straně, tak se čtenář ulistuje k smrti. Dopište do obrázku popisky. Obrázek se stane \uv{samonosným}.

\bf Detaily do přílohy: \rm Nesnažte se v práci za každou cenu detailně popisovat kompletní digram tříd, tabulek, objektů, \ldots{} Vložte jedno celkové schéma a popište základní komponenty. Dále se pak věnujte jádru systému (třeba ty tři označené objekty). Ty jsou klíčové, a~stojí na nich Vaše práce. Popis ostatních \uv{podpůrných} objektů (např. načítání a vykreslování dat) můžete dát do přílohy. V textu se pak jednou větou odkážete, že: \it Detailní popis všech tříd a jejich metod je v Příloze 1. \rm

\bf Nepište dlouhá souvětí: \rm Věta přes celý odstavec (na 6,5 řádku). Na třetím řádku čtenář zapomene, co četl na začátku věty. Nebojte se souvětí dělit na menší. Místo zasypání a udušení čtenáře pod hromadou slov po něm střílejte krátké věty. Dáte mu tím možnost se mezi nimi nadechnout. Nebo reprezentujte informace výčetem (\tt itemize\rm ).

\bf Nedělitelné mezery: \rm Udělat takovou typografickou hrubku jako je např. ponechání \uv{s}~na~konci řádku hned v názvu BP či DP je ostuda. Existuje něco, čemu se říká nedělitelná mezera, tedy mezera, ve které se nemůže zalomit konec řádku. V \LaTeX{}u se tato mezera označuje vlnkou \textasciitilde. Též pěkně popsáno v knize {\LaTeX} pro začátečníky \cite{Rybicka}.

\chapter{Odevzdání práce}
\label{odevzdani}

Bakalářská či diplomová práce se odevzdává v listinné a v elektronické formě, kde elektronická forma se odevzdává na přiloženém paměťovém médiu s listinnou formou práce a~současně do IS FIT. Za odevzdanou se považuje až ve chvíli, kdy jsou korektně odevzdány obě formy práce. Tato kapitola je do značné míry převzata z oficiálních pokynů na~webu \cite{formalniBP}, \cite{formalniDP}.

Ti, kteří se rozhodli utajit některé části práce, o to musí min. měsíc před odevzdáním podat žádost. Dle úpravy vysokoškolského zákona č.~111/1998 zákonem č. 137/2016, \S 47b, odstavce 4 musí být v případě odloženého zveřejnění od roku 2017 vždy odeslán výtisk úplné verze práce na MŠMT. Student tedy musí odevzdat o~jeden výtisk plné verze práce v~knihařské vazbě více. Tento 2. výtisk pak bude shodný s~prvním a bude rovněž obsahovat paměťové médium. Do obou výtisků volně vloží papír s oznámením, že se~jedná o utajovaný výtisk, na kterém bude vyznačena doba trvání překážky pro zveřejnění (které však lze odložit nejdéle na dobu 3 let) a uvedeno odůvodnění odkladu zveřejnění.

Před odevzdáním důkladně zkontrolujte checklist přiložený k této šabloně (příloha \ref{checklist}). Dále ověřte, že je práce v souladu se směrnicemi \cite{smernice} a \cite{smerniceFIT}. Podstatné detaily, na které se občas zapomíná:
\begin{itemize}
	\item \textbf{Titulní list:} Nezapomeňte nastavit správný rok (odevzdání) a ústav dle zadání.
    \item \textbf{Zadání:} Nezapomeňte na stažení elektronické verze zadání (šablona ji očekává v~souboru nazvaném \texttt{zadani.pdf}).
    \item \textbf{Prohlášení:} Nezapomeňte před odevzdáním podepsat prohlášení v obou výtiscích práce.
    \item \textbf{Bibliografické citace:} zkontrolujte, zda v textu odkazujete všechnu citovanou literaturu.
\end{itemize}

Pak práci vytiskněte a nechte si ji vyvázat. Pokud svázanou práci otevřete na libovolné stránce, okamžitě najdete chybu. Ignorujte to, je to normální. Nic není dokonalé, prostě \uv{just ship it}\footnote{\url{http://blog.igor.szoke.cz/2011/08/zkoumejte-bezte-za-bod-odkud-neni.html}} \cite{rady}.


\section{Odevzdání listinné formy práce}

Bakalářská či diplomová práce se odevzdává ve dvou výtiscích, přičemž každý z nich bude obsahovat podepsané prohlášení. Výtisk pro archivaci musí být svázán nerozebíratelným způsobem. Doporučeny jsou desky z polotvrdého papíru, tmavé (modrá, šedá), na vnitřní straně zadní desky bude vlepena obálka s CD/DVD či jiným povoleným nosičem tak, aby bylo možné jej vyjmout. Výtisk určený pro nahlédnutí před obhajobou v knihovně FIT, který bude diplomantovi po obhajobě vrácen, může být svázán i rozebíratelným způsobem (např. kroužková vazba).

Student kromě vytištěné písemné zprávy odevzdává následující součásti práce v elektronické podobě:
\begin{itemize}
  \item{písemnou zprávu ve formátu PDF (i v IS FIT),}
  \item{zdrojový tvar písemné zprávy (včetně všech náležitostí tak, aby bylo možné text diplomové práce upravit a znovu vytisknout),}
  \item{úplnou dokumentaci (návod k instalaci, uživatelskou příručku, obvodová schemata apod.),}
  \item{zdrojové texty programů (binární programy musí být přeložitelné z dodaných zdrojových textů),}
  \item{všechny programy ve spustitelné formě schopné běhu v prostředí CVT FIT. Pokud toho nelze dosáhnout (např. v CVT není nainstalován potřebný SW nebo HW nebo pokud je výsledkem práce nový hardware), po dohodě s vedoucím předvede diplomant funkční produkt vhodným způsobem oponentovi.}
\end{itemize}

Dokument z \LaTeX{}u můžete do PDF převést aplikacemi pdflatex nebo výsledné dvi programem dvipdf případně z postscriptu programem ps2pdf. Výsledný dokument by měl mít řádově max. jednotky MB, pokud dosahuje několik desítek MB, je někde koncepční chyba. Zpravidla obsahuje obrázky s neúčelně vysokým rozlišením.

Kompletní elektronická podoba musí být přiložena na nepřepisovatelném paměťovém médiu CD-R, DVD-R, DVD+R ve formátu ISO9660 (s rozšířením RockRidge a/nebo Jolliet) nebo UDF nebo paměťové kartě SD (Secure Digital) ve formátu FAT32 nebo exFAT s~nastavenou ochranou proti přepisu.

\section{Jak odevzdat práci v IS FIT}

Do IS FIT je třeba vložit kompletní text diplomové práce ve formátu PDF -- v části Registrace -- Přihlašování a odevzdávání závěrečných prací po kliknutí na příslušné zadání. Kromě dokumentu je nutno vyplnit také abstrakt a klíčová slova, obojí česky (slovensky) i~anglicky (při kopírování z \LaTeX{}u nezapomeňte nahradit nezlomitelné mezery) a vyznačit svolení ke zveřejnění. Jediný případ, kdy lze odložit zveřejnění textu práce, je z důvodu ochrany duševního vlastnictví, což musí předem schválit příslušný proděkan (jinak je jedinou možností zveřejnění ihned). Student v tomto případě odevzdává písemně i~elektronicky plnou verzi práce pro recenzi. Bez těchto náležitostí se práce nepovažuje za~odevzdanou.

Semestrální projekt se u diplomové práce rovněž odevzdává do IS FIT. U bakalářské práce se odevzdává pouze tehdy, vyžaduje-li to vedoucí práce.


\chapter{Závěr}
\label{zaver}

V tomto textu bylo uvedeno, jak začít s tvorbou bakalářské či diplomové práce, napsat abstrakt, připravit základní strukturu práce a co uvést do jednotlivých kapitol. Při tom bylo vysvětleno, že bakalářská práce je také diplomová a je třeba k ní přistupovat stejně zodpovědným způsobem. Následně byla věnována pozornost bibliografickým citacím a formální stránce práce. V předposlední kapitole jsou uvedeny důležité informace k odevzdání v listinné i v elektronické podobě.

Je třeba zdůraznit, že diplomová práce je unikátním individuálním dílem, které vzniká pod vedením zkušeného odborníka. Ať už je v této šabloně uvedeno cokoliv, závazné jsou pouze oficiální pokyny na stránkách fakulty. Pro konkrétní diplomovou práci je potřeba vždy zvažovat, co je z výše uvedeného textu relevantní a co nikoliv a řídit se především pokyny vedoucího, který rozumí dané problematice a je tak schopen poskytnout ty nejlepší rady, co lze k práci dostat.

I přes velkou snahu nikdy není možné do šablony zahrnout všechny prvky, co budou při tvorbě práce potřeba, a zaručit, že po doplnění textu, obrázků, literatury apod. bude vše v~pořádku pro všechny možné diplomové práce. Bude-li někde delší text, než se předpokládalo, a zalomí-li se na dva řádky, bude-li v literatuře položka, se kterou nebyl otestován využitý styl, a v dalších případech může být výsledek neuspokojivý a může být potřeba do~šablony zasáhnout a chybu, která se projevuje třeba jen pro jednu práci ze sta, opravit. Výsledné PDF a následně i vytištěnou papírovou verzi je tedy vždy nutné pečlivě zkontrolovat a~nespoléhat se na to, že \uv{tohle přece generuje šablona, tak to musí být správně}. Najdete-li v šabloně nějaké chyby nebo budete-li mít návrhy na její vylepěšení, napište prosím na e-mail \texttt{sablona@fit.vutbr.cz} a pomozte nám s jejím vylepšováním. Veškeré připomínky a návrhy jsou vítány.

S kontrolou výsledku může výrazně pomoci vedoucí práce. Nelze však předpokládat, že vedoucí poslední noc před odevzdáním bude sedět v práci připraven na kontrolu desítek stran textu. Proto je nutné mít vše připravené v předstihu a konzultovat průběžně. Kritický pohled vedoucího pak umožní dosažení kvalitního výsledku a aktivita, kterou uvidí, přispěje k pozitivnímu hodnocení práce z jeho strany.

Na závěr bych jménem autorů této šablony popřál všem, kteří právě vytvářejí svoje diplomové práce nebo se k jejich tvorbě připravují, úspěšné dokončení a obhájení práce.




%===============================================================================

  \fi
  
  % Kompilace po částech (viz výše, nutno odkomentovat)
  %\subfile{projekt-01-uvod-introduction}
  % ...
  %\subfile{chapters/projekt-05-conclusion}


  % Pouzita literatura / Bibliography
  % ----------------------------------------------
\ifslovak
  \makeatletter
  \def\@openbib@code{\addcontentsline{toc}{chapter}{Literatúra}}
  \makeatother
  \bibliographystyle{bib-styles/Pysny/skplain}
\else
  \ifczech
    \makeatletter
    \def\@openbib@code{\addcontentsline{toc}{chapter}{Literatura}}
    \makeatother
    \bibliographystyle{bib-styles/Pysny/czplain}
  \else 
    \makeatletter
    \def\@openbib@code{\addcontentsline{toc}{chapter}{Bibliography}}
    \makeatother
    \bibliographystyle{bib-styles/Pysny/enplain}
  %  \bibliographystyle{alpha}
  \fi
\fi
  \begin{flushleft}
  \bibliography{projekt-20-literatura-bibliography}
  \end{flushleft}

  % vynechani stranky v oboustrannem rezimu
  \iftwoside
    \cleardoublepage
  \fi

  % Prilohy / Appendices
  % ---------------------------------------------
  \appendix
\ifczech
  \renewcommand{\appendixpagename}{Přílohy}
  \renewcommand{\appendixtocname}{Přílohy}
  \renewcommand{\appendixname}{Příloha}
\fi
\ifslovak
  \renewcommand{\appendixpagename}{Prílohy}
  \renewcommand{\appendixtocname}{Prílohy}
  \renewcommand{\appendixname}{Príloha}
\fi
%  \appendixpage

% vynechani stranky v oboustrannem rezimu
%\iftwoside
%  \cleardoublepage
%\fi
  
\ifslovak
%  \section*{Zoznam príloh}
%  \addcontentsline{toc}{section}{Zoznam príloh}
\else
  \ifczech
%    \section*{Seznam příloh}
%    \addcontentsline{toc}{section}{Seznam příloh}
  \else
%    \section*{List of Appendices}
%    \addcontentsline{toc}{section}{List of Appendices}
  \fi
\fi
  \startcontents[chapters]
  \setlength{\parskip}{0pt} 
  % seznam příloh / list of appendices
  % \printcontents[chapters]{l}{0}{\setcounter{tocdepth}{2}}
  
  \ifODSAZ
    \setlength{\parskip}{0.5\bigskipamount}
  \else
    \setlength{\parskip}{0pt}
  \fi
  
  % vynechani stranky v oboustrannem rezimu
  \iftwoside
    \cleardoublepage
  \fi
  
  % Compilation piecewise (see above, it is necessary to uncomment it)
  %\subfile{projekt-30-prilohy-appendices}
  
\end{document}
